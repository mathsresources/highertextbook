% Options for packages loaded elsewhere
\PassOptionsToPackage{unicode}{hyperref}
\PassOptionsToPackage{hyphens}{url}
\documentclass[
  12pt,
  openany]{book}
\usepackage{xcolor}
\usepackage{amsmath,amssymb}
\setcounter{secnumdepth}{5}
\usepackage{iftex}
\ifPDFTeX
  \usepackage[T1]{fontenc}
  \usepackage[utf8]{inputenc}
  \usepackage{textcomp} % provide euro and other symbols
\else % if luatex or xetex
  \usepackage{unicode-math} % this also loads fontspec
  \defaultfontfeatures{Scale=MatchLowercase}
  \defaultfontfeatures[\rmfamily]{Ligatures=TeX,Scale=1}
\fi
\usepackage{lmodern}
\ifPDFTeX\else
  % xetex/luatex font selection
  \setmainfont[]{Times New Roman}
\fi
% Use upquote if available, for straight quotes in verbatim environments
\IfFileExists{upquote.sty}{\usepackage{upquote}}{}
\IfFileExists{microtype.sty}{% use microtype if available
  \usepackage[]{microtype}
  \UseMicrotypeSet[protrusion]{basicmath} % disable protrusion for tt fonts
}{}
\makeatletter
\@ifundefined{KOMAClassName}{% if non-KOMA class
  \IfFileExists{parskip.sty}{%
    \usepackage{parskip}
  }{% else
    \setlength{\parindent}{0pt}
    \setlength{\parskip}{6pt plus 2pt minus 1pt}}
}{% if KOMA class
  \KOMAoptions{parskip=half}}
\makeatother
\usepackage{longtable,booktabs,array}
\usepackage{calc} % for calculating minipage widths
% Correct order of tables after \paragraph or \subparagraph
\usepackage{etoolbox}
\makeatletter
\patchcmd\longtable{\par}{\if@noskipsec\mbox{}\fi\par}{}{}
\makeatother
% Allow footnotes in longtable head/foot
\IfFileExists{footnotehyper.sty}{\usepackage{footnotehyper}}{\usepackage{footnote}}
\makesavenoteenv{longtable}
\setlength{\emergencystretch}{3em} % prevent overfull lines
\providecommand{\tightlist}{%
  \setlength{\itemsep}{0pt}\setlength{\parskip}{0pt}}
\usepackage[]{natbib}
\bibliographystyle{apalike}
\usepackage{booktabs}
\usepackage[top=1cm,bottom=1cm,left=1.5cm,right=1.5cm,includehead,a4paper]{geometry}
\usepackage{fancyhdr}
  \pagestyle{fancy}
  \fancypagestyle{plain}{%
  \renewcommand{\headrulewidth}{0pt}%
  \fancyhf{}%
}
  \fancyhf{}
  \fancyhf[HRO,HLE]{\thepage}
  \fancyhf[HC]{\leftmark}

%% Book Structure
\usepackage{booktabs}
\usepackage{titlesec}
\usepackage{caption}
\usepackage{subcaption}

\usepackage[no-math]{fontspec}
\usepackage{lmodern}
\renewcommand{\familydefault}{lmss}
\usepackage{amsthm}
\usepackage{mathtools}
\usepackage{unicode-math}

%% Utilities
\usepackage{tcolorbox}
\usepackage{hyperref}
\usepackage{tabularx}
\usepackage{multirow}
\usepackage{array}
\usepackage{wrapfig}
\graphicspath{ {images/} }
\usepackage{exsheets}
\usepackage{setspace}
\usepackage{etoolbox}
\usepackage{enumitem}
\usepackage{graphbox,graphicx}
\usepackage{float}
\usepackage{adjustbox}
\usepackage{quotchap}
\usepackage{dirtytalk}
\usepackage{setspace}
\usepackage{multicol}
\usepackage{epigraph}
\usepackage[switch, modulo]{lineno}
\usepackage{vwcol}

%% Maths and Tikz
\usepackage{tikz}
\usepackage{tikzlings}
\usetikzlibrary{math}
\usetikzlibrary{patterns}
\usetikzlibrary{decorations.pathreplacing,angles,quotes}
\usetikzlibrary{fadings}
\usepackage{units}
\usepackage{amssymb}
\usepackage{pifont}
\usepackage{nth}
\usepackage{mathtools}
\usepackage[draft]{tikzpeople}
\usepackage{gensymb}
\usepackage{bm}
\usepackage{pgfplots}
\usepackage{pgfmath}
\pgfplotsset{compat=1.8}
\usetikzlibrary{shapes.geometric}

%% Defining Stuff
\newcommand{\PreserveBackslash}[1]{\let\temp=\\#1\let\\=\temp}
\newcolumntype{C}[1]{>{\PreserveBackslash\centering}p{#1}}
\newcolumntype{R}[1]{>{\PreserveBackslash\raggedleft}p{#1}}
\newcolumntype{L}[1]{>{\PreserveBackslash\raggedright}p{#1}}
\newenvironment{cols}[1][]{}{}
\newenvironment{col}[1]{\begin{minipage}{#1}\ignorespaces}{%
\end{minipage}
\ifhmode\unskip\fi
\aftergroup\useignorespacesandallpars}

\def\useignorespacesandallpars#1\ignorespaces\fi{%
#1\fi\ignorespacesandallpars}
\makeatletter
\def\ignorespacesandallpars{%
  \@ifnextchar\par
    {\expandafter\ignorespacesandallpars\@gobble}%
    {}%
}
\makeatother
\tolerance=1
\emergencystretch=\maxdimen
\hyphenpenalty=10000
\hbadness=10000
\setlist{topsep=0pt, leftmargin=*}
\setlist[2]{itemsep=5pt}


\setlength{\parindent}{0pt}
\setlength{\parskip}{1em}

\tcbuselibrary{skins}


\usepackage[dvipsnames]{xcolor}

\usepackage{cancel}
\newcommand\hcancel[2][black]{\setbox0=\hbox{$#2$}%
\rlap{\raisebox{.45\ht0}{\textcolor{#1}{\rule{\wd0}{1pt}}}}#2}

\usepackage{sectsty}
\usepackage{fontawesome5}

\def\doubleunderline#1{\underline{\underline{#1}}}
\AtBeginDocument{\frontmatter}
\usepackage{bookmark}
\IfFileExists{xurl.sty}{\usepackage{xurl}}{} % add URL line breaks if available
\urlstyle{same}
\hypersetup{
  hidelinks,
  pdfcreator={LaTeX via pandoc}}

\author{}
\date{\vspace{-2.5em}}

\begin{document}

{
\setcounter{tocdepth}{1}
\tableofcontents
}
\mainmatter

\chapter*{Introduction}\label{introduction}
\addcontentsline{toc}{chapter}{Introduction}

Intro

\chapterfont{\color{white}}

\chapter{The Straight Line}\label{the-straight-line}

\vspace{-12cm}
\begin{center}
\begin{tikzpicture}
\draw[white] (10,5) circle (0.01);
\draw[WildStrawberry,rounded corners,very thick] (1,0.2) rectangle (5,1.8);
\draw[WildStrawberry,very thick] (5,1) -- (6,1);
\draw[WildStrawberry,fill=WildStrawberry,rounded corners] (6,1.8) rectangle (18.2,0.2);
\node[white] at (12.1,1) {\Huge{\textsc{The Straight Line}}};
\node at (3,1) {\Large{\textsc{Chapter 1}}};
\end{tikzpicture}
\end{center}

\section{Straight Line Equations}\label{straight-line-equations}

\newpage

\subsection*{Exercise}\label{exercise}
\addcontentsline{toc}{subsection}{Exercise}

\newpage

\section{Parallel and Perpendicular Lines}\label{parallel-and-perpendicular-lines}

\newpage

\subsection*{Exercise}\label{exercise-1}
\addcontentsline{toc}{subsection}{Exercise}

\newpage

\section{\texorpdfstring{The Angle a Line Makes with the \(x\)-Axis}{The Angle a Line Makes with the x-Axis}}\label{the-angle-a-line-makes-with-the-x-axis}

\newpage

\subsection*{Exercise}\label{exercise-2}
\addcontentsline{toc}{subsection}{Exercise}

\newpage

\section{Collinearity}\label{collinearity}

\newpage

\subsection*{Exercise}\label{exercise-3}
\addcontentsline{toc}{subsection}{Exercise}

\newpage

\section{The Midpoint}\label{the-midpoint}

\newpage

\subsection*{Exercise}\label{exercise-4}
\addcontentsline{toc}{subsection}{Exercise}

\newpage

\section{Altitudes of a Triangle}\label{altitudes-of-a-triangle}

\newpage

\subsection*{Exercise}\label{exercise-5}
\addcontentsline{toc}{subsection}{Exercise}

\newpage

\section{Medians of a Triangle}\label{medians-of-a-triangle}

\newpage

\subsection*{Exercise}\label{exercise-6}
\addcontentsline{toc}{subsection}{Exercise}

\newpage

\section{Perpendicular Bisectors}\label{perpendicular-bisectors}

\newpage

\subsection*{Exercise}\label{exercise-7}
\addcontentsline{toc}{subsection}{Exercise}

\newpage

\section*{Review Exercise}\label{review-exercise}
\addcontentsline{toc}{section}{Review Exercise}

\chapterfont{\color{white}}

\chapter{Recurrence Relations}\label{recurrence-relations}

\vspace{-12cm}
\begin{center}
\begin{tikzpicture}
\draw[white] (10,5) circle (0.01);
\draw[PineGreen,rounded corners,very thick] (1,0.2) rectangle (5,1.8);
\draw[PineGreen,very thick] (5,1) -- (6,1);
\draw[PineGreen,fill=PineGreen,rounded corners] (6,1.8) rectangle (18.2,0.2);
\node[white] at (12.1,1) {\Huge{\textsc{Recurrence Relations}}};
\node at (3,1) {\Large{\textsc{Chapter 2}}};
\end{tikzpicture}
\end{center}

\section{Finding Terms in a Sequence}\label{finding-terms-in-a-sequence}

\newpage

\subsection*{Exercise}\label{exercise-8}
\addcontentsline{toc}{subsection}{Exercise}

\newpage

\section{Determining a Recurrence Relation}\label{determining-a-recurrence-relation}

\newpage

\subsection*{Exercise}\label{exercise-9}
\addcontentsline{toc}{subsection}{Exercise}

\newpage

\section{The Limit of a Recurrence Relation}\label{the-limit-of-a-recurrence-relation}

\newpage

\subsection*{Exercise}\label{exercise-10}
\addcontentsline{toc}{subsection}{Exercise}

\newpage

\section*{Review Exercise}\label{review-exercise-1}
\addcontentsline{toc}{section}{Review Exercise}

\chapterfont{\color{white}}

\chapter{Differentiation I}\label{differentiation-i}

\vspace{-12cm}
\begin{center}
\begin{tikzpicture}
\draw[white] (10,5) circle (0.01);
\draw[NavyBlue,rounded corners,very thick] (1,0.2) rectangle (5,1.8);
\draw[NavyBlue,very thick] (5,1) -- (6,1);
\draw[NavyBlue,fill=NavyBlue,rounded corners] (6,1.8) rectangle (18.2,0.2);
\node[white] at (12.1,1) {\Huge{\textsc{Differentiation I}}};
\node at (3,1) {\Large{\textsc{Chapter 3}}};
\end{tikzpicture}
\end{center}

\section{\texorpdfstring{The Derivative of \(ax^n\)}{The Derivative of ax\^{}n}}\label{the-derivative-of-axn}

\newpage

\subsection*{Exercise}\label{exercise-11}
\addcontentsline{toc}{subsection}{Exercise}

\newpage

\section{Differentiable Form}\label{differentiable-form}

\newpage

\subsection*{Exercise}\label{exercise-12}
\addcontentsline{toc}{subsection}{Exercise}

\newpage

\section{Leibniz Notation}\label{leibniz-notation}

\newpage

\subsection*{Exercise}\label{exercise-13}
\addcontentsline{toc}{subsection}{Exercise}

\newpage

\section{The Gradient of a Tangent to a Curve}\label{the-gradient-of-a-tangent-to-a-curve}

\newpage

\subsection*{Exercise}\label{exercise-14}
\addcontentsline{toc}{subsection}{Exercise}

\newpage

\section{The Equation of a Tangent to a Curve}\label{the-equation-of-a-tangent-to-a-curve}

\newpage

\subsection*{Exercise}\label{exercise-15}
\addcontentsline{toc}{subsection}{Exercise}

\newpage

\section{The Rate of Change of a Function}\label{the-rate-of-change-of-a-function}

\newpage

\subsection*{Exercise}\label{exercise-16}
\addcontentsline{toc}{subsection}{Exercise}

\newpage

\section*{Review Exercise}\label{review-exercise-2}
\addcontentsline{toc}{section}{Review Exercise}

\chapterfont{\color{white}}

\chapter{Quadratic Theory}\label{quadratic-theory}

\vspace{-12cm}
\begin{center}
\begin{tikzpicture}
\draw[white] (10,5) circle (0.01);
\draw[PineGreen,rounded corners,very thick] (1,0.2) rectangle (5,1.8);
\draw[PineGreen,very thick] (5,1) -- (6,1);
\draw[PineGreen,fill=PineGreen,rounded corners] (6,1.8) rectangle (18.2,0.2);
\node[white] at (12.1,1) {\Huge{\textsc{Quadratic Theory}}};
\node at (3,1) {\Large{\textsc{Chapter 4}}};
\end{tikzpicture}
\end{center}

\section{Solving Quadratic Inequations}\label{solving-quadratic-inequations}

\newpage

\subsection*{Exercise}\label{exercise-17}
\addcontentsline{toc}{subsection}{Exercise}

\newpage

\section{Completing the Square}\label{completing-the-square}

\newpage

\subsection*{Exercise}\label{exercise-18}
\addcontentsline{toc}{subsection}{Exercise}

\newpage

\section{The Disciminant}\label{the-disciminant}

\newpage

\subsection*{Exercise}\label{exercise-19}
\addcontentsline{toc}{subsection}{Exercise}

\newpage

\section{The Intersection of a Line and a Curve}\label{the-intersection-of-a-line-and-a-curve}

\newpage

\subsection*{Exercise}\label{exercise-20}
\addcontentsline{toc}{subsection}{Exercise}

\newpage

\section{Tangency}\label{tangency}

\newpage

\subsection*{Exercise}\label{exercise-21}
\addcontentsline{toc}{subsection}{Exercise}

\newpage

\section*{Review Exercise}\label{review-exercise-3}
\addcontentsline{toc}{section}{Review Exercise}

\chapterfont{\color{white}}

\chapter{Sets and Functions}\label{sets-and-functions}

\vspace{-12cm}
\begin{center}
\begin{tikzpicture}
\draw[white] (10,5) circle (0.01);
\draw[PineGreen,rounded corners,very thick] (1,0.2) rectangle (5,1.8);
\draw[PineGreen,very thick] (5,1) -- (6,1);
\draw[PineGreen,fill=PineGreen,rounded corners] (6,1.8) rectangle (18.2,0.2);
\node[white] at (12.1,1) {\Huge{\textsc{Sets and Functions}}};
\node at (3,1) {\Large{\textsc{Chapter 5}}};
\end{tikzpicture}
\end{center}

\section{The Domain of a Function}\label{the-domain-of-a-function}

\newpage

\subsection*{Exercise}\label{exercise-22}
\addcontentsline{toc}{subsection}{Exercise}

\newpage

\section{The Range of a Function}\label{the-range-of-a-function}

\newpage

\subsection*{Exercise}\label{exercise-23}
\addcontentsline{toc}{subsection}{Exercise}

\newpage

\section{Composite Functions}\label{composite-functions}

\newpage

\subsection*{Exercise}\label{exercise-24}
\addcontentsline{toc}{subsection}{Exercise}

\newpage

\section{Inverse Functions}\label{inverse-functions}

\newpage

\subsection*{Exercise}\label{exercise-25}
\addcontentsline{toc}{subsection}{Exercise}

\newpage

\section*{Review Exercise}\label{review-exercise-4}
\addcontentsline{toc}{section}{Review Exercise}

\chapterfont{\color{white}}

\chapter{Trigonometry}\label{trigonometry}

\vspace{-12cm}
\begin{center}
\begin{tikzpicture}
\draw[white] (10,5) circle (0.01);
\draw[Mulberry,rounded corners,very thick] (1,0.2) rectangle (5,1.8);
\draw[Mulberry,very thick] (5,1) -- (6,1);
\draw[Mulberry,fill=Mulberry,rounded corners] (6,1.8) rectangle (18.2,0.2);
\node[white] at (12.1,1) {\Huge{\textsc{Trigonometry}}};
\node at (3,1) {\Large{\textsc{Chapter 6}}};
\end{tikzpicture}
\end{center}

\section{Exact Values}\label{exact-values}

\newpage

\subsection*{Exercise}\label{exercise-26}
\addcontentsline{toc}{subsection}{Exercise}

\newpage

\section{Solving Trig Equations using Exact Values}\label{solving-trig-equations-using-exact-values}

\newpage

\subsection*{Exercise}\label{exercise-27}
\addcontentsline{toc}{subsection}{Exercise}

\newpage

\section{Radians}\label{radians}

\newpage

\subsection*{Exercise}\label{exercise-28}
\addcontentsline{toc}{subsection}{Exercise}

\newpage

\section{Solving Trig Equations and Radians}\label{solving-trig-equations-and-radians}

\newpage

\subsection*{Exercise}\label{exercise-29}
\addcontentsline{toc}{subsection}{Exercise}

\newpage

\section{Solving Phase Angle Equations}\label{solving-phase-angle-equations}

\newpage

\subsection*{Exercise}\label{exercise-30}
\addcontentsline{toc}{subsection}{Exercise}

\newpage

\section{Solving Multiple Angle Equations}\label{solving-multiple-angle-equations}

\newpage

\subsection*{Exercise}\label{exercise-31}
\addcontentsline{toc}{subsection}{Exercise}

\newpage

\section{Review Exercise}\label{review-exercise-5}

\chapterfont{\color{white}}

\chapter{Graph Transformations}\label{graph-transformations}

\vspace{-10cm}
\begin{center}
\begin{tikzpicture}
\draw[white] (10,5) circle (0.01);
\draw[PineGreen,rounded corners,very thick] (1,0.2) rectangle (5,1.8);
\draw[PineGreen,very thick] (5,1) -- (6,1);
\draw[PineGreen,fill=PineGreen,rounded corners] (6,1.8) rectangle (18.2,0.2);
\node[white] at (12.1,1) {\Huge{\textsc{Graph Transformations}}};
\node at (3,1) {\Large{\textsc{Chapter 7}}};
\end{tikzpicture}
\end{center}

\section{Translations}\label{translations}

\newpage

\subsection*{Exercise}\label{exercise-32}
\addcontentsline{toc}{subsection}{Exercise}

\newpage

\section{Stretches and Compressions}\label{stretches-and-compressions}

\newpage

\subsection*{Exercise}\label{exercise-33}
\addcontentsline{toc}{subsection}{Exercise}

\newpage

\section{Reflections}\label{reflections}

\newpage

\subsection*{Exercise}\label{exercise-34}
\addcontentsline{toc}{subsection}{Exercise}

\newpage

\section{Composite Transformations}\label{composite-transformations}

\newpage

\subsection*{Exercise}\label{exercise-35}
\addcontentsline{toc}{subsection}{Exercise}

\newpage

\section*{Review Exercise}\label{review-exercise-6}
\addcontentsline{toc}{section}{Review Exercise}

\chapterfont{\color{white}}

\chapter{Vectors}\label{vectors}

\vspace{-12cm}
\begin{center}
\begin{tikzpicture}
\draw[white] (10,5) circle (0.01);
\draw[WildStrawberry,rounded corners,very thick] (1,0.2) rectangle (5,1.8);
\draw[WildStrawberry,very thick] (5,1) -- (6,1);
\draw[WildStrawberry,fill=WildStrawberry,rounded corners] (6,1.8) rectangle (18.2,0.2);
\node[white] at (12.1,1) {\Huge{\textsc{Vectors}}};
\node at (3,1) {\Large{\textsc{Chapter 8}}};
\end{tikzpicture}
\end{center}

\section{Parallel Vectors}\label{parallel-vectors}

\newpage

\subsection*{Exercise}\label{exercise-36}
\addcontentsline{toc}{subsection}{Exercise}

\newpage

\section{Collinearity using Vectors}\label{collinearity-using-vectors}

\newpage

\subsection*{Exercise}\label{exercise-37}
\addcontentsline{toc}{subsection}{Exercise}

\newpage

\section{The Division of a Line in a Ratio}\label{the-division-of-a-line-in-a-ratio}

\newpage

\subsection*{Exercise}\label{exercise-38}
\addcontentsline{toc}{subsection}{Exercise}

\newpage

\section{\texorpdfstring{Unit Vectors \(i\), \(j\) and \(k\)}{Unit Vectors i, j and k}}\label{unit-vectors-i-j-and-k}

\newpage

\subsection*{Exercise}\label{exercise-39}
\addcontentsline{toc}{subsection}{Exercise}

\newpage

\section{The Scalar Product}\label{the-scalar-product}

\newpage

\subsection*{Exercise}\label{exercise-40}
\addcontentsline{toc}{subsection}{Exercise}

\newpage

\section{The Angle Between Two Vectors}\label{the-angle-between-two-vectors}

\newpage

\subsection*{Exercise}\label{exercise-41}
\addcontentsline{toc}{subsection}{Exercise}

\newpage

\section{Properties of the Scalar Product}\label{properties-of-the-scalar-product}

\newpage

\subsection*{Exercise}\label{exercise-42}
\addcontentsline{toc}{subsection}{Exercise}

\newpage

\section{Vector Pathways}\label{vector-pathways}

\newpage

\subsection*{Exercise}\label{exercise-43}
\addcontentsline{toc}{subsection}{Exercise}

\newpage

\section*{Review Exercise}\label{review-exercise-7}
\addcontentsline{toc}{section}{Review Exercise}

\chapterfont{\color{white}}

\chapter{Differentiation II}\label{differentiation-ii}

\vspace{-12cm}
\begin{center}
\begin{tikzpicture}
\draw[white] (10,5) circle (0.01);
\draw[NavyBlue,rounded corners,very thick] (1,0.2) rectangle (5,1.8);
\draw[NavyBlue,very thick] (5,1) -- (6,1);
\draw[NavyBlue,fill=NavyBlue,rounded corners] (6,1.8) rectangle (18.2,0.2);
\node[white] at (12.1,1) {\Huge{\textsc{Differentiation II}}};
\node at (3,1) {\Large{\textsc{Chapter 9}}};
\end{tikzpicture}
\end{center}

\section{Increasing and Decreasing Functions}\label{increasing-and-decreasing-functions}

A function can be said to be \textit{increasing} for values of \(x\) for which a tangent to the curve has a \textit{positive gradient}, and \textit{decreasing} where a tangent to the curve has a \textit{negative gradient}.

\vspace{-0.3cm}

\begin{multicols}{2}
\begin{center}
\begin{tikzpicture}
    \draw[thick,double,green!70!black,domain=-3.7:-1.852,smooth] plot (\x,0.1*\x*\x*\x-\x+1);
    \draw[thick,double,red!70!black,domain=-1.8:1.8,smooth] plot (\x,0.1*\x*\x*\x-\x+1);
    \draw[thick,double,green!70!black,domain=1.852:3.7,smooth] plot (\x,0.1*\x*\x*\x-\x+1);
    \draw[domain=-3.7:3.7,smooth] plot (\x,0.1*\x*\x*\x-\x+1) node[above] {$y=f(x)$};
    \node[rotate=62,green!70!black] at (-3.5,1.2) {increasing};
    \node[rotate=60,green!70!black] at (2.7,1.2) {increasing};
    \node[rotate=-43,red!70!black] at (0.4,1.2) {decreasing};
    \node[green!70!black] at (-3.2,-1) {$f'(x)>0$};
    \node[red!70!black] at (0,-1) {$f'(x)<0$};
    \node[green!70!black] at (3.2,-1) {$f'(x)>0$};
    \draw[thin,dashed] (-1.826,2.5) -- (-1.826,-1.2);
    \draw[thin,dashed] (1.826,2.5) -- (1.826,-1.2);
    \node at (-1.826,-1.45) {$a$};
    \node at (1.826,-1.45) {$b$};
\end{tikzpicture}
\end{center}
\vfill\null
\columnbreak
\null\vfill
\begin{center}
    \begin{tcolorbox}[center,width=7cm,colback=red!5,colframe=red!70!black]
        A function $f(x)$ is:\\[0.2em]
        \begin{itemize}
            \item Increasing when $f'(x)>0$.
            \item Decreasing when $f'(x)<0$.
        \end{itemize}
    \end{tcolorbox}
\end{center}
\vfill\null

\end{multicols}

\vspace{-1cm}

Function \(f\) above is \textit{increasing} for \(x<a,\text{ }x>b\) and \textit{decreasing} for \(a<x<b\).

\begin{tcolorbox}[title=Example 9.1.1,colback=blue!1, colframe=blue!70!black!70!]
Part of the graph of a function $f(x)$ is below. State the values of $x$ for which $f(x)$ is:\\[0.1em]

\begin{enumerate}
    \item[(a)] Increasing.
    \item[(b)] Decreasing.
\end{enumerate}

\vspace{-2.3cm}
\begin{center}
\begin{tikzpicture}
    \draw[-stealth] (-4,0.6) -- (4,0.6) node[right] {$x$};
    \draw[-stealth] (-1,-1.2) -- (-1,2.7) node[above] {$y$};
    \node[anchor=north east] at (-0.5,0.4) {$0$};
    \draw[domain=-4:4,smooth] plot (\x,0.08*\x*\x*\x-0.8*\x+0.8) node[above] {$y=f(x)$};
    \draw[fill=black] (1.82,-0.086*2) circle (0.05) node[below] {$(6,-2)$};
    \draw[fill=black] (-1.82,0.886*2) circle (0.05) node[above] {$(-2,3)$};
\end{tikzpicture}
\end{center}

\tcblower

\begin{enumerate}
    \item[(a)] The function $f(x)$ is increasing for $x<-2$ and $x>6$.
    \item[(b)] The function $f(x)$ is decreasing for $-2<x<6$.
\end{enumerate}

\end{tcolorbox}

\vspace{0.5cm}

Evaluating \(f'(x)\) can show whether a function is increasing or decreasing at that point:

\vspace{0.5cm}

\begin{tcolorbox}[title=Example 9.1.2,colback=blue!1, colframe=blue!70!black!70!]
Determine whether the function $f(x)=5x^2-6\sqrt{x}^3$ is increasing or decreasing when $x=4$.

\tcblower

\vspace{-0.5cm}
\begin{center}
    \begin{tikzpicture}
        \draw[blue!1] (-8.5,0) -- (8.5,0);
        \node[below] at (0,0){
        \begin{minipage}{8cm}
        \begin{align*}
            f(x)&=5x^2-6x^{\frac{3}{2}}\\
            f'(x)&=10x-9x^{\frac{1}{2}}\\
            &=10x-9\sqrt{x}\\
            f'(4)&=10(4)-9\sqrt{4}\\
            &=22
        \end{align*}
        \end{minipage}
        };
        \node[below,blue!70!black!70!] at (5,0.01){
        \begin{minipage}{6cm}
        \begin{align*}
            \longleftarrow & \text{ prepare to differentiate}\\
            \longleftarrow & \text{ differentiate}\\[0.1em]
            \longleftarrow & \text{ write in radical form}\\[0.3em]
            \longleftarrow & \text{ substitute}\\[0.1em]
            \longleftarrow & \text{ evaluate}\\
        \end{align*}
        \end{minipage}
        };
    \end{tikzpicture}
\end{center}

\vspace{-0.5cm}

$$\text{Since }f'(4)>0,\text{ the function }f\text{ is increasing when }x=4.$$

\end{tcolorbox}

\pagebreak

The range of values of \(x\) for which a function is increasing or decreasing can also be determined:

\begin{tcolorbox}[title=Example 9.1.3,colback=blue!1, colframe=blue!70!black!70!]
Determine the range of values of $x$ for which the function $f(x)=2x^2-8x+11$ is increasing.

\tcblower

$$\text{Increasing}\implies f'(x)>0$$

\vspace{-0.5cm}

\begin{center}
    \begin{tikzpicture}
        \draw[blue!1] (-8.5,0) -- (8.5,0);
        \node[below] at (0,0){
        \begin{minipage}{8cm}
        \begin{align*}
    f'(x)=4x-8&\\
    4x-8&>0\\
    4x&>8\\
    x&>2
\end{align*}
        \end{minipage}
        };
        \node[below,blue!70!black!70!] at (5,0.01){
        \begin{minipage}{6cm}
        \begin{align*}
            \longleftarrow & \text{ differentiate}\\
            \longleftarrow & \text{ apply condition}\\
            \\
            \longleftarrow & \text{ solve}
        \end{align*}
        \end{minipage}
        };
    \end{tikzpicture}
\end{center}

\vspace{-0.5cm}

$$\text{So }f'(x)\text{ is increasing when }x>2$$

\end{tcolorbox}

Quadratic inequations may arise when finding ranges, which require a sketch when solving.

\begin{tcolorbox}[title=Example 9.1.4,colback=blue!1, colframe=blue!70!black!70!]
Determine the range of values of $x$ for which $y=\frac{2}{3}x^3+3x^2-8x+7$ is decreasing.

\tcblower

\vspace{-0.3cm}

\begin{center}
    \begin{tikzpicture}
        \draw[blue!1] (-8.5,0) -- (8.5,0);
        \node[below] at (-5,0){
        \begin{minipage}{6cm}
        \centering
        $\text{Decreasing}\implies \dfrac{\text{d}y}{\text{d}x}<0$
        \begin{align*}
            \frac{\text{d}y}{\text{d}x}=2x^2+6x-8&\\
            2x^2+6x-8&<0\\
            x^2+3x-4&<0
        \end{align*}
        \end{minipage}
        };
        \node[below] at (5,0){
        \begin{minipage}{6cm}
        \begin{align*}
            \textcolor{blue!70!black!70!}{\text{  Consider roots:}}&\\
            x^2+3x-4&=0\\
            (x+4)(x-1)&=0\\
            x=-4,x=1
        \end{align*}
        \end{minipage}
        };
        \draw[blue!70!black!70!,-stealth] (-2.5,-3.2) -- (2.5,-1.2);
        \node[blue!70!black!70!,rotate=21.5,above] at (0,-2.2) {quadratic inequation};
    \end{tikzpicture}
\end{center}




\begin{center}
\begin{tikzpicture}
    \draw[-stealth] (-2,0) -- (2,0) node[right] {$x$};
    \draw[domain=-2:2,smooth] plot (\x,0.3*\x*\x-0.5);
    \node[below left] at (-1.29,0) {\scriptsize{$-4$}};
    \node[below right] at (1.29,0) {\scriptsize{$1$}};
    \draw[blue!70!black!70!,thick,double distance = 0.03cm,domain=-1.29:1.29,smooth] plot (\x,0.3*\x*\x-0.5);
\end{tikzpicture}
\end{center}

\vspace{-0.5cm}
$$\text{So }f'(x)\text{ is decreasing when }-4<x<1$$

\end{tcolorbox}

A function can be said to be \textbf{strictly increasing} if \(f'(x)>0\) across its domain.

A function can be said to be \textbf{strictly decreasing} if \(f'(x)<0\) across its domain.

\begin{tcolorbox}[title=Example 9.1.5,colback=blue!1, colframe=blue!70!black!70!]
Show that $f(x)=\frac{1}{3}x^3-4x^2+21x-10$ is strictly increasing for all values of $x$.\\[0.5em]
\textit{Note that }$x^2-8x+1$\textit{ can be expressed as }$(x-4)^2+5$.

\tcblower

\vspace{-0.5cm}

\begin{center}
    \begin{tikzpicture}
        \draw[blue!1] (-8.5,0) -- (8.5,0);
        \node[below] at (-1.5,0){
        \begin{minipage}{8cm}
        \begin{align*}
            f'(x)&=x^2-8x+21\\
            &=(x-4)^2+5
        \end{align*}
        \vspace{-0.75cm}
        \begin{gather*}
                \therefore\text{ minimum value of }f'(x)\text{ is }5\\[0.2em]
                \text{Since }5>0, \text{ }f'(x)>0\text{ for all }x,\\[0.2em]
                \text{so the function }f\text{ is strictly increasing.}
        \end{gather*}
        \end{minipage}
        };
        \node[below,blue!70!black!70!] at (5.5,0.01){
        \begin{minipage}{6cm}
        \begin{align*}
            \longleftarrow & \text{ differentiate}\\
            \longleftarrow & \text{ complete the square}\\[0.45cm]
            \longleftarrow & \text{ state minimum value of }f'\\[0.07cm]
            \longleftarrow & \text{ compare to zero}\\[0.1cm]
            \longleftarrow & \text{ conclusion about }f
        \end{align*}
        \end{minipage}
        };
    \end{tikzpicture}
\end{center}

\end{tcolorbox}

\pagebreak

\section{Stationary Points}\label{stationary-points}

\vspace{-0.5cm}

A \textit{stationary point} can be found where a tangent to a curve has a \textit{gradient of zero}, or:

\vspace{-1.2cm}

\begin{multicols}{2}

\null\vfill

\begin{center}
\begin{tikzpicture}
    \draw[-stealth] (-4,1) -- (4,1) node[right] {$x$};
    \draw[-stealth] (-1,-0.5) -- (-1,3) node[above] {$y$};
    \node[anchor=north east] at (-1,1) {$0$};
    \draw[domain=-3.5:3.5,smooth] plot (\x,0.1*\x*\x*\x-\x+1) node[above] {$y=f(x)$};
    \draw[fill=black] (1.82574,-0.21716) circle (0.05) node[below] {$(6,-2)$};
    \draw[fill=black] (-1.82574,2.21716) circle (0.05) node[above] {$(-2,3)$};
\end{tikzpicture}
\end{center}

\vfill\null

\columnbreak

\null\vfill

\begin{center}
\begin{tcolorbox}[colback=red!5,colframe=red!70!black]
\textbf{Stationary points} occur where $\dfrac{\text{d}y}{\text{d}x}=0$
\end{tcolorbox}
\end{center}

The graph of $y=f(x)$ has stationary points at $(-2,3)$ and $(6,-2)$.

\vfill\null

\end{multicols}

\vspace{-0.7cm}
\begin{tcolorbox}[title=Example 9.2.1,colback=blue!1, colframe=blue!70!black!70!]
Find the coordinates of the stationary points on the curve $y=x^3-\frac{9}{2}x^2+6x$.
\tcblower

\begin{center}
    \begin{tikzpicture}
    \draw[blue!1] (-8.5,0) -- (8.5,0);
    \node[below,blue!70!black!70!] at (-4.25,0) {Find $x$-coordinate(s)};
        \node[below] at (-4.25,-0.65) {
        \begin{minipage}{8cm}
        \centering
        $\text{Stationary points occur where }\dfrac{\text{d}y}{\text{d}x}=0$\\[-1em]
            \begin{align*}
                \dfrac{\text{d}y}{\text{d}x}=3x^2-9x+6&&\\
                3x^2-9x+6&=0\\
                x^2-3x+2&=0\\
                (x-1)(x-2)&=0\\
                x=1,x=2&\\
            \end{align*}
        \end{minipage}
        };
        \draw[blue!70!black!70!] (0,0) -- (0,-5.5);
        \node[below,blue!70!black!70!] at (4.25,0) {Find $y$-coordinates};
        \node[below] at (4.25,-0.2) {
        \begin{minipage}{8cm}
            \begin{align*}
                \text{When }x=1,\quad y&=(1)^3-\frac{9}{2}(1)^2+6(1)&\\
                &=\frac{5}{2}\\
                &\\
                \text{When }x=2,\quad y&=(2)^3-\frac{9}{2}(2)^2+6(2)&\\
                &=2
            \end{align*}
        \end{minipage}
        };
        \node at (0,-6) {$\therefore \text{stationary points occur at }\left(1,\frac{5}{2}\right)\text{ and }\left(1,2\right)$};
    \end{tikzpicture}
\end{center}

\end{tcolorbox}

Either Leibniz notation \(\left(\dfrac{\text{d}y}{\text{d}x}\right)\) or function notation \((f'(x))\) may be used depending on the context.

\textbf{Determining the nature} of a stationary point means to find out \textit{what type} it is. There are four:

\begin{multicols}{4}
\begin{center}
    \begin{tikzpicture}[scale=0.9]
        \draw[domain=-1.5:1.5,smooth] plot (\x,\x*\x);
        \draw[fill] (0,0) circle (0.05);
    \end{tikzpicture}\\
    minimum\\
    turning point
    \begin{tikzpicture}[scale=0.9]
        \draw[domain=-1.5:1.5,smooth] plot (\x,-\x*\x);
        \draw[fill] (0,0) circle (0.05);
    \end{tikzpicture}\\
    maximum\\
    turning point
    \begin{tikzpicture}[scale=0.9]
        \draw[domain=-1.5:1.5,smooth] plot (\x,0.33*\x*\x*\x);
        \draw[fill] (0,0) circle (0.05);
    \end{tikzpicture}\\
    rising point\\
    of inflection
    \begin{tikzpicture}[scale=0.9]
        \draw[domain=-1.5:1.5,smooth] plot (\x,-0.33*\x*\x*\x);
        \draw[fill] (0,0) circle (0.05);
    \end{tikzpicture}\\
    falling point\\
    of inflection
\end{center}
\end{multicols}

\vspace{-0.5cm}

The \textbf{second derivative}, \(f''(x)\) or \(\dfrac{\text{d}^2 y}{\text{d}x^2}\), is obtained by \textbf{differentiating the first derivative}.

For example:
\vspace{-0.5cm}
\begin{align*}
    f(x)&=4x^3-3x^2+7x-1\\
    f'(x)&=12x^2-6x+7\\
    f''(x)&=24x-6
\end{align*}
\vspace{-0.7cm}

\begin{center}
\begin{tcolorbox}[colback=red!5,width=15cm,colframe=red!70!black]
Stationary point $(a,b)$ is a:
\vspace{-0.5cm}
\begin{itemize}[leftmargin=6cm]
    \item \textbf{maximum turning point} if $f''(a)<0$
    \item \textbf{minimum turning point} if $f''(a)>0$
\end{itemize}
\end{tcolorbox}
\end{center}

\pagebreak

\begin{tcolorbox}[title=Example 9.2.2,colback=blue!1, colframe=blue!70!black!70!]
Determine the coordinates and nature of the stationary points of $f(x)=x^3-12x^2+36x$.
\tcblower


\begin{center}
    \begin{tikzpicture}
    \draw[blue!70!black!70!] (-3,0) -- (-3,-5);
    \draw[blue!70!black!70!] (3,0) -- (3,-5);
    \draw[blue!1] (-8.5,0) -- (8.5,0);
    \node[below,blue!70!black!70!] at (-6,0) {Find $x$-coordinate(s)};
        \node[below] at (-6,-0.65) {
        \begin{minipage}{5cm}
        \centering
        $\text{SPs occur where }f'(x)=0$\\[-1.5em]
            \begin{align*}
                f'(x)=3x^2-24x+36&\\
                3x^2-24x+36&=0\\
                x^2-8x+12&=0\\
                (x-2)(x-6)&=0\\
                x=2,x=6&\\
            \end{align*}
        \end{minipage}
        };
        \node[below,blue!70!black!70!] at (0,0) {Find $y$-coordinates};
        \node[below] at (0,-0.3) {
        \begin{minipage}{5cm}
            \begin{align*}
                f(2)&=(2)(2-6)^2\\
                &=32\\
                &\\
                f(6)&=(6)(6-6)^2\\
                &=0
            \end{align*}
        \end{minipage}
        };
        \node[below,blue!70!black!70!] at (6,0) {Determine Nature};
        \node[below] at (6,-0.05) {
        \begin{minipage}{5cm}
            \begin{align*}
                f''(x)&=6x-24\\[0.5em]
                f''(2)&=6(2)-24\\
                &=-12\\
                f''(2)&<0\therefore \text{max TP}\\[0.5em]
                f''(6)&=6(6)-24\\
                &=12\\
                f''(6)&>0\therefore \text{min TP}
            \end{align*}
        \end{minipage}
        };
        \node at (0,-6.5) {$\therefore \text{maximum turning point at }(2,32)\text{ and minimum turning point at }(6,0)$};
    \end{tikzpicture}
\end{center}

\end{tcolorbox}

If the second derivative is 0, a \textbf{nature table} can be used instead, where \(\frac{\text{d}y}{\text{d}x}\) is evaluated (\(>0\) or \(<0\)) for values of \(x\) either side of a stationary point, to visualise its nature. For stationary point \((a,b)\):
\vspace{-0.3cm}

\begin{multicols}{4}

\begin{tikzpicture}
    \draw[white] (-0.1,-1.5) rectangle (4.1,3.1);
    \draw (0,0) grid (4,3);
    \node[rotate=45] at (0.5,0.5) {slope};
    \node at (0.5,1.5) {$\frac{\text{d}y}{\text{d}x}$};
    \node at (0.5,2.5) {$x$};
    \node at (2.5,2.5) {$a$};
    \draw[-stealth] (1.2,2.5) -- (1.8,2.5);
    \draw[-stealth] (3.2,2.5) -- (3.8,2.5);
    \node at (2.5,1.5) {$0$};
    \draw[ultra thick] (2.2,0.5) -- (2.8,0.5);
    \node[align=center,below, text width = 3.5cm] at (2,0) {minimum turning point};
    \node at (1.5,1.5) {$-ve$};
    \node at (3.5,1.5) {$+ve$};
    \draw[ultra thick] (1.2,0.8) -- (1.8,0.5);
    \draw[ultra thick] (3.2,0.5) -- (3.8,0.8);
\end{tikzpicture}

\begin{tikzpicture}
    \draw[white] (-0.1,-1.5) rectangle (4.1,3.1);
    \draw (0,0) grid (4,3);
    \node[rotate=45] at (0.5,0.5) {slope};
    \node at (0.5,1.5) {$\frac{\text{d}y}{\text{d}x}$};
    \node at (0.5,2.5) {$x$};
    \node at (2.5,2.5) {$a$};
    \draw[-stealth] (1.2,2.5) -- (1.8,2.5);
    \draw[-stealth] (3.2,2.5) -- (3.8,2.5);
    \node at (2.5,1.5) {$0$};
    \draw[ultra thick] (2.2,0.5) -- (2.8,0.5);
    \node[align=center,below, text width = 3.5cm] at (2,0) {maximum turning point};
    \node at (1.5,1.5) {$+ve$};
    \node at (3.5,1.5) {$-ve$};
    \draw[ultra thick] (1.2,0.2) -- (1.8,0.5);
    \draw[ultra thick] (3.2,0.5) -- (3.8,0.2);
\end{tikzpicture}

\begin{tikzpicture}
    \draw[white] (-0.1,-1.5) rectangle (4.1,3.1);
    \draw (0,0) grid (4,3);
    \node[rotate=45] at (0.5,0.5) {slope};
    \node at (0.5,1.5) {$\frac{\text{d}y}{\text{d}x}$};
    \node at (0.5,2.5) {$x$};
    \node at (2.5,2.5) {$a$};
    \draw[-stealth] (1.2,2.5) -- (1.8,2.5);
    \draw[-stealth] (3.2,2.5) -- (3.8,2.5);
    \node at (2.5,1.5) {$0$};
    \draw[ultra thick] (2.2,0.5) -- (2.8,0.5);
    \node[align=center,below, text width = 3.5cm] at (2,0) {rising point of inflection};
    \node at (1.5,1.5) {$+ve$};
    \node at (3.5,1.5) {$+ve$};
    \draw[ultra thick] (1.2,0.2) -- (1.8,0.5);
    \draw[ultra thick] (3.2,0.5) -- (3.8,0.8);
\end{tikzpicture}

\begin{tikzpicture}
    \draw[white] (-0.1,-1.5) rectangle (4.1,3.1);
    \draw (0,0) grid (4,3);
    \node[rotate=45] at (0.5,0.5) {slope};
    \node at (0.5,1.5) {$\frac{\text{d}y}{\text{d}x}$};
    \node at (0.5,2.5) {$x$};
    \node at (2.5,2.5) {$a$};
    \draw[-stealth] (1.2,2.5) -- (1.8,2.5);
    \draw[-stealth] (3.2,2.5) -- (3.8,2.5);
    \node at (2.5,1.5) {$0$};
    \draw[ultra thick] (2.2,0.5) -- (2.8,0.5);
    \node[align=center,below, text width = 3.5cm] at (2,0) {falling point of inflection};
    \node at (1.5,1.5) {$-ve$};
    \node at (3.5,1.5) {$-ve$};
    \draw[ultra thick] (1.2,0.8) -- (1.8,0.5);
    \draw[ultra thick] (3.2,0.5) -- (3.8,0.2);
\end{tikzpicture}
    
\end{multicols}
\vspace{-1cm}

\begin{tcolorbox}[title=Example 9.2.3,colback=blue!1, colframe=blue!70!black!70!]
Find the coordinates and nature of the stationary point on the curve $y=13+6x^2-12x-x^3$.
\tcblower

\begin{center}
    \begin{tikzpicture}
    \draw[blue!70!black!70!] (-3,0) -- (-3,-5);
    \draw[blue!70!black!70!] (3,0) -- (3,-5);
    \draw[blue!1] (-8.5,0) -- (8.5,0);
    \node[below,blue!70!black!70!] at (-6,0) {Find $x$-coordinate(s)};
        \node[below] at (-6,-0.65) {
        \begin{minipage}{5cm}
        \centering
        $\text{SPs occur where }\frac{\text{d}y}{\text{d}x}=0$\\[-1.5em]
            \begin{align*}
                \frac{\text{d}y}{\text{d}x}=12x-12-3x^2&\\
                12x-12x-3x^2&=0\\
                4x-4-x^2&=0\\
                x^2-4x-4&=0\\
                (x-2)(x-2)&=0\\
                x=2&\\
            \end{align*}
        \end{minipage}
        };
        \node[below,blue!70!black!70!] at (0,-0.05) {\text{Find }$y$\text{-coordinate}};
        \node[below] at (0,-0.75) {
        \begin{minipage}{5cm}
            \centering
            When $x=2$,
            \begin{align*}
                y&=13+6(2)^2-12(2)-(2)^3\\
                &=13+24-24-8\\
                &=5
            \end{align*}
        \end{minipage}
        };
        \node[below,blue!70!black!70!] at (6,-0.05) {Determine Nature};
        \node[below] at (6,-0.15) {
        \begin{minipage}{5cm}
            \centering
            \begin{equation*}
                \dfrac{\text{d}^2 y}{\text{d}x^2}=-12-6x
            \end{equation*}
            When $x=2$,
            \begin{align*}
                \dfrac{\text{d}^2 y}{\text{d}x^2}&=-12-6(2)\\
                &=0
            \end{align*}
            $\therefore \text{nature table}$
        \end{minipage}
        };
    \end{tikzpicture}
\end{center}

\vspace{-2.5cm}
\begin{center}
\begin{tikzpicture}
    \draw[white] (-0.1,-1.5) rectangle (4.1,3.1);
    \draw (0,0) grid (4,3);
    \node[rotate=45] at (0.5,0.5) {slope};
    \node at (0.5,1.5) {$\frac{\text{d}y}{\text{d}x}$};
    \node at (0.5,2.5) {$x$};
    \node at (2.5,2.5) {$2$};
    \draw[-stealth] (1.2,2.5) -- (1.8,2.5);
    \draw[-stealth] (3.2,2.5) -- (3.8,2.5);
    \node at (1.5,2.75) {\scriptsize{$1$}};
    \node at (3.5,2.75) {\scriptsize{$3$}};
    \node at (2.5,1.5) {$0$};
    \draw[ultra thick] (2.2,0.5) -- (2.8,0.5);
    \node at (1.5,1.5) {$-ve$};
    \node at (3.5,1.5) {$-ve$};
    \draw[ultra thick] (1.2,0.8) -- (1.8,0.5);
    \draw[ultra thick] (3.2,0.5) -- (3.8,0.2);
    \node[below] at (2,-0.5) {$\therefore$falling point of inflection at $(2,5)$};
\end{tikzpicture}
\end{center}

\end{tcolorbox}

\pagebreak

\section{Closed Intervals}\label{closed-intervals}

Given a smooth, continuous function across a \textit{closed interval}, finding the coordinates and nature of stationary points can allow the maximum and minimum values to be determined.

\begin{multicols}{2}
\begin{center}
\begin{tikzpicture}
    \draw[domain=-3.4:3.7,smooth,dashed] plot (\x,0.08*\x*\x*\x-0.6*\x+1) node[above] {$y=f(x)$};
    \draw[domain=-2.8:3.5,smooth] plot (\x,0.08*\x*\x*\x-0.6*\x+1);
    \draw[blue!70!black!70!] (-2.8,2.7) -- (-2.8,-0.4) node[black,below] {$a$};
    \draw[blue!70!black!70!] (3.5,2.7) -- (3.5,-0.4) node[black,below] {$b$};
    \draw[fill] (-1.58,1.63) circle (0.05);
    \draw[fill] (1.58,0.37) circle (0.05);
    \draw[fill] (-2.8,0.92) circle (0.05);
    \draw[fill] (3.5,2.33) circle (0.05);
    \draw[stealth-] (1.08,0.37) -- (0.28,0.37) node[left] {min value};
    \draw[stealth-] (3.3,2.33) -- (2.5,2.33) node[left] {max value};
\end{tikzpicture}
\end{center}

\columnbreak

\begin{center}
    \begin{tcolorbox}[center,width=9.5cm,colback=red!5,colframe=red!70!black]
        Maximum and minimum values can occur at:\\
        \begin{itemize}
            \item Turning points
            \item \textit{Bounds} of the interval
        \end{itemize}
    \end{tcolorbox}
\end{center}
\vspace{-0.8cm}
For $f(x)$, the maximum value occurs when $x=b$ and the minimum at the minimum turning point.

\end{multicols}

\vspace{-0.5cm}

\begin{tcolorbox}[title=Example 9.3.1,colback=blue!1, colframe=blue!70!black!70!]
Function $f(x)=\frac{2}{3}x^{3}-x^{2}-4x-\frac{1}{3}$ has two stationary points, and part of the graph of $y=f(x)$ is shown below. Find the maximum and minimum values of $f$ on the interval $-3\leqslant x \leqslant 3$.

\begin{center}
\begin{tikzpicture}[scale=0.8]
    \draw[-stealth] (-3,0) -- (4,0) node[right] {$x$};
    \draw[-stealth] (0,-1.5) -- (0,1) node[above] {$y$};
    \node[anchor=north east] at (0,0) {$0$};
    \draw[domain=-2.5:3.7,smooth] plot (\x,2*0.06666*\x*\x*\x-2*0.1*\x*\x-2*0.44*\x-2*0.03333) node[above] {$y=f(x)$};
    \draw[fill=black] (-1.06528,0.48265) circle (0.05) node[above] {$(-1,2)$};
    \draw[fill=black] (2.06538,-1.56274) circle (0.05) node[below] {$(2,-7)$};
\end{tikzpicture}
\end{center}

\tcblower

\textcolor{blue!70!black!70!}{Check $y$-coordinates at the lower and upper bounds:}

\vspace{0.1cm}
\hspace{14cm}
\begin{tikzpicture}[remember picture,overlay]
    \draw[-stealth] (-3,0) -- (2.6,0) node[right] {$x$};
    \draw[-stealth] (0,-1.5) -- (0,0.6) node[above] {$y$};
    \node[anchor=north east] at (0,0) {$0$};
    \draw[domain=-0.769*3.2:0.769*3.2,smooth,dashed] plot (\x,2.197*2*0.06666*\x*\x*\x-1.69*2*0.1*\x*\x-1.3*2*0.44*\x-2*0.03333);
    \draw[domain=-0.769*3:0.769*3,smooth] plot (\x,2.197*2*0.06666*\x*\x*\x-1.69*2*0.1*\x*\x-1.3*2*0.44*\x-2*0.03333);
    \node[right] at (-2.8*0.769,-3.2) {$y=f(x)$};
    \draw[fill=black] (-0.769*1.06528,0.48265) circle (0.05) node[above] {$(-1,2)$};
    \draw[fill=black] (0.769*2.06538,-1.56274) circle (0.05) node[below] {$(2,-7)$};
    \draw[fill=black] (-0.769*3,-2.9) circle (0.05);
    \draw[fill=black] (0.769*3,-0.88) circle (0.05);
    \draw[thin,blue!70!black!70!] (-3*0.769,0.6) -- (-3*0.769,-3.7) node[black, below] {\small{$-3$}};
    \draw[thin,blue!70!black!70!] (3*0.769,0.6) -- (3*0.769,-3.7) node[black, below] {\small{$3$}};
\end{tikzpicture}

$\text{When }x=-3,\quad y=\frac{2}{3}(-3)^3-(-3)^2-4(-3)-\frac{1}{3}=-\frac{46}{3}$\\[0.1em]

$\text{When }x=3,\quad y=\frac{2}{3}(3)^3-(3)^2-4(3)-\frac{1}{3}=-\frac{10}{3}$

\begin{flalign*}
    \therefore & \text{ maximum value of }f\text{ is }2,\text{ which occurs when }x=-1&\\
    & \text{ minimum value of }f\text{ is }-\frac{46}{3}, \text{ which occurs when }x=-3
\end{flalign*}

\end{tcolorbox}

Where a function is \textit{strictly increasing} or \textit{strictly decreasing} in a closed interval, its maximum and minimum values can \textit{only} be located at the bounds of the interval.

\begin{tcolorbox}[title=Example 9.3.2,colback=blue!1, colframe=blue!70!black!70!,height=5.7cm]
Given $y=x^{3}+\frac{1}{2}x^{2}-\frac{7}{2}$ is strictly increasing, find the greatest and least values of $y$ in the interval $-1\leqslant x\leqslant 3$.

\tcblower

$\text{When }x=-1,\quad y=(-2)^{3}+\frac{1}{2}(-1)^{2}-\frac{7}{2}=-4$\\[0.05em]

$\text{When }x=3,\quad y=(3)^{3}+\frac{1}{2}(3)^{2}-\frac{7}{2}=28$\hspace{3.5cm}
\begin{tikzpicture}[remember picture,overlay]
    \draw[-stealth] (-1.2,0) -- (3.5,0) node[right] {$x$};
    \draw[-stealth] (0,-0.8) -- (0,0.04*30) node[above] {$y$};
    \draw[domain=-1:3,smooth] plot (\x,0.04*\x*\x*\x+0.04*0.5*\x*\x-0.04*3.5) node[right] {$y=x^{3}+\frac{1}{2}x^{2}-\frac{7}{2}$};
    \draw[thin,blue!70!black!70!] (-1,0.04*35) -- (-1,-0.8) node[black, below] {\small{$-1$}};
    \draw[thin,blue!70!black!70!] (3,0.04*35) -- (3,-0.8) node[black, below] {\small{$3$}};
    \node[anchor=north east] at (0,-0.05) {\scriptsize{$0$}};
\end{tikzpicture}

\vspace{-0.5cm}

\begin{flalign*}
\therefore & \text{ the maximum value of }y\text{ is }28&\\
\text{ and}&\text{ the minimum value of }y\text{ is }-4    
\end{flalign*}


\end{tcolorbox}

\pagebreak

\section*{9.4 Maximum and Minimum Values}

If a smooth, continuous function has only one stationary point within its domain, and it is a \textit{maximum turning point}, the \textit{greatest value} of the function will be located at the turning point. The corresponding will be true for the \textit{least} value of a function with only a \textit{minimum} turning point.\textbackslash{[}0.1em{]}

\begin{tcolorbox}[title=Example 9.4.1,colback=blue!1, colframe=blue!70!black!70!]
Determine the value of $x$ for which the function $f$, defined by $f(x)=x+\frac{1}{x}$, $x>0$, has its minimum value.

\tcblower

\begin{center}
    \begin{tikzpicture}
    \draw[blue!1] (-8.5,0) -- (8.5,0);
    \node[below,blue!70!black!70!] at (-4.25,0) {Find $x$-coordinate(s)};
        \node[below] at (-4.25,-0.65) {
        \begin{minipage}{8cm}
        \centering
        $\text{Stationary points occur where }f'(x)=0$\\[-1em]
            \begin{align*}
                f(x)=x+x^{-1}&\\
                f'(x)=1-x^{-2}&\\
                1-x^{-2}&=0\\
                1-\frac{1}{x^2}&=0\\
                x^2-1&=0\\
                x^2&=1\\
                x&=1\textcolor{blue!70!black!70!}{\text{ (Since }x>0\text{)}}
            \end{align*}
        \end{minipage}
        };
        \draw[blue!70!black!70!] (0,0) -- (0,-6.5);
        \node[below,blue!70!black!70!] at (4.25,0) {Determine Nature};
        \node[below] at (4.25,-0.05) {
        \begin{minipage}{8cm}
            \centering
            \begin{align*}
                f''(x)&=2x^{-3}\\
                &=\frac{2}{x^3}\\[0.2cm]
                f''(1)&=\frac{2}{1^3}\\[0.1cm]
                &=2
            \end{align*}
            Since $f''(1)>0$, min TP when $x=1$\\[0.4cm]
            $\therefore$ $f$ is minimised when $x=1$
        \end{minipage}
        };
    \end{tikzpicture}
\end{center}

\end{tcolorbox}

\vspace{0.1cm}

When using variables other than \(x\) and \(y\) (and function \(f\)), care should be taken with notation.

\vspace{0.1cm}

\begin{tcolorbox}[title=Example 9.4.2,colback=blue!1, colframe=blue!70!black!70!]
Determine the maximum value of $P$ given $P=6t-t^3$, where $t>0$, and the value of $t$ for which it occurs.

\tcblower

\begin{center}
    \begin{tikzpicture}
    \draw[blue!1] (-8.5,0) -- (8.5,0);
    \node[below,blue!70!black!70!] at (-4.25,0) {Find $x$-coordinate(s)};
        \node[below] at (-4.25,-0.65) {
        \begin{minipage}{8cm}
        \centering
        $\text{Stationary points occur where }\frac{\text{d}P}{\text{d}t}=0$\\[-1em]
            \begin{align*}
                    \frac{\text{d}P}{\text{d}t}=6-3t^2&\\[0.2em]
                    6-3t^2&=0\\
                    6&=3t^2\\
                    2&=t^2\\
                    \sqrt{2}&=t     
            \end{align*}
        \end{minipage}
        };
        \draw[blue!70!black!70!] (0,0) -- (0,-5.5);
        \node[below,blue!70!black!70!] at (4.25,0) {Determine Nature};
        \node[below] at (4.25,-0.05) {
        \begin{minipage}{8cm}
            \centering
            \begin{align*}
                \frac{\text{d}^2 P}{\text{d}t^2}&=-6t\\[0.5em]
                \text{When }t=\sqrt{2},\quad \frac{\text{d}^2 P}{\text{d}t^2}&=-6\sqrt{2}\\[0.5em]
                \frac{\text{d}^2P}{\text{d}t^2}<0\therefore&\text{ max TP when }t=\sqrt{2}\\[0.5em]
            \end{align*}
            \textcolor{blue!70!black!70!}{Maximum Value}\\[0.5em]
            When $t=\sqrt{2}$, $P=6(\sqrt{2})-\sqrt{2}^3=4\sqrt{2}$\\[0.5em]
            $\therefore$ max value of $P$ is $4\sqrt{2}$, when $t=\sqrt{2}$
        \end{minipage}
        };
    \end{tikzpicture}
\end{center}

\end{tcolorbox}

If a function has a single stationary point within an interval, which is a maximum turning point, and the \textit{minimum} value of the function is required, then it must be located at one of the bounds, and the value of the function at each must be checked.

\pagebreak

\section{Optimisation}\label{optimisation}

When a mathematical equation is used to model some real-world variables, finding maximum or minimum values on an interval is often desired. This is referred to as \textit{optimisation}.

\begin{tcolorbox}[title=Example 9.5.1,colback=blue!1, colframe=blue!70!black!70!]
A company has determined that a mathematical equation can model the amount of revenue in thousands of pounds, $R$, which may be earned by setting the price for a new product as $x$ pounds.
    \vspace{-0.5cm}
    \begin{multicols}{2}
    \begin{center}
        \begin{tikzpicture}[scale=0.5]
            \draw[-stealth] (-0.5,0) -- (9.5,0) node[right] {$x$};
            \draw[-stealth] (0,-0.5) -- (0,5) node[above] {$R$}; 
            \node[below left] {$0$};
            \draw[smooth,domain=0:8] plot (\x,0.01*8*\x*\x*\x-0.01*\x*\x*\x*\x);
            \node[below] at (8,0) {$8$};
        \end{tikzpicture}
    \end{center}
    The diagram illustrates the model used by the company, which is:
    $$R(x)=8x^3-x^4\text{ for }0< x \leqslant 8$$
    Find the value of $x$ which gives the maximum revenue for the product.
    \end{multicols}

\tcblower


\textcolor{blue!70!black!70!}{\text{Stationary Points}} occur where $R'(x)=0$\hspace{1cm}\textcolor{blue!70!black!70!}{Determine Nature}
\vspace{-0.2cm}
\begin{align*}
    R'(x)=24x^2-4x^3&&\hspace{2cm}R''(x)&=48x-12x^2\\
    24x^2-4x^3&=0&R''(6)&=48(6)-12(6)^2\\
    6x^2-x^3&=0&&=-144\\
    x^2(6-x)&=0&R''(6)&<0\therefore \text{maximum revenue when }x=6\\
    x=6&\text{ since }x>0
\end{align*}

\end{tcolorbox}

\begin{tcolorbox}[title=Example 9.5.2,colback=blue!1, colframe=blue!70!black!70!]
A chicken is aiming to cross a road. It can fly faster than it can walk, but doing so tires it.
\vspace{-0.2cm}
    \begin{center}
        \begin{tikzpicture}[scale=0.9]
            \draw[fill=black!30] (1,0.5) -- (8,0.5) -- (9,1.5) -- (2,1.5) -- cycle;
            \chicken[scale=0.5,xshift=0.5cm];
            \chicken[baby=yellow!95!black,scale=0.5,xshift=15cm,yshift=3cm];
            \draw[ultra thick,dashed,yellow] (1.6,1) -- (8.4,1);
        \end{tikzpicture}
    \end{center}
    \vspace{-0.2cm}
    If it flies for $x$ metres and walks the rest of the way then the amount of time $T$ it will take to reach its destination is given, in seconds, by:
    \vspace{-0.3cm}
    $$T(x)=x^2+\frac{16}{x}$$
    Determine the value of $x$ which minimises the chicken's time to get to the other side.

\tcblower

\textcolor{blue!70!black!70!}{\text{Stationary Points}} occur where $T'(x)=0$
\vspace{-0.1cm}
\begin{align*}
    T(x)=x^2+16x^{-1}&&&\textcolor{blue!70!black!70!}{Determine Nature}\\
    T'(x)=2x-16x^{-2}&&T''(x)&=2+32x^{-3}\\
    2x-16x^{-2}&=0&&=2+\frac{32}{x^3}\\
    2x-\frac{16}{x^2}&=0&T''(2)&=2+\frac{32}{2^3}\\
    2x^3-16&=0&&=2+4\\
    x^3-8&=0&&=6\\
    x^3&=8&T''(2)&>0\\
    x&=2&\therefore T\text{ is }&\text{minimised when }x=2
\end{align*}

\end{tcolorbox}

\pagebreak

It can be asked for the equation used in the mathematical model to be \textit{shown} to be correct. Establishing the equation will require an amount of problem-solving. A common context involves the volume and surface area of a 3D object.

\begin{tcolorbox}[title=Example 9.5.3,colback=blue!1, colframe=blue!70!black!70!]
A solid cuboid measures $x\text{ cm}$ by $x\text{ cm}$ by $h\text{ cm}$. The volume of this cuboid is $125\text{ cm}^3$.
    \begin{multicols}{2}
    \begin{center}
    \tikzmath{\length = 2; \percbreadth=2; \breadth = 0.5*\percbreadth; \height = 4; \x=0; \y=0;}
    \begin{tikzpicture}[scale=1.1]
    \draw[fill=blue!10] (\x,\y) -- (\x+\length,\y) -- (\x+\length+0.866*\breadth,\y+0.5*\breadth) -- (\x+\length+0.866*\breadth,\y+0.5*\breadth+\height) -- (\x+0.866*\breadth,\y+0.5*\breadth+\height) -- (\x,\y+\height) -- cycle;
    \draw (\x,\y) -- (\x+\length,\y) -- (\x+\length+0.866*\breadth,\y+0.5*\breadth) -- (\x+0.866*\breadth,\y+0.5*\breadth) -- cycle;
    \draw (\x,\y) -- (\x,\y+\height);
    \draw (\x+\length,\y) -- (\x+\length,\y+\height);
    \draw (\x+\length+0.866*\breadth,\y+0.5*\breadth) -- (\x+\length+0.866*\breadth,\y+0.5*\breadth+\height);
    \draw (\x+0.866*\breadth,\y+0.5*\breadth) -- (\x+0.866*\breadth,\y+0.5*\breadth+\height);
    \draw (\x,\y+\height) -- (\x+\length,\y+\height) -- (\x+\length+0.866*\breadth,\y+0.5*\breadth+\height) -- (\x+0.866*\breadth,\y+0.5*\breadth+\height) -- cycle;
    \draw[stealth-stealth] (\x,\y-0.14) -- (\x+\length,\y-0.14) node[anchor=north] at (\x+0.5*\length,-0.2) {$x$ cm};
    \draw[stealth-stealth] (\x+\length+0.1,\y-0.1) -- (\x+\length+0.866*\breadth+0.1,\y+0.5*\breadth-0.1) node[anchor=north west] at (\x+\length+0.433*\breadth,\y+0.25*\breadth) {$x$ cm};
    \draw[stealth-stealth] (\x+\length+0.866*\breadth+0.14,\y+0.5*\breadth) -- (\x+\length+0.866*\breadth+0.14,\y+0.5*\breadth+\height) node[anchor=west] at (\x+\length+0.866*\breadth+0.14,\y+0.5*\breadth+0.5*\height) {$h$ cm};
    \end{tikzpicture}
    \end{center}
    \begin{enumerate}
        \item[(a)] Show that the surface area of the cuboid can be given by: 
        $$A(x)=2x^2+\dfrac{500}{x}$$
        \item[(b)] Find the value of $x$ such that the surface area is minimised, and find the minimum surface area.
    \end{enumerate}
    \end{multicols} 

\end{tcolorbox}

It is recommended to first form an equation for the surface area, \(A\), in terms of \(x\) \textit{and} \(h\). By then finding the relationship between \(x\) and \(h\), substitution allows \(h\) to be eliminated.

\begin{tcolorbox}[colback=blue!1, colframe=blue!70!black!70!]

(a) 
\vspace{-0.9cm}
\begin{align*}
    &\textcolor{blue!70!black!70!}{\text{Surface Area}}&&\textcolor{blue!70!black!70!}{\text{Volume}}\\
    A&=2\times x^2+4\times xh& V&=L\times B\times H\\
    &=2x^2+4xh& 125&=x^2h\\
    &=2x^2+4x\left(\frac{125}{x^2}\right)&\frac{125}{x^2}&=h\\
    &=2x^2+\frac{500}{x}\text{ as required}
\end{align*}

\end{tcolorbox}

It should be noted that part (b) can be answered in full even if part (a) is not completed.

\begin{tcolorbox}[colback=blue!1, colframe=blue!70!black!70!]

(b) 
\textcolor{blue!70!black!70!}{\text{Stationary Points}} occur where $A'(x)=0$
\vspace{-0.1cm}
\begin{align*}
    A(x)=2x^2+500x^{-1}&&&\textcolor{blue!70!black!70!}{\text{Determine Nature}}\\
    A'(x)=4x-500x^{-2}&&A''(x)&=4+1000x^{-3}\\
    4x-500x^{-2}&=0&&=4+\frac{1000}{x^3}\\
    4x-\frac{500}{x^2}&=0&A''(5)&=4+\frac{1000}{5^3}\\
    4x^3-500&=0&&=4+8\\
    x^3-125&=0&&=12\\
    x^3&=125&A''(5)&>0\\
    x&=5&\therefore A\text{ is }&\text{minimised when }x=5\\
    &&&\textcolor{blue!70!black!70!}{\text{Minimum Area}}\\
    &&A(5)&=2(5)^2+\frac{500}{5^2}\\
    &&&=70\text{cm}^3
\end{align*}

\end{tcolorbox}

\pagebreak

\section{The Graph of the Derivative}\label{the-graph-of-the-derivative}

Part of the graph of \(f(x)\) for function \(f\) is shown below.

\begin{center}
\begin{tikzpicture}
    \draw[white] (-8,2.5) -- (9,2.5);
    \node[anchor=north east] at (1,0) {$0$};
    \draw[smooth,domain=-4:5.8] plot (\x,0.01*\x*\x*\x*\x-0.02*\x*\x*\x-0.2*\x*\x+0.8) node[above] {$f(x)$};
    \draw[-stealth] (-4,0) -- (6.3,0) node[right] {$x$};
    \draw[-stealth] (1,-1.5) -- (1,2) node[above] {$f(x)$};
    \draw[fill] (-2.5,0.25313) circle (0.06cm);
    \draw[fill] (0,0.8) circle (0.06cm);
    \draw[fill] (4,-1.12) circle (0.06cm);
    \draw[dotted] (-2.5,0.25313) -- (-2.5,0) node[below] {$-7$};
    \draw[dotted] (0,0.8) -- (0,0) node[below] {$-2$};
    \draw[dotted] (4,-1.12) -- (4,0) node[above] {$6$};
\end{tikzpicture}
\end{center}

The graph of the \textit{derived function}, \(f'(x)\), plots the \textit{rate of change} of \(f\) against values of \(x\).\textbackslash{} Where \(f(x)\) is \textit{increasing}, \(f'(x)\) has a positive value, \(f(x)\) is \textit{decreasing}, \(f'(x)\) has a negative value

\begin{center}
\begin{tikzpicture}
    \draw[white] (-8,2.5) -- (9,2.5);
    \node[anchor=north east] at (1,0) {$0$};
    \draw[thick,double,red,smooth,domain=-4:-2.6] plot (\x,0.01*\x*\x*\x*\x-0.02*\x*\x*\x-0.2*\x*\x+0.8);
    \draw[thick,double,green,smooth,domain=-2.4:-0.1] plot (\x,0.01*\x*\x*\x*\x-0.02*\x*\x*\x-0.2*\x*\x+0.8);
    \draw[thick,double,red,smooth,domain=0.1:3.9] plot (\x,0.01*\x*\x*\x*\x-0.02*\x*\x*\x-0.2*\x*\x+0.8);
    \draw[thick,double,green,smooth,domain=4.1:5.8] plot (\x,0.01*\x*\x*\x*\x-0.02*\x*\x*\x-0.2*\x*\x+0.8);
    \draw[smooth,domain=-4:5.8] plot (\x,0.01*\x*\x*\x*\x-0.02*\x*\x*\x-0.2*\x*\x+0.8) node[above] {$f(x)$};
    \draw[-stealth] (-4,0) -- (6.3,0) node[right] {$x$};
    \draw[-stealth] (1,-1.5) -- (1,2) node[above] {$f(x)$};
    \draw[fill] (-2.5,0.25313) circle (0.06cm);
    \draw[fill] (0,0.8) circle (0.06cm);
    \draw[fill] (4,-1.12) circle (0.06cm);
    \draw[dotted] (-2.5,0.25313) -- (-2.5,0) node[below] {$-7$};
    \draw[dotted] (0,0.8) -- (0,0) node[below] {$-2$};
    \draw[dotted] (4,-1.12) -- (4,0) node[above] {$6$};
    \node[green!70!black] at (-1.25,-1.5) {$f'(x)>0$};
    \node[red!70!black] at (-3.5,-1.5) {$f'(x)<0$};
    \node[green!70!black] at (5.5,-1.5) {$f'(x)>0$};
    \node[red!70!black] at (2.2,-1.5) {$f'(x)<0$};
\end{tikzpicture}
\end{center}

\textit{Stationary points} on \(f(x)\), where \(f'(x)=0\), give corresponding \textit{roots} on the graph of \(f'(x)\).

\begin{center}
\begin{tikzpicture}
    \draw[white] (-8,1.5) -- (9,1.5);
    \node[anchor=north east] at (1,0) {$0$};
    \draw[-stealth] (-4,0) -- (6.3,0) node[right] {$x$};
    \draw[-stealth] (1,-1) -- (1,1) node[above] {$f'(x)$};
    \draw[fill] (-2.5,0) circle (0.06cm);
    \draw[fill] (0,0) circle (0.06cm);
    \draw[fill] (4,0) circle (0.06cm);
    \draw[dotted] (-2.5,0) -- (-2.5,0) node[below] {$-7$};
    \draw[dotted] (0,0) -- (0,0) node[below] {$-2$};
    \draw[dotted] (4,0) -- (4,0) node[below] {$6$};
\end{tikzpicture}
\end{center}

\textit{Above} the \(x\)-axis when \(f(x)\) is increasing, and below when \(f(x)\) is decreasing, \(f'(x)\) can be sketched.

\begin{center}
\begin{tikzpicture}
    \draw[white] (-8,2.5) -- (9,2.5);
    \node[anchor=north east] at (1,0) {$0$};
    \draw[red,thick,double,smooth,domain=-4:-2.6] plot (\x,0.04*\x*\x*\x-0.06*\x*\x-0.4*\x);
    \draw[green,thick,double,smooth,domain=-2.4:-0.1] plot (\x,0.04*\x*\x*\x-0.06*\x*\x-0.4*\x);
    \draw[red,thick,double,smooth,domain=0.1:3.9] plot (\x,0.04*\x*\x*\x-0.06*\x*\x-0.4*\x);
    \draw[green,thick,double,smooth,domain=4.1:5.2] plot (\x,0.04*\x*\x*\x-0.06*\x*\x-0.4*\x);
    \draw[smooth,domain=-4:5.2] plot (\x,0.04*\x*\x*\x-0.06*\x*\x-0.4*\x) node[above] {$f'(x)$};
    \draw[-stealth] (-4,0) -- (6.3,0) node[right] {$x$};
    \draw[-stealth] (1,-2) -- (1,2) node[above] {$f'(x)$};
    \draw[fill] (-2.5,0) circle (0.06cm);
    \draw[fill] (0,0) circle (0.06cm);
    \draw[fill] (4,0) circle (0.06cm);
    \draw[dotted] (-2.5,0) -- (-2.5,0) node[below] {$-7$};
    \draw[dotted] (0,0) -- (0,0) node[below] {$-2$};
    \draw[dotted] (4,0) -- (4,0) node[below] {$6$};
\end{tikzpicture}
\end{center}

\pagebreak

\begin{tcolorbox}[title=Example 9.6.1,colback=blue!1, colframe=blue!70!black!70!]
Part of the graph of $y=f(x)$ is shown below. Sketch $y=f'(x)$.

\begin{center}
\begin{tikzpicture}
    \draw[blue!1] (-7,2.5) -- (7,2.5);
    \node[anchor=north east] at (0,0) {$0$};
    \draw[-stealth] (-5,0) -- (5,0) node[right] {$x$};
    \draw[-stealth] (0,-2) -- (0,3) node[above] {$y$};
    \draw[smooth,domain=-3:4] plot (\x,-0.1*\x*\x*\x*\x+0.2*\x*\x*\x+1.1*\x*\x-1.2*\x-1) node[below] {$y=f(x)$};
    \draw[fill] (-2,2.6) circle (0.06) node[above] {$(-4,5)$};
    \draw[fill] (0.5,-1.3) circle (0.06) node[below,xshift=0.2cm] {$(1,-3)$};
    \draw[fill] (3,2.6) circle (0.06) node[above] {$(6,5)$};
    \draw[fill] (-2.87,0) circle (0.06) node[below left] {$-6$};
    \draw[fill] (-0.573,0) circle (0.06) node[below left] {$-1$};
    \draw[fill] (1.573,0) circle (0.06) node[below right] {$3$};
    \draw[fill] (3.87,0) circle (0.06) node[below left] {$8$};
    \draw[fill] (0,-1) circle (0.06) node[below left] {$-2$};
\end{tikzpicture}
\end{center}

\tcblower

\textcolor{blue!70!black!70!}{\textit{Stationary points on }$y=f(x)$\textit{ are roots of }$y=f'(x)$:}

\begin{center}
\begin{tikzpicture}
    \draw[blue!1] (-7,2.5) -- (7,2.5);
    \node[anchor=north east] at (0,0) {$0$};
    \draw[-stealth] (-5,0) -- (5,0) node[right] {$x$};
    \draw[-stealth] (0,-2) -- (0,2) node[above] {$y$};
    \draw[smooth,domain=-2.9:3.8] plot (\x,-0.1*\x*\x*\x+0.15*\x*\x+0.55*\x-0.3) node[below] {$y=f'(x)$};
    \draw[fill] (-2,0) circle (0.06) node[below,xshift=-0.3cm] {$-4$};
    \draw[fill] (0.5,0) circle (0.06) node[below] {$1$};
    \draw[fill] (3,0) circle (0.06) node[below] {$6$};
\end{tikzpicture}
\end{center}

\end{tcolorbox}

Details such as that the derivative of a \textit{cubic} function is a \textit{quadratic} function should be observed.

\begin{tcolorbox}[title=Example 9.6.2,colback=blue!1, colframe=blue!70!black!70!]
For cubic function $g(x)$, part of the graph of $y=g(x)$ is shown below. Sketch $y=g'(x)$.

\begin{center}
\begin{tikzpicture}
    \draw[blue!1] (-7,2.5) -- (7,2.5);
    \node[anchor=north east] at (0,0) {$0$};
    \draw[-stealth] (-5,0) -- (5,0) node[right] {$x$};
    \draw[-stealth] (0,-1) -- (0,1.5) node[above] {$y$};
    \draw[smooth,domain=-4.5:4.5] plot (\x,0.03333*\x*\x*\x+0.05*\x*\x-0.6*\x) node[above] {$y=g(x)$};
    \draw[fill] (-3,1.35) circle (0.06) node[above] {$(-9,4)$};
    \draw[fill] (2,-0.73) circle (0.06) node[below] {$(6,-2)$};
    \draw[fill] (3.56,0) circle (0.06) node[below right] {$8$};
    \draw[fill] (0,0) circle (0.06);
\end{tikzpicture}
\end{center}

\tcblower

\textcolor{blue!70!black!70!}{\textit{Since }$g(x)$\textit{ is cubic, }$g'(x)$\textit{ must be quadratic, and so the graph is parabolic:}}

\begin{center}
\begin{tikzpicture}
    \draw[blue!1] (-7,2.5) -- (7,2.5);
    \node[anchor=north east] at (0,0) {$0$};
    \draw[-stealth] (-5,0) -- (5,0) node[right] {$x$};
    \draw[-stealth] (0,-2) -- (0,2) node[above] {$y$};
    \draw[smooth,domain=-4.5:4] plot (\x,0.1*\x*\x+0.1*\x-0.6) node[above] {$y=g'(x)$};
    \draw[fill] (-3,0) circle (0.06) node[below] {$-9$};
    \draw[fill] (2,0) circle (0.06) node[below] {$6$};
    \node[blue!70!black!70!] [anchor=east] at (7,-2) {\textit{(Rounding is to the nearest integer)}};
\end{tikzpicture}
\end{center}

\end{tcolorbox}

\pagebreak

\section*{Review Exercise}\label{review-exercise-8}
\addcontentsline{toc}{section}{Review Exercise}

\chapterfont{\color{white}}

\chapter{Polynomials}\label{polynomials}

\vspace{-12cm}
\begin{center}
\begin{tikzpicture}
\draw[white] (10,5) circle (0.01);
\draw[PineGreen,rounded corners,very thick] (1,0.2) rectangle (5,1.8);
\draw[PineGreen,very thick] (5,1) -- (6,1);
\draw[PineGreen,fill=PineGreen,rounded corners] (6,1.8) rectangle (18.2,0.2);
\node[white] at (12.1,1) {\Huge{\textsc{Polynomials}}};
\node at (3,1) {\Large{\textsc{Chapter 10}}};
\end{tikzpicture}
\end{center}

\section{Polynomials and Synthetic Division}\label{polynomials-and-synthetic-division}

\vspace{-0.5cm}

Expressions comprised of sums of \textit{positive, integer} powers of \(x\) are referred to as \textit{polynomials in }\(x\).

\begin{multicols}{5}
    \begin{center}
        \begin{tikzpicture}
            \draw[white] (-1.5,-1.7) rectangle (1.5,1.7);
            \node at (0,2.5) {\textbf{Constant}};
            \node at (0,2) {\textcolor{PineGreen}{\textit{Degree 0}}};
            \node at (0,1.5) {e.g. $7$};
            \draw[domain=-1.3:1.3,smooth] plot (\x,0);
        \end{tikzpicture}
    \end{center}
    \begin{center}
        \begin{tikzpicture}
            \draw[white] (-1.5,-1.7) rectangle (1.5,1.7);
            \node at (0,2.5) {\textbf{Linear}};
            \node at (0,2) {\textcolor{PineGreen}{\textit{Degree 1}}};
            \node at (0,1.5) {e.g. $2x+4$};
            \draw[domain=-1.3:1.3,smooth] plot (\x,0.5*\x);
        \end{tikzpicture}
    \end{center}
    \begin{center}
        \begin{tikzpicture}
            \draw[white] (-1.5,-1.7) rectangle (1.5,1.7);
            \node at (0,2.5) {\textbf{Quadratic}};
            \node at (0,2) {\textcolor{PineGreen}{\textit{Degree 2}}};
            \node at (0,1.5) {e.g. $x^2-3x+1$};
            \draw[domain=-1.4:1.4,smooth] plot (\x,0.5*\x*\x-0.5);
        \end{tikzpicture}
    \end{center}
    \begin{center}
        \begin{tikzpicture}
            \draw[white] (-1.5,-1.7) rectangle (1.5,1.7);
            \node at (0,2.5) {\textbf{Cubic}};
            \node at (0,2) {\textcolor{PineGreen}{\textit{Degree 3}}};
            \node at (0,1.5) {e.g. $4x^3-2x$};
            \draw[domain=-1.1:1.6,smooth] plot (\x,\x*\x*\x-\x*\x-\x+0.5);
        \end{tikzpicture}
    \end{center}
    \begin{center}
        \begin{tikzpicture}
            \draw[white] (-1.5,-1.7) rectangle (1.5,1.7);
            \node at (0,2.5) {\textbf{Quartic}};
            \node at (0,2) {\textcolor{PineGreen}{\textit{Degree 4}}};
            \node at (0,1.5) {e.g. $x^4-3x^3+1$};
            \draw[domain=-1.1:1.4,smooth] plot (\x,\x*\x*\x*\x-\x*\x*\x-\x*\x+\x);
        \end{tikzpicture}
    \end{center}
\end{multicols}

\vspace{-1.3cm}

A polynomial function such as \(f(x)=2x^3-7x^2+4x-1\) can be evaluated for some value of \(x\):

\vspace{-0.6cm}

\begin{align*}
    f(3)&=2(3)^3-7(3)^2+4(3)+1\\
    &=54-63+12+1\\
    &=4
\end{align*}

\vspace{-0.4cm}

Another way to evaluate \(f(3)\) is to use synthetic division, as follows:

\begin{enumerate}
\def\labelenumi{\arabic{enumi})}
\tightlist
\item
  Draw a table as below and fill in the first row, then bring the first coefficient down:
\end{enumerate}

\vspace{-0.2cm}

\begin{center}
    \begin{tikzpicture}
        \draw[white] (-2,1.6) -- (14,1.6);
        \draw (0,1.6) -- (0,0) -- (6,0);
        \draw (4.5,0) -- (4.5,-0.8) -- (6,-0.8);
        \node at (0.75,1.2) {$2$};
        \node at (2.25,1.2) {$-7$};
        \node at (3.75,1.2) {$4$};
        \node at (5.25,1.2) {$1$};
        \node at (-0.75,1.2) {$3$};
        \node at (0.75,-0.4) {$2$};
        \draw[stealth-,PineGreen] (6.5,1.2) -- (7.5,1.2) node[right] {\text{Coefficients of }$2x^3-7x^2+4x+1$};
        \draw[stealth-,PineGreen] (-0.9,0.9) -- (-1.5,0) node[below] {$x=3$};
        \draw[-stealth,PineGreen] (0.75,0.8) -- (0.75,0.2);
    \end{tikzpicture}
\end{center}

\vspace{-0.2cm}

\begin{enumerate}
\def\labelenumi{\arabic{enumi})}
\setcounter{enumi}{1}
\tightlist
\item
  The 6 is obtained by \textit{multiplying} 2 by 3, and the -1 by \textit{adding} the two values above:
\end{enumerate}

\vspace{-0.2cm}

\begin{center}
    \begin{tikzpicture}
        \draw[white] (-2,1.6) -- (14,1.6);
        \draw[PineGreen!10,fill=PineGreen!10] (1.9,-0.7) rectangle (2.6,1.5);
        \draw (0,1.6) -- (0,0) -- (6,0);
        \draw (4.5,0) -- (4.5,-0.8) -- (6,-0.8);
        \node at (0.75,1.2) {$2$};
        \node at (2.25,1.2) {$-7$};
        \node at (3.75,1.2) {$4$};
        \node at (5.25,1.2) {$1$};
        \node at (-0.75,1.2) {$3$};
        \node at (0.75,-0.4) {$2$};
        \node at (2.25,0.4) {$\phantom{-}6$};
        \node at (2.25,-0.4) {$-1$};
        \node[right,PineGreen] at (7.5,1.2) {\textit{Multiply diagonally by 3}};
        \node[right,PineGreen] at (7.5,0.4) {\textit{Add columns}};
        \draw[PineGreen] (-0.75,1.2) circle (0.3cm);
        \draw[-stealth,PineGreen] (1.1,-0.2) -- (1.7,0.2);
    \end{tikzpicture}
\end{center}

\vspace{-0.2cm}

\begin{enumerate}
\def\labelenumi{\arabic{enumi})}
\setcounter{enumi}{2}
\tightlist
\item
  By continuing this \textit{multiplying then adding} pattern, the value of \(f(3)=4\) can be obtained:
\end{enumerate}

\vspace{-0.2cm}

\begin{center}
    \begin{tikzpicture}
        \draw[white] (-2,1.6) -- (14,1.6);
        \draw[PineGreen!10,fill=PineGreen!10] (4.5,-0.8) rectangle (6,0);
        \draw (0,1.6) -- (0,0) -- (6,0);
        \draw (4.5,0) -- (4.5,-0.8) -- (6,-0.8);
        \node at (0.75,1.2) {$2$};
        \node at (2.25,1.2) {$-7$};
        \node at (3.75,1.2) {$\phantom{-}4$};
        \node at (5.25,1.2) {$1$};
        \node at (-0.75,1.2) {$3$};
        \node at (0.75,-0.4) {$2$};
        \node at (2.25,0.4) {$\phantom{-}6$};
        \node at (2.25,-0.4) {$-1$};
        \node at (3.75,0.4) {$-3$};
        \node at (3.75,-0.4) {$\phantom{-}1$};
        \node at (5.25,0.4) {$3$};
        \node at (5.25,-0.4) {$4$};
        \draw[stealth-,PineGreen] (6.5,-0.4) -- (7.5,-0.4) node[right] {$\therefore f(3)=4$};
    \end{tikzpicture}
\end{center}

\begin{tcolorbox}[title=Example 10.1.1,colback=PineGreen!2, colframe=PineGreen]

Given $f(x)=x^4-4x^3+3x^2+9$, use synthetic division to evaluate $f(2)$.

\tcblower

\begin{center}
    \begin{tikzpicture}
        \draw[white] (-5,1.6) -- (11,1.6);
        \draw (-1.5,1.6) -- (-1.5,0) -- (6,0);
        \draw (4.5,0) -- (4.5,-0.8) -- (6,-0.8);
        \node at (-2.25,1.2) {$2$};
        \node at (-0.75,1.2) {$1$};
        \node at (0.75,1.2) {$-4$};
        \node at (2.25,1.2) {$\phantom{-}3$};
        \node at (3.75,1.2) {$\phantom{-}0$};
        \node at (5.25,1.2) {$\phantom{-}9$};
        \node at (-0.75,-0.4) {$1$};
        \node at (0.75,0.4) {$\phantom{-}2$};
        \node at (0.75,-0.4) {$-2$};
        \node at (2.25,0.4) {$-4$};
        \node at (2.25,-0.4) {$-1$};
        \node at (3.75,0.4) {$-2$};
        \node at (3.75,-0.4) {$-2$};
        \node at (5.25,0.4) {$-4$};
        \node at (5.25,-0.4) {$\phantom{-}5$};
        \node[right] at (6.5,-0.4) {$\therefore f(2)=5$};
    \end{tikzpicture}
\end{center}

\end{tcolorbox}

The top line within a synthetic division box should be \textit{ordered in decreasing powers of }\(x\).\textbackslash Any missing powers are included with a coefficient of \textit{zero}.

\pagebreak

\section{Remainder Theorem}\label{remainder-theorem}

\vspace{-0.5cm}

Addition, subtraction and multiplication of polynomials were all required for the National 5 course.

Since \(3\times 4 + 1=13\), this implies that:

\[\dfrac{13}{3}=4\textbf{ Remainder }1\]

Since \((x-2)(x^2-3x+4)+7=x^3-5x^2+x-1\), this implies that:

\[\dfrac{x^3-5x^2+x-1}{x-2}=x^2-3x+4\textbf{ Remainder }7\]

Note that for \(f(x)=x^3-5x^2+x-1\), \(f(2)=(2)^3-5(2)^2+(2)-1=7\).

\begin{center}
\begin{tcolorbox}[colback=red!5,width=18cm,colframe=red!70!black]
\textbf{Remainder Theorem:}\\
    \begin{center}
        If polynomial $f(x)$ is divided by a linear divisor $(x-a)$ then the remainder is $f(a)$, and\\[0.4em] $f(x)=(x-a)Q(x)+f(a)$ where quotient $Q(x)$ is a polynomial of degree one less than $f(x)$.
    \end{center}
\end{tcolorbox}
\end{center}

\begin{tcolorbox}[title=Example 10.2.1,colback=PineGreen!2, colframe=PineGreen]

Find the remainder when $3x^3-x^2+5$ is divided by $(x-1)$.

\tcblower

\begin{center}
    \begin{tikzpicture}
        \draw[white] (-5,1.6) -- (11,1.6);
        \draw (0,1.6) -- (0,0) -- (6,0);
        \draw (4.5,0) -- (4.5,-0.8) -- (6,-0.8);
        \node at (-0.75,1.2) {$1$};
        \node at (0.75,1.2) {$3$};
        \node at (2.25,1.2) {$-1$};
        \node at (3.75,1.2) {$0$};
        \node at (5.25,1.2) {$5$};
        \node at (0.75,-0.4) {$3$};
        \node at (2.25,0.4) {$\phantom{-}3$};
        \node at (2.25,-0.4) {$\phantom{-}2$};
        \node at (3.75,0.4) {$2$};
        \node at (3.75,-0.4) {$2$};
        \node at (5.25,0.4) {$2$};
        \node at (5.25,-0.4) {$7$};
        \node[right,PineGreen] at (6.5,-0.4) {$\longleftarrow$ \textit{Remainder}};
        \node[left,PineGreen] at (-2.5,1.2) {$(x-1)$};
        \node[left,PineGreen] at (-2.5,0.4) {$\rightarrow f(1)$};
    \end{tikzpicture}
\end{center}

So the remainder is $7$.

\end{tcolorbox}

The \textit{quotient}, \(Q(x)\), can be determined from the values to the left of the remainder.\textbackslash{} Since \(3x^3-x^2+5\) is a \textit{cubic}, the quotient \(Q(x)\) is \textit{quadratic}:\quad \(3x^2+2x+2\).

\[3x^3-x^2+5=(x-1)(3x^2+2x+2)+7\]

\begin{tcolorbox}[title=Example 10.2.2,colback=PineGreen!2, colframe=PineGreen]

Function $f$ is defined by $f(x)=x^3+4x^2-3x-7$. Find the remainder $r$ when $f(x)$ is divided by $(x+1)$, and hence express $f(x)$ in the form $(x+1)Q(x)+r$ where $Q(x)$ is a quadratic.

\tcblower

\begin{center}
    \begin{tikzpicture}
        \draw[white] (-5,1.6) -- (11,1.6);
        \draw (0,1.6) -- (0,0) -- (6,0);
        \draw (4.5,0) -- (4.5,-0.8) -- (6,-0.8);
        \node at (-0.75,1.2) {$-1$};
        \node at (0.75,1.2) {$1$};
        \node at (2.25,1.2) {$\phantom{-}4$};
        \node at (3.75,1.2) {$-3$};
        \node at (5.25,1.2) {$-7$};
        \node at (0.75,-0.4) {$1$};
        \node at (2.25,0.4) {$-1$};
        \node at (2.25,-0.4) {$\phantom{-}3$};
        \node at (3.75,0.4) {$-3$};
        \node at (3.75,-0.4) {$-6$};
        \node at (5.25,0.4) {$\phantom{-}6$};
        \node at (5.25,-0.4) {$-1$};
        \node[right,PineGreen] at (6.5,-0.4) {$\longleftarrow$ \textit{Remainder}};
        \node[left,PineGreen] at (-0.5,-0.4) {\textit{Quotient}$\longrightarrow$};
    \end{tikzpicture}
\end{center}

So the remainder is $-1$ and $x^3+4x^2-3x-7=(x+1)(x^2+3x-6)-1$.

\end{tcolorbox}

\pagebreak

\section*{10.3 Factor Theorem}

\vspace{-0.5cm}

Since 20 can be divided by 5 to give 4 \textbf{with no remainder}, this means 5 is a \textbf{factor} of 20.

\vspace{-0.2cm}

\begin{center}
\begin{tcolorbox}[colback=red!5,width=18cm,colframe=red!70!black]
\textbf{Factor Theorem:}\\[-0.5em]
    \begin{center}
        If polynomial $f(x)$ is divided by a linear \textbf{factor} $(x-a)$ then the \textbf{remainder }${f(a)=0}$, and\\[0.4em] $f(x)=(x-a)Q(x)$ where quotient $Q(x)$ is a polynomial of degree one less than $f(x)$. Or:
    $$f(a)=0\iff (x-a)\text{ is a factor of }f(x)$$
    \end{center}
\end{tcolorbox}
\end{center}

\vspace{-0.3cm}

Given a linear factor, synthetic division allows a cubic expression \(f(x)\) to be expressed as a the product of the \textit{linear} factor and a \textit{quadratic} quotient, which may potentially by further factorised:
\vspace{-0.1cm}
\begin{align*}
    f(x)=&\text{ }(\textit{cubic})\\
    \qquad \qquad \qquad \qquad =&\text{ }(\textit{linear})(\textit{quadratic})&&\textcolor{PineGreen}{\longleftarrow\textit{ Using synthetic division}}\\
    =&\text{ }(\textit{linear})(\textit{linear})(\textit{linear})
\end{align*}

\vspace{-0.3cm}

\begin{tcolorbox}[title=Example 10.3.1,colback=PineGreen!2, colframe=PineGreen]

Show that $(x+4)$ is a factor of $x^3-x^2-14x+24$, and hence factorise it fully.

\tcblower

\begin{center}
    \begin{tikzpicture}
        \draw[white] (-5,1.6) -- (11,1.6);
        \draw (0,1.6) -- (0,0) -- (6,0);
        \draw (4.5,0) -- (4.5,-0.8) -- (6,-0.8);
        \node at (-0.75,1.2) {$-4$};
        \node at (0.75,1.2) {$1$};
        \node at (2.25,1.2) {$-1$};
        \node at (3.75,1.2) {$-14$};
        \node at (5.25,1.2) {$\phantom{-}24$};
        \node at (0.75,-0.4) {$1$};
        \node at (2.25,0.4) {$-4$};
        \node at (2.25,-0.4) {$-5$};
        \node at (3.75,0.4) {$\phantom{-}20$};
        \node at (3.75,-0.4) {$\phantom{-}6$};
        \node at (5.25,0.4) {$-24$};
        \node at (5.25,-0.4) {$\phantom{-}0$};
        \node[right,PineGreen] at (6.5,-0.4) {$\longleftarrow$ \textit{Remainder}};
        \node[left,PineGreen] at (-0.5,-0.4) {\textit{Quotient}$\longrightarrow$};
    \end{tikzpicture}
\end{center}
\centering
Remainder$=0\therefore (x+4)$ is a factor. \textcolor{PineGreen}{$\longleftarrow$ \textit{``Show that..''}}

\vspace{-0.3cm}
\begin{align*}
    &\text{ }x^3-x^2-14x+24\\
    =&\text{ }(x+4)(x^2-5x+6)\\
    =&\text{ }(x+4)(x-2)(x-3)
\end{align*}

\end{tcolorbox}

For some cubic expressions, a resulting quadratic quotient may not be able to be factorised.

\begin{tcolorbox}[title=Example 10.3.2,colback=PineGreen!2, colframe=PineGreen]

Show that $(x-2)$ is a factor of $2x^3-x^2+x-14$, and hence factorise it fully.

\tcblower

\begin{center}
    \begin{tikzpicture}
        \draw[white] (-5,1.6) -- (11,1.6);
        \draw (0,1.6) -- (0,0) -- (6,0);
        \draw (4.5,0) -- (4.5,-0.8) -- (6,-0.8);
        \node at (-0.75,1.2) {$2$};
        \node at (0.75,1.2) {$2$};
        \node at (2.25,1.2) {$-1$};
        \node at (3.75,1.2) {$1$};
        \node at (5.25,1.2) {$-14$};
        \node at (0.75,-0.4) {$2$};
        \node at (2.25,0.4) {$\phantom{-}4$};
        \node at (2.25,-0.4) {$\phantom{-}3$};
        \node at (3.75,0.4) {$6$};
        \node at (3.75,-0.4) {$7$};
        \node at (5.25,0.4) {$\phantom{-}14$};
        \node at (5.25,-0.4) {$\phantom{-}0$};
    \end{tikzpicture}
\end{center}
\centering
\textcolor{PineGreen}{\textit{``Show that }$(x-2)$\textit{ is a factor...'' }$\longrightarrow$} Remainder$=0\therefore (x-2)$ is a factor.

\vspace{-0.3cm}
\begin{align*}
    &\text{ }2x^3-x^2+x-14\\
    =&\text{ }(x-2)(2x^2+3x+7)
\end{align*}

Since $3^2-4\times2\times7=-65$, and $-65<0$, the quadratic quotient cannot be factorised\\ Therefore the cubic cannot be further factorised. 

\end{tcolorbox}

\pagebreak

Where a linear factor of a polynomial function is unknown, a suggested approach is to consider the factors of the constant term with the aim of obtaining a value such that the remainder is zero.

\begin{tcolorbox}[title=Example 10.3.3,colback=PineGreen!2, colframe=PineGreen]

Factorise $x^3-x^2-17x=15$.

\tcblower

\textcolor{PineGreen}{\textit{Consider }$\pm1,\pm3\pm5\pm15$}\\[-0.5em]

$f(1)=(1)^3-(1)^2-17(1)-15=-32\therefore$ $(x-1)$ is \textbf{not} a factor.\\[-0.5em]

$f(-1)=(-1)^3-(-1)^2-17(-1)-15=0\therefore$ $(x+1)$ \textbf{is} a factor.


\begin{center}
    \begin{tikzpicture}
        \draw[white] (-5,1.6) -- (11,1.6);
        \draw (0,1.6) -- (0,0) -- (6,0);
        \draw (4.5,0) -- (4.5,-0.8) -- (6,-0.8);
        \node at (-0.75,1.2) {$-1$};
        \node at (0.75,1.2) {$1$};
        \node at (2.25,1.2) {$-1$};
        \node at (3.75,1.2) {$-17$};
        \node at (5.25,1.2) {$-15$};
        \node at (0.75,-0.4) {$1$};
        \node at (2.25,0.4) {$-1$};
        \node at (2.25,-0.4) {$-2$};
        \node at (3.75,0.4) {$\phantom{-}2$};
        \node at (3.75,-0.4) {$-15$};
        \node at (5.25,0.4) {$\phantom{-}15$};
        \node at (5.25,-0.4) {$\phantom{-}0$};
    \end{tikzpicture}
\end{center}
\centering

\vspace{-0.7cm}
\begin{align*}
    &\text{ }x^3-x^2-17x-15\\
    =&\text{ }(x+1)(x^2-2x-15)\\
    =&\text{ }(x+1)(x+3)(x-5)
\end{align*}

\end{tcolorbox}

Repeated use of synthetic dvision may allow a \textit{quartic} (\(x^4\)) to be expressed as a product of up to four linear factors.

\vspace{-1cm}

\begin{align*}
    f(x)=&\text{ }(\textit{quartic})\\
    =&\text{ }(\textit{linear})(\textit{cubic})&&\textcolor{PineGreen}{\longleftarrow\textit{ Using synthetic division}}\\
    \qquad \qquad \qquad =&\text{ }(\textit{linear})(\textit{linear})(\textit{quadratic})&&\textcolor{PineGreen}{\longleftarrow\textit{ Using synthetic division}}\\
    =&\text{ }(\textit{linear})(\textit{linear})(\textit{linear})(\textit{linear})
\end{align*}

\vspace{-0.1cm}

\begin{tcolorbox}[title=Example 10.3.4,colback=PineGreen!2, colframe=PineGreen]

Show that $(x-2)$ is a factor of $x^4-2x^3-7x^2+8x+12$ and hence factorise it fully.

\tcblower

\begin{center}
    \begin{tikzpicture}
        \draw[white] (-5,1.6) -- (11,1.6);
        \draw (-1.5,1.6) -- (-1.5,0) -- (6,0);
        \draw (4.5,0) -- (4.5,-0.8) -- (6,-0.8);
        \node at (-2.25,1.2) {$2$};
        \node at (-0.75,1.2) {$1$};
        \node at (0.75,1.2) {$-2$};
        \node at (2.25,1.2) {$-7$};
        \node at (3.75,1.2) {$\phantom{-}8$};
        \node at (5.25,1.2) {$\phantom{-}12$};
        \node at (-0.75,-0.4) {$1$};
        \node at (0.75,0.4) {$\phantom{-}2$};
        \node at (0.75,-0.4) {$\phantom{-}0$};
        \node at (2.25,0.4) {$\phantom{-}0$};
        \node at (2.25,-0.4) {$-7$};
        \node at (3.75,0.4) {$-14$};
        \node at (3.75,-0.4) {$-6$};
        \node at (5.25,0.4) {$-12$};
        \node at (5.25,-0.4) {$\phantom{-}0$};
    \end{tikzpicture}
\end{center}

\centering
Remainder$=0\therefore (x-2)$ is a factor.\\[0.5em]

$x^4-2x^3-7x^2+8x+12=(x-2)(x^3-7x-6)$

\begin{center}
    \begin{tikzpicture}
        \draw[white] (-5,1.6) -- (11,1.6);
        \draw (0,1.6) -- (0,0) -- (6,0);
        \draw (4.5,0) -- (4.5,-0.8) -- (6,-0.8);
        \node at (-0.75,1.2) {$3$};
        \node at (0.75,1.2) {$1$};
        \node at (2.25,1.2) {$0$};
        \node at (3.75,1.2) {$-7$};
        \node at (5.25,1.2) {$-6$};
        \node at (0.75,-0.4) {$1$};
        \node at (2.25,0.4) {$3$};
        \node at (2.25,-0.4) {$3$};
        \node at (3.75,0.4) {$\phantom{-}9$};
        \node at (3.75,-0.4) {$\phantom{-}2$};
        \node at (5.25,0.4) {$\phantom{-}6$};
        \node at (5.25,-0.4) {$\phantom{-}0$};
        \node[left] at (-1.5,0) {\textcolor{PineGreen}{\textit{Try factors of 6}}};
        \draw[PineGreen,-stealth] (-1.5,0.3) -- (-0.9,0.9);
    \end{tikzpicture}
\end{center}

\vspace{-0.8cm}

\begin{align*}
    x^4-2x^3-7x^2+8x+12&=(x-2)(x^3-7x-6)\\
    &=(x-2)(x-3)(x^2+3x+2)\\
    &=(x-2)(x-3)(x+1)(x+2)
\end{align*}

\end{tcolorbox}

\pagebreak

\section{Finding Unknown Coefficients}\label{finding-unknown-coefficients}

\vspace{-0.5cm}

An unknown coefficient of a polynomial expression may be calculated if a factor of the expression is known, or the remainder given a linear divisor is known. It may be preferred to use synthetic division, or to evaluate the expression directly.

\begin{tcolorbox}[title=Example 10.4.1,colback=PineGreen!2, colframe=PineGreen]

Given that $(x+2)$ is a factor of $2x^3+kx^2-14x+8$, determine the value of $k$.

\tcblower


\begin{center}
    \begin{tikzpicture}
        \draw[white] (-4,1.6) -- (12,1.6);
        \fill[PineGreen!10] (5.5,-0.8) rectangle (7.5,0);
        \draw (0,1.6) -- (0,0) -- (7.5,0);
        \draw (5.5,0) -- (5.5,-0.8) -- (7.5,-0.8);
        \node at (-0.75,1.2) {$-2$};
        \node at (0.75,1.2) {$2$};
        \node at (2.25,1.2) {$\phantom{-}k$};
        \node at (4.25,1.2) {$-14$};
        \node at (6.55,1.2) {$8$};
        \node at (0.75,-0.4) {$2$};
        \node at (2.25,0.4) {$-4$};
        \node at (2.25,-0.4) {$k-4$};
        \node at (4.25,0.4) {$-2k+8$};
        \node at (4.25,-0.4) {$-2k-6$};
        \node at (6.55,0.4) {$4k+12$};
        \node at (6.55,-0.4) {$4k+20$};
        \node[right] at (7.7,-0.4) {\textcolor{PineGreen}{$\longleftarrow$\textit{ Remainder}}};
    \end{tikzpicture}
\end{center}
\centering
Since $(x+2)$ is a factor, remainder$=0$.
\begin{align*}
    4k+20&=0\\
    4k&=-20\\
    k&=-5
\end{align*}
\end{tcolorbox}

Where a polynomial has \textit{two} unknown coefficients, \textit{simultaneous equations} may be formed and solved.

\begin{tcolorbox}[title=Example 10.4.2,colback=PineGreen!2, colframe=PineGreen]

Find the values of $a$ and $b$ given that:\\
\begin{itemize}
    \item $(x-2)$ is a factor of $x^3+ax^2+bx+12$.
    \item $x^3+ax^2+bx+12$ divided by $(x-1)$ has a remainder of 10.
\end{itemize}

\tcblower


\begin{center}
    \begin{tikzpicture}
        \draw[white] (-3.5,1.6) -- (11.5,1.6);
        \draw (0,1.6) -- (0,0) -- (8.9,0);
        \draw (6.1,0) -- (6.1,-0.8) -- (8.9,-0.8);
        \node at (-0.75,1.2) {$2$};
        \node at (0.75,1.2) {$1$};
        \node at (2.25,1.2) {$a$};
        \node at (4.55,1.2) {$b$};
        \node at (7.55,1.2) {$12$};
        \node at (0.75,-0.4) {$1$};
        \node at (2.25,0.4) {$2$};
        \node at (2.25,-0.4) {$a+2$};
        \node at (4.55,0.4) {$2a+4$};
        \node at (4.55,-0.4) {$2a+b+4$};
        \node at (7.55,0.4) {$4a+2b+8$};
        \node at (7.55,-0.4) {$4a+2b+20$};
    \end{tikzpicture}
\end{center}
\centering
Since $(x-2)$ is a factor, remainder $=0$ $\implies$ $4a+2b+20=0$.

\begin{center}
    \begin{tikzpicture}
        \draw[white] (-3.5,1.6) -- (11.5,1.6);
        \draw (0,1.6) -- (0,0) -- (8.9,0);
        \draw (6.1,0) -- (6.1,-0.8) -- (8.9,-0.8);
        \node at (-0.75,1.2) {$1$};
        \node at (0.75,1.2) {$1$};
        \node at (2.25,1.2) {$a$};
        \node at (4.55,1.2) {$b$};
        \node at (7.55,1.2) {$12$};
        \node at (0.75,-0.4) {$1$};
        \node at (2.25,0.4) {$1$};
        \node at (2.25,-0.4) {$a+1$};
        \node at (4.55,0.4) {$a+1$};
        \node at (4.55,-0.4) {$a+b+1$};
        \node at (7.55,0.4) {$a+b+1$};
        \node at (7.55,-0.4) {$a+b+13$};
    \end{tikzpicture}
\end{center}

\centering
Remainder $=10$ $\implies$ $a+b+13=10$.
\vspace{-0.3cm}
\flushleft
\textcolor{PineGreen}{\textit{Solve simultaneously:}}
\vspace{-0.8cm}
\begin{center}
    \begin{tikzpicture}
        \node[below left] at (0,0) {
        \begin{minipage}{7cm}
            \begin{align*}
                4a+2b&=-20&\qquad \qquad 4a+2b&=-20\\[0.2em]
                a+b&=-3&\qquad \qquad 2a+2b&=-6\\[0.2em]
                &&2a&=-14\\
                &&a&=-7
            \end{align*}
        \end{minipage}
        };
        \node[below right] at (1.5,-0.57) {
        \begin{minipage}{4cm}
            \centering
            \textcolor{PineGreen}{\textit{Sub }$a=-7$:}
            \vspace{-0.3cm}
            \begin{align*}
                a+b&=-3\\[0.2em]
                (-7)+b&=-3\\[0.2em]
                b&=4
            \end{align*}
        \end{minipage}
        };
        \draw[PineGreen] (-3,-2) -- (0,-2) node[right] {\scriptsize{$(-)$}};
        \draw [decoration={brace,amplitude=0.5em},decorate,thick,PineGreen]
(-4.5,-0.8) -- (-4.5,-1.8);
        \node[PineGreen] at (-4.2,-1.8) {\scriptsize{$\times 2$}};
    \end{tikzpicture}
\end{center}
\end{tcolorbox}

\pagebreak

\section{Solving Polynomial Equations}\label{solving-polynomial-equations}

\vspace{-0.5cm}

\textit{Quadratic} equations such as \(x^2-4x-21=0\) are typically solved through factorising, or using the quadratic formula. There is no equivalent \textit{general formula} for solving polynomial equations of degree 3 (\textit{cubic}) or higher. Instead, such polynomial equations are solved through \textit{factorising}.

\newpage

\section{Identifying Polynomial Functions}\label{identifying-polynomial-functions}

Given the form a polynomial function takes and its graph, showing any \(x\)-intercepts and at least one other coordinate, the factorised form of the polynomial may be deduced.

\begin{itemize}
    \item An $x$-intercept (or \textit{root}) when $x=a$ implies that a polynomial has a \textit{factor} of $(x-a)$.
    \item A \textit{stationary point} at an $x$-intercept is a \textit{repeated root}, and its factor will be (at least) squared.
\end{itemize}

\begin{multicols}{2}

\begin{center}
    \begin{tikzpicture}
        \draw[white] (-4,-2.5) rectangle (4,1.5);
        \draw[-stealth] (-3,0) -- (3,0) node[below] {$x$};
        \draw[smooth,domain=-2:2,yscale=0.1] plot (\x,\x*\x+4*\x);
        \draw[fill] (0,0) circle (0.05) node[below] {$a$};
        \node[PineGreen] at (0,-1) {Root when $x=a$};
        \node[PineGreen] at (0,-1.8) {Factor of $(x-a)$};
    \end{tikzpicture}
\end{center}

\begin{center}
    \begin{tikzpicture}
        \draw[white] (-4,-2.5) rectangle (4,1.5);
        \draw[-stealth] (-3,0) -- (3,0) node[below] {$x$};
        \draw[smooth,domain=-2:2,yscale=0.2] plot (\x,\x*\x);
        \draw[fill] (0,0) circle (0.05) node[below] {$a$};
        \node[PineGreen] at (0,-1) {Repeated root when $x=a$};
        \node[PineGreen] at (0,-1.8) {Factor of $(x-a)^n$};
    \end{tikzpicture}
\end{center}
    
\end{multicols}

The diagram below shows the graph of a quartic function with equation \(y=k(x-a)(x-b)(x-c)^2\).

\begin{center}
    \begin{tikzpicture}
        \draw[-stealth] (-4,0) -- (4,0) node[below] {$x$};
        \draw[-stealth] (0,-2) -- (0,3) node[left] {$y$};
        \draw[smooth,domain=-3:3.3,yscale=0.1] plot (\x,\x*\x*\x*\x-9*\x*\x-4*\x+12) node[above] {$y=k(x-a)(x-b)(x-c)^2$};
        \node[below left] {$0$};
        \draw[fill] (-2,0) circle (0.05) node[below] {$-2\phantom{-}$};
        \draw[fill] (1,0) circle (0.05) node[below] {$1\phantom{.}$};
        \draw[fill] (3,0) circle (0.05) node[below] {$\phantom{..}3$};
        \draw[fill] (0,1.2) circle (0.05) node[above left] {$6$};
        \draw[PineGreen,stealth-] (-2.5,-0.5) -- (-3.5,-1.5) node[below left] {\textcolor{PineGreen}{\textit{Repeated root}}};
    \end{tikzpicture}
\end{center}

With roots at \(x=1\) and \(x=3\), and a \textit{repeated root} at \(x=-2\), the quartic can be expressed as:

\[y=k(x-1)(x-3)(x+2)^2\]

The value of \(k\) can only be found if a \textit{non-root} coordinate is also known. By substitution:

\begin{align*}
    y&=k(x-1)(x-3)(x+2)^2&&\\
    6&=k(0-1)(0-3)(0+2)^2&&\textcolor{PineGreen}{\textit{Substitute in }(0,6)}\\
    6&=k(-1)(-3)(2)^2&&\\
    6&=12k&&\\
    \frac{6}{12}&=k\\
    \frac{1}{2}&=k\\
    y&=\frac{1}{2}(x-1)(x-3)(x+2)^2&&\textcolor{PineGreen}{\textit{State equation}}
\end{align*}

By comparing the form given and the final equation, it can be seen that: \(k=\frac{1}{2},a=1,b=3,c=-2\)

\pagebreak

\section*{Review Exercise}\label{review-exercise-9}
\addcontentsline{toc}{section}{Review Exercise}

\chapterfont{\color{white}}

\chapter{Integration}\label{integration}

\vspace{-12cm}
\begin{center}
\begin{tikzpicture}
\draw[white] (10,5) circle (0.01);
\draw[NavyBlue,rounded corners,very thick] (1,0.2) rectangle (5,1.8);
\draw[NavyBlue,very thick] (5,1) -- (6,1);
\draw[NavyBlue,fill=NavyBlue,rounded corners] (6,1.8) rectangle (18.2,0.2);
\node[white] at (12.1,1) {\Huge{\textsc{Integration}}};
\node at (3,1) {\Large{\textsc{Chapter 11}}};
\end{tikzpicture}
\end{center}

\subsection*{Introduction and Overview}\label{introduction-and-overview}
\addcontentsline{toc}{subsection}{Introduction and Overview}

In mathematics, \textit{integration} can be thought of as the process of calculating some kind of \emph{``sum''}, such as the total area under a curve, by breaking it down into sums of increasingly small parts. This technique was known about and used thousands of years ago in both Ancient Greece and other parts of the world.

\begin{center}
    \begin{multicols}{3}

    \begin{tikzpicture}[scale=0.9]
        \draw[white] (-3,-1) rectangle (3.2,2);
        \draw[fill=blue!20] (-1.6,0) rectangle (-1.2,0.72);
        \draw[fill=blue!20] (-1.2,0) rectangle (-0.8,1.28);
        \draw[fill=blue!20] (-0.8,0) rectangle (-0.4,1.68);
        \draw[fill=blue!20] (-0.4,0) rectangle (0,1.92);
        \draw[fill=blue!20] (0,0) rectangle (0.4,2);
        \draw[fill=blue!20] (0.4,0) rectangle (0.8,1.92);
        \draw[fill=blue!20] (0.8,0) rectangle (1.2,1.68);
        \draw[fill=blue!20] (1.2,0) rectangle (1.6,1.28);
        \draw[fill=blue!20] (1.6,0) rectangle (2,0.72);
        \draw[-stealth,thick] (-3,0) -- (3,0) node[below] {$x$};
        \draw[smooth,domain=-2:2,thick] plot (\x,-0.5*\x*\x+2);
        \node[below] at (0,0) {$\Delta x=0.4$};
    \end{tikzpicture}
    
    \begin{tikzpicture}[scale=0.9]
        \draw[white] (-3,-1) rectangle (3.2,2);
        \foreach \x in {-2,-1.8,...,1.8}{
        \draw[fill=blue!20] (\x,0) rectangle (\x+0.2,-0.5*\x*\x+2);
        }
        \draw[-stealth,thick] (-3,0) -- (3,0) node[below] {$x$};
        \draw[smooth,domain=-2:2,thick] plot (\x,-0.5*\x*\x+2);
        \node[below] at (0,0) {$\Delta x=0.2$};
    \end{tikzpicture}
    
    \begin{tikzpicture}[scale=0.9]
        \draw[white] (-3,-1) rectangle (3.2,2);
        \draw[smooth,domain=-2:2,,thick,fill=blue!20] plot (\x,-0.5*\x*\x+2);
        \draw[-stealth,thick] (-3,0) -- (3,0) node[below] {$x$};
        \node[below] at (0,0) {As $\Delta x\to 0$};
    \end{tikzpicture}
    
    \end{multicols}
\end{center}

In the 17th century both Gottfried Wilhelm Leibniz and Isaac Newton independently discovered the \emph{fundamental theorem of calculus}, showing that \emph{integration} can be performed using \emph{antidifferentiation}, which is the `reverse' of differentiation. Now, \emph{integration} is used to refer to both the idea of calculating such a `sum', and to the process of antidifferentiation.

\begin{center}
    \begin{tikzpicture}
        \draw[very thick,NavyBlue,-stealth] (120:6) arc (120:60:6);
        \draw[very thick,NavyBlue,-stealth] (120:6) arc (120:60:6);
    \end{tikzpicture}
\end{center}

This chapter will first introduce the use of integration to calculate an \emph{antiderivative}, before moving on to using this concept to calculating area enclosed using curves.

\subsection*{Chapter Contents}\label{chapter-contents}
\addcontentsline{toc}{subsection}{Chapter Contents}

\begin{itemize}
    \setlength\itemsep{2em}
    \item[\textcolor{NavyBlue}{\textbullet}] 11.1 Indefinite Integrals
    \item[\textcolor{NavyBlue}{\textbullet}] 11.2 Differential Equations
    \item[\textcolor{NavyBlue}{\textbullet}] 11.3 Definite Integrals
    \item[\textcolor{NavyBlue}{\textbullet}] 11.4 The Area Under a Curve
    \item[\textcolor{NavyBlue}{\textbullet}] 11.5 The Area Between Two Curves
\end{itemize}

Integration skills covered in this chapter will be extended upon in the later chapter of \textbf{Further Calculus}.

\newpage

\section{Indefinite Integrals}\label{indefinite-integrals}

The \textit{derivative} of the function \(f(x)=x^3-4x^2-7x+3\) is obtained by \textit{multiplying by the power} and then \textit{reducing the power by one} for each term.

\vspace{-0.5cm}

\begin{align*}
    \text{If }f(x)&=x^3-4x^2+3\\
    \text{Then }f'(x)&=3x^2-8x
\end{align*}

\vspace{-0.5cm}

\textit{Integration} can be considered the \textit{inverse} of differentiation, with the aim of taking the \textit{derivative} \(f'(x)\) and determining the \textit{original function} \(f(x)\). However, given only the derivative \(f'(x)=3x^2-8x\) it would not be possible to know any constant term the original function \(f(x)\) contained. Finding the \textit{indefinite integral} of a function requires the inclusion of a \textcolor{NavyBlue}{\textbf{Constant of Integration}, $C$}:

\vspace{-0.5cm}

\begin{flalign*}
    \text{If }f'(x)&=3x^2-8x\\
    \text{Then }f(x)&=x^3-4x^2\color{NavyBlue}+C
\end{flalign*}

\vspace{-0.5cm}

\begin{center}
\begin{tcolorbox}[colback=red!5,width=14cm,colframe=red!70!black]
\textbf{Indefinite Integrals:}
\textcolor{black}{\textit{"The (indefinite) integral of }$x^n$\textit{ with respect to }$x$\textit{..."}}
    \begin{center}
        \[\int (x^n)\phantom{.}dx = \frac{x^{n+1}}{n+1}+C\]
    \end{center}
    In other words, \textbf{increase the power by 1} then \textbf{divide by the new power.}
\end{tcolorbox}
\end{center}

The \textit{integral sign}, \(\displaystyle\int\), should always appear accompanied by \(dx\), for a function in \(x\).

Note: \(\displaystyle{\int (f(x)+g(x))}\;dx=\displaystyle{\int f(x)\;dx+\int g(x\;dx)}\)

\begin{tcolorbox}[title=Example 11.1.1,colback=NavyBlue!2, colframe=NavyBlue]

Find $\displaystyle{\int (6x^2+8x)}\,dx $.

\tcblower

\vspace{-0.3cm}

\begin{align*}
    &\int (6x^2+8x)\,dx&&\\[0.5em]
    =\;&\frac{6x^3}{3}+\frac{8x^2}{2}+C&&\color{NavyBlue}\longleftarrow \text{Constant of integration}\\[0.5em]
    =\;&2x^3+4x^2+C&&\color{NavyBlue}\longleftarrow \text{Simplify}
\end{align*}

\end{tcolorbox}

\pagebreak

As with differentiation, \textit{preparation for integration} may be needed.

\begin{tcolorbox}[title=Example 11.1.2,colback=NavyBlue!2, colframe=NavyBlue]

Find $\displaystyle{\int \left(4\sqrt{x}-\frac{3}{x^2}\right)}\,dx,\;,x>0$.

\tcblower

\vspace{-0.3cm}

\begin{align*}
    &\int \left(4\sqrt{x}-\frac{3}{x^2}\right)\,dx&&\\[0.5em]
    =\;&\int \left(4x^{\frac{1}{2}}-3x^{-2}\right)\,dx&&\color{NavyBlue}\longleftarrow \text{Preparing to integrate}\\[0.5em]
    =\;&\frac{4x^{\frac{3}{2}}}{\frac{3}{2}}-\frac{3x^{-1}}{-1}+C&&\color{NavyBlue}\longleftarrow \text{Integrate: }\frac{1}{2}+1=\frac{1}{2}+\frac{2}{2}=\frac{3}{2}\\[0.5em]
    =\;&\frac{8}{3}x^{\frac{3}{2}}+3x^{-1}+C&&\color{NavyBlue}\longleftarrow \text{Simplify: }4\div\frac{3}{2}=\frac{4}{1}\times\frac{2}{3}=\frac{8}{3}
\end{align*}

\end{tcolorbox}

The \textit{derivative} of a \textit{linear} term is a \textit{constant}, so the integral of a \textit{constant} is \textit{linear}.

\begin{tcolorbox}[title=Example 11.1.3,colback=NavyBlue!2, colframe=NavyBlue]

Find $\displaystyle{\int \left((2x-1)(x+3)\right)}\,dx$.

\tcblower

\vspace{-0.3cm}

\begin{align*}
    &\int \left((2x-1)(x+3)\right)\,dx&&\\[0.5em]
    =\;&\int \left(2x^2-5x-3\right)\,dx&&\color{NavyBlue}\longleftarrow \text{Preparing to integrate}\\[0.5em]
    =\;&\frac{2x^3}{3}-\frac{5x^2}{2}-3x+C&&\color{NavyBlue}\longleftarrow \text{Integral of }-3\text{ is }-3x
\end{align*}

\end{tcolorbox}

Integration \textit{with respect to variables other than} \(x\) should not use \(dx\), and the notation used should match the variable of the function instead.

\begin{tcolorbox}[title=Example 11.1.4,colback=NavyBlue!2, colframe=NavyBlue]

Find $\displaystyle{\int \left(3t^2-5\right)}\,dt$.

\tcblower

\vspace{-0.3cm}

\begin{align*}
    &\int \left(3t^2-5\right)\,dt&&\\[0.5em]
    =\;&\frac{3t^3}{3}-5t+C&&\\[0.5em]
    =\;&t^3-5t+C&&
\end{align*}

\end{tcolorbox}

\pagebreak

\section{Differential Equations}\label{differential-equations}

\pagebreak

\section{Definite Integrals}\label{definite-integrals}

\vspace{-0.5cm}

The indefinite integral of the function \(f(x)\) can be notated as \(F(x)\):

\[\int f(x)\;dx=F(x)\]

The \textit{definite integral} of \(f(x)\) from \(a\) to \(b\) is the difference between \(F(b)\) and \(F(a)\):

\[\int_a^b f(x)\;dx=F(b)-F(a)\]

Together, they form a core part of the \textbf{Fundamental Theorem of Calculus}:

\begin{center}
\begin{tcolorbox}[colback=red!5,width=14cm,colframe=red!70!black]
\textbf{Fundamental Theorem of Calculus:}

\[\displaystyle{\int_a^b f(x)\;dx=F(b)-F(a)}\]
\centering
where 
\[\displaystyle{F(x) = \int f(x)\;dx}\]

\end{tcolorbox}
\end{center}

Since any \textit{constant of integration} within \(F(x)\) will cancel through subtraction (\(C-C\)), it is \textit{not included} when calculating a \textit{definite integral}.

\begin{tcolorbox}[title=Example 11.3.1,colback=NavyBlue!2, colframe=NavyBlue]

Find $\displaystyle{\int_1^2 \left(3x^2+6x-2\right)}\,dx$.

\tcblower

\vspace{-0.3cm}

\begin{align*}
    &\displaystyle{\int_1^2 \left(3x^2+6x-2\right)}\,dx&&\\[0.5em]
    =\;&\left[\frac{3x^3}{3}+\frac{6x^2}{2}-2x\right]_1^2&&\color{NavyBlue}\longleftarrow \text{Integrate}\\[0.5em]
    =\;&\left[x^3+3x^2-2x\right]_1^2&&\color{NavyBlue}\longleftarrow \text{Simplify}\\[0.5em]
    =\;&\left((2)^3+3(2)^2-2(2)\right)-\left((1)^3+5(1)^2-2(1)\right)&&\color{NavyBlue}\longleftarrow \text{Substitute}\\[0.5em]
    =\;&16-4\\[0.5em]
    =\;&12&&\color{NavyBlue}\longleftarrow \text{Evaluate}
\end{align*}

\end{tcolorbox}

\pagebreak

\section{The Area Under a Curve}\label{the-area-under-a-curve}

\vspace{-0.5cm}

The area shaded below, with each \textit{"bar"} having width \(1\) unit, is given by the \textit{sum} \(1+2+3+1+2\).

\begin{center}
    \begin{tikzpicture}
        \draw[fill=blue!20,domain=1:3,smooth] (0.6,0) rectangle (1.2,0.6);
        \draw[fill=blue!20,domain=1:3,smooth] (1.2,0) rectangle (1.8,1.2);
        \draw[fill=blue!20,domain=1:3,smooth] (1.8,0) rectangle (2.4,1.8);
        \draw[fill=blue!20,domain=1:3,smooth] (2.4,0) rectangle (3,0.6);
        \draw[fill=blue!20,domain=1:3,smooth] (3,0) rectangle (3.6,1.2);
        \draw[-stealth] (-1,0) -- (4.5,0) node[below] {$x$};
        \draw[=stealth] (0,-0.5) -- (0,2.5) node[left] {$y$};
        \foreach \x in {1,2,...,6}{
        \node[below] at (\x*0.6,0) {\x};
        }
        \node[left] at (0,0.6) {$1$};
        \node[left] at (0,1.2) {$2$};
        \node[left] at (0,1.8) {$3$};
        \node[below left] {$0$};
    \end{tikzpicture}
\end{center}

An integral, \(\displaystyle{\int}\), can be seen as \textit{"sum"}, but for a continuous function. Given a function \(f(x)\), the area enclosed between a section of the curve \(y=f(x)\) and the \(x\)-axis can be calculated using a definite integral.

\begin{center}
\begin{tcolorbox}[colback=red!5,width=16cm,colframe=red!70!black]
\textbf{Area Under a Curve:}

\vspace{0.2cm}

\begin{center}
    \begin{tikzpicture}
        \fill[blue!20,domain=1:3,smooth] (3,0) -- (1,0) -- (1,1) -- (3,1.6) -- cycle;
        \fill[red!5,domain=1:3,smooth] plot (\x,0.1*\x*\x-0.1*\x+1);
        \draw[-stealth] (-1,0) -- (5,0) node[below] {$x$};
        \draw[=stealth] (0,-0.5) -- (0,3) node[left] {$y$};
        \draw[smooth,domain=-0.3:
        4,yscale=0.1] plot (\x,\x*\x-\x+10) node[right] {$y=f(x)$};
        \draw[dashed] (1,0) -- (1,1);
        \draw[dashed] (3,0) -- (3,1.6);
        \node at (1,-0.3) {$a$};
        \node at (3,-0.3) {$b$};
        \node at (10,1.2) {Area $\displaystyle{=\int_a^b f(x)\;dx=F(b)-F(a)}$};
    \end{tikzpicture}
\end{center}

\end{tcolorbox}
\end{center}

The values of \(a\) and \(b\) can be referred to as the \textit{bounds} of the integration.

\begin{tcolorbox}[title=Example 11.4.1,colback=NavyBlue!2, colframe=NavyBlue]

Part of the graph of $y=5+2x-x^2$ is shown. Calculate the shaded area.

\begin{center}
    \begin{tikzpicture}
        \fill[blue!20,smooth,domain=-1:2] plot (\x,0.25*5+0.5*\x-0.25*\x*\x);
        \fill[blue!20] (-1,0) -- (-1,0.5) -- (2,1.25) -- (2,0) -- cycle; 
        \draw[-stealth] (-2.5,0) -- (4.5,0) node[below] {$x$};
        \draw[-stealth] (0,-1) -- (0,2) node[left] {$y$};
        \node[below left] {$0$};
        \draw[smooth,domain=-1.8:3.8,yscale=0.25] plot (\x,5+2*\x-\x*\x) node[below] {$y=5+2x-x^2$};
        \draw (-1,0) -- (-1,0.5);
        \draw (2,0) -- (2,1.25); 
        \node at (-1,-0.3) {$-1$};
        \node at (2,-0.3) {$2$};
    \end{tikzpicture}
\end{center}

\tcblower


\begin{align*}
    &\displaystyle{\int_{-1}^2 \left(5+2x-x^2\right)}\,dx&&\color{NavyBlue}\longleftarrow \text{Area}\\[0.5em]
    =\;&\left[5x+x^2-\frac{x^3}{3}\right]_{-1}^2&&\color{NavyBlue}\longleftarrow \text{Integrate}\\[0.5em]
    =\;&\left(5(2)+(2)^2-\frac{(2)^3}{3}\right)-\left(5(-1)+(-1)^2-\frac{(-1)^3}{3}\right)&&\color{NavyBlue}\longleftarrow \text{Substitute}\\[0.5em]
    =\;&\frac{34}{3}-\left(-\frac{11}{3}\right)&&\color{NavyBlue}\longleftarrow \text{Evaluate}\\[0.5em]
    =\;&15\text{ square units}
\end{align*}

\end{tcolorbox}

\newpage

Where the area between a curve and the \(x\)-axis lies \textit{under} the \(x\)-axis, the definite integral will be negative.

\begin{center}
    \begin{tikzpicture}
        \fill[green!25,smooth,domain=1:3,yscale=0.25] plot (\x,\x*\x*\x-10*\x*\x+27*\x-18);
        \fill[red!25,smooth,domain=3:6,yscale=0.25] plot (\x,\x*\x*\x-10*\x*\x+27*\x-18);
        \draw[-stealth] (-0.5,0) -- (7.5,0) node[below] {$x$};
        \draw[-stealth] (0,-2) -- (0,2.5) node[left] {$y$};
        \draw[smooth,domain=0.6:6.2,yscale=0.25] plot (\x,\x*\x*\x-10*\x*\x+27*\x-18) node[right] {$y=f(x)$};
        \draw[stealth-] (2.8,0.6) -- (3.5,1.6) node[above] {$\displaystyle{\int_a^b f(x)\;dx=A_1}$};
        \node at (2,0.5) {$A_1$};
        \node at (4.5,-1) {$A_2$};
        \draw[stealth-] (6,-1) -- (7,-1.5) node[right] {$\displaystyle{\int_b^c f(x)\;dx=-A_2}$};
        \draw[fill] (1,0) circle (0.03);
        \draw[fill] (3,0) circle (0.03);
        \draw[fill] (6,0) circle (0.03);
        \node at (1.1,-0.2) {\small{$a$}};
        \node at (2.9,-0.2) {\small{$b$}};
        \node at (6.1,-0.2) {\small{$c$}};
    \end{tikzpicture}
\end{center}

To avoid \textit{"positve"} and \textit{"negative"} areas cancelling each other out, such sections must be calculated as separate integrals, and their \textit{absolute values} added.

\[\text{Area }=A_1+A_2\]

\begin{tcolorbox}[title=Example 11.4.2,colback=NavyBlue!2, colframe=NavyBlue]

Part of the graph of $y=x^3-10x^2+27x-18$ is shown. Calculate the shaded area.

\begin{center}
    \begin{tikzpicture}
        \fill[blue!20,smooth,domain=1:3,yscale=0.25] plot (\x,\x*\x*\x-10*\x*\x+27*\x-18);
        \fill[blue!20,smooth,domain=3:6,yscale=0.25] plot (\x,\x*\x*\x-10*\x*\x+27*\x-18);
        \draw[-stealth] (-0.5,0) -- (7.5,0) node[below] {$x$};
        \draw[-stealth] (0,-2) -- (0,2) node[left] {$y$};
        \draw[smooth,domain=0.6:6.2,yscale=0.25] plot (\x,\x*\x*\x-10*\x*\x+27*\x-18) node[above right] {$y=x^3-10x^2+27x-18$};
        \draw[fill] (1,0) circle (0.03);
        \draw[fill] (3,0) circle (0.03);
        \draw[fill] (6,0) circle (0.03);
        \node at (1.1,-0.2) {\small{$1$}};
        \node at (2.9,-0.2) {\small{$3$}};
        \node at (6.1,-0.2) {\small{$6$}};
    \end{tikzpicture}
\end{center}

\tcblower


\begin{align*}
    &\displaystyle{\int_1^3 \left(x^3-10x^2+27x-18\right)}\,dx&&\displaystyle{\int_3^6 \left(x^3-10x^2+27x-18\right)}\,dx\\[0.5em]
    &=\left[\frac{x^4}{4}-\frac{10x^3}{3}+\frac{27x^2}{2}-18x\right]_1^3&&=\left[\frac{x^4}{4}-\frac{10x^3}{3}+\frac{27x^2}{2}-18x\right]_3^6\\[0.5em]
    &=\left(\frac{(3)^4}{4}-\frac{10(3)^3}{3}+\frac{27(3)^2}{2}-18(3)\right)&&=\left(\frac{(6)^4}{4}-\frac{10(6)^3}{3}+\frac{27(6)^2}{2}-18(6)\right)\\[0.5em]
    &-\left(\frac{(1)^4}{4}-\frac{10(1)^3}{3}+\frac{27(1)^2}{2}-18(1)\right)&&-\left(\frac{(3)^4}{4}-\frac{10(3)^3}{3}+\frac{27(3)^2}{2}-18(3)\right)\\[0.5em]
    &=-\frac{9}{4}-\left(-\frac{91}{12}\right)&&=-18-\left(-\frac{9}{4}\right)\\[0.5em]
    &=\frac{16}{3}&&=-\frac{63}{4}
\end{align*}
\vspace{0.3cm}
\hspace{3cm}
$\therefore\text{ Area }=\dfrac{16}{3}+\dfrac{63}{4}=\dfrac{253}{12}\text{ square units}$

\end{tcolorbox}

\pagebreak

\section{The Area Between Two Curves}\label{the-area-between-two-curves}

\vspace{-0.5cm}

The area \textbf{between} two curves can be calculated using the subtraction of one definite integral from another:

\begin{multicols}{3}

\begin{center}
    \begin{tikzpicture}
        \fill[blue!20,domain=1:3,smooth] (3,0) -- (1,0) -- (1,1) -- (3,1.6) -- cycle;
        \fill[blue!20,domain=1:3,smooth] plot (\x,-0.1*\x*\x+0.7*\x+0.4);
        \draw[-stealth] (-0.5,0) -- (4.5,0) node[below] {$x$};
        \draw[=stealth] (0,-0.5) -- (0,2.5) node[left] {$y$};
        \draw[smooth,domain=-0.3:
        3.5,yscale=0.1] plot (\x,\x*\x-\x+10) node[above right] {$g(x)$};
        \draw[smooth,domain=-0.3:
        3.5,yscale=0.1] plot (\x,-\x*\x+7*\x+4) node[right] {$f(x)$};
        \draw[dashed] (1,0) -- (1,1);
        \draw[dashed] (3,0) -- (3,1.6);
        \node at (1,-0.3) {$a$};
        \node at (3,-0.3) {$b$};
        \node at (2,-1.3) {$\displaystyle{\int_a^b f(x)\;dx}$};
    \end{tikzpicture}
\end{center}

    \begin{center}
    \begin{tikzpicture}
        \fill[blue!20,domain=1:3,smooth] (3,0) -- (1,0) -- (1,1) -- (3,1.6) -- cycle;
        \fill[pattern color = NavyBlue,pattern=north west lines,domain=1:3,smooth] (3,0) -- (1,0) -- (1,1) -- (3,1.6) -- cycle;
        \fill[white,domain=1:3,smooth] plot (\x,0.1*\x*\x-0.1*\x+1);
        \draw[-stealth] (-0.5,0) -- (4.5,0) node[below] {$x$};
        \draw[=stealth] (0,-0.5) -- (0,2.5) node[left] {$y$};
        \draw[smooth,domain=-0.3:
        3.5,yscale=0.1] plot (\x,\x*\x-\x+10) node[above right] {$g(x)$};
        \draw[smooth,domain=-0.3:
        3.5,yscale=0.1] plot (\x,-\x*\x+7*\x+4) node[right] {$f(x)$};
        \draw[dashed] (1,0) -- (1,1);
        \draw[dashed] (3,0) -- (3,1.6);
        \node at (1,-0.3) {$a$};
        \node at (3,-0.3) {$b$};
        \node at (2,-1.3) {$\displaystyle{\int_a^b g(x)\;dx}$};
    \end{tikzpicture}
\end{center}

\begin{center}
    \begin{tikzpicture}
        \fill[blue!20,domain=1:3,smooth] plot (\x,-0.1*\x*\x+0.7*\x+0.4);
        \fill[blue!20,domain=1:3,smooth] plot (\x,0.1*\x*\x-0.1*\x+1);
        \draw[-stealth] (-0.5,0) -- (4.5,0) node[below] {$x$};
        \draw[=stealth] (0,-0.5) -- (0,2.5) node[left] {$y$};
        \draw[smooth,domain=-0.3:
        3.5,yscale=0.1] plot (\x,\x*\x-\x+10) node[above right] {$g(x)$};
        \draw[smooth,domain=-0.3:
        3.5,yscale=0.1] plot (\x,-\x*\x+7*\x+4) node[right] {$f(x)$};
        \draw[dashed] (1,0) -- (1,1);
        \draw[dashed] (3,0) -- (3,1.6);
        \node at (1,-0.3) {$a$};
        \node at (3,-0.3) {$b$};
        \node at (2,-1.3) {$\displaystyle{\int_a^b f(x)\;dx-\int_a^b g(x)\;dx}$};
    \end{tikzpicture}
\end{center}

\end{multicols}

\vspace{-0.2cm}

Note that \(\displaystyle{\int_a^b f(x)\;dx-\int_a^b g(x)\;dx=\int_a^b (f(x)-g(x))\;dx}\), leading to the following formula:

\begin{center}
\begin{tcolorbox}[colback=red!5,width=16cm,colframe=red!70!black]

\textbf{Area Between Two Curves:}

\vspace{-0.5cm}

\begin{center}
    \begin{tikzpicture}
        \fill[blue!20,domain=1:3,smooth] plot (\x,-0.1*\x*\x+0.7*\x+0.4);
        \fill[blue!20,domain=1:3,smooth] plot (\x,0.1*\x*\x-0.1*\x+1);
        \draw[-stealth] (-0.5,0) -- (4.5,0) node[below] {$x$};
        \draw[=stealth] (0,-0.5) -- (0,2.5) node[left] {$y$};
        \draw[smooth,domain=-0.3:
        3.5,yscale=0.1] plot (\x,\x*\x-\x+10) node[above right] {$g(x)$ $\longleftarrow$ Lower function};
        \draw[smooth,domain=-0.3:
        3.5,yscale=0.1] plot (\x,-\x*\x+7*\x+4) node[right] {$f(x)$ $\longleftarrow$ Upper function};
        \draw[dashed] (1,0) -- (1,1);
        \draw[dashed] (3,0) -- (3,1.6);
        \node at (1,-0.3) {$a$};
        \node at (3,-0.3) {$b$};
        \node at (2,-1.4) {Area $=\displaystyle{\int_a^b (f(x)-g(x))\;dx}$ \quad\textbf{or}\quad Area $=\displaystyle{\int_a^b (\text{Upper}-\text{Lower})\;dx}$};
        \node at (-6,1) {\phantom{.}};
    \end{tikzpicture}
\end{center}

\end{tcolorbox}
\end{center}

\begin{tcolorbox}[title=Example 11.5.1,colback=NavyBlue!2, colframe=NavyBlue]

Part of the graphs of $y=-3x$ and $y=4-x^2$ are shown. Calculate the shaded area.

\begin{center}
    \begin{tikzpicture}
        \fill[blue!20,domain=-1:4] plot (\x,0.4-0.1*\x*\x);
        \draw[-stealth] (-3,0) -- (4.5,0) node[below] {$x$};
        \draw[-stealth] (0,-1.4) -- (0,1.4) node[left] {$y$};
        \draw[smooth,domain=-2.5:4.2,yscale=0.1] plot (\x,4-\x*\x);
        \draw[smooth,domain=-2.1:4.2,yscale=0.1] plot (\x,-3*\x);
        \node[above left] at (-2.1,0.63) {$y=-3x$};
        \node[below left] at (-2.5,-0.225) {$y=4-x^2$};
        \node[below left] {$0$};
        \draw[fill] (-1,0.3) circle (0.04);
        \draw[fill] (4,-1.2) circle (0.04);
        \draw[dashed] (-1,0.3) -- (-1,0) node[below] {$-1$};
        \draw[dashed] (4,-1.2) -- (4,0) node[above] {$4$};
    \end{tikzpicture}
\end{center}

\tcblower

\vspace{-0.2cm}

\begin{align*}
    &\displaystyle{\int_{-1}^4 \left(4-x^2-(-3x)\right)}\,dx&&\color{NavyBlue}\longleftarrow \text{Upper}-\text{Lower}\\[0.5em]
    &\displaystyle{\int_{-1}^4 \left(4-x^2+3x\right)}\,dx&&\color{NavyBlue}\longleftarrow \text{Simplify}\\[0.5em]
    =\;&\left[4x-\frac{x^3}{3}+\frac{3x^2}{2}\right]_{-1}^4&&\color{NavyBlue}\longleftarrow \text{Integrate}\\[0.5em]
    =\;&\left(4(4)-\frac{(4)^3}{3}+\frac{3(4)^2}{2}\right)-\left(4(-1)-\frac{(-1)^3}{3}+\frac{3(-1)^2}{2}\right)&&\color{NavyBlue}\longleftarrow \text{Substitute}\\[0.5em]
    =\;&\frac{56}{3}-\left(-\frac{13}{6}\right)&&\color{NavyBlue}\longleftarrow \text{Evaluate}\\[0.5em]
    =\;&\frac{125}{6}\text{ square units}
\end{align*}

\end{tcolorbox}

\newpage

\section*{Review Exercise}\label{review-exercise-10}
\addcontentsline{toc}{section}{Review Exercise}

\chapterfont{\color{white}}

\chapter{Compound Angles}\label{compound-angles}

\vspace{-12cm}
\begin{center}
\begin{tikzpicture}
\draw[white] (10,5) circle (0.01);
\draw[Mulberry,rounded corners,very thick] (1,0.2) rectangle (5,1.8);
\draw[Mulberry,very thick] (5,1) -- (6,1);
\draw[Mulberry,fill=Mulberry,rounded corners] (6,1.8) rectangle (18.2,0.2);
\node[white] at (12.1,1) {\Huge{\textsc{Compound Angles}}};
\node at (3,1) {\Large{\textsc{Chapter 12}}};
\end{tikzpicture}
\end{center}

\section{The Addition Formulae}\label{the-addition-formulae}

\vspace{-0.5cm}

It should be noted that \(\sin{(A+B)}\ne \sin{A}+\sin{B}\):

\vspace{-0.3cm}

\begin{align*}
    &\sin{(60\degree+30\degree)}=\sin{90\degree}=1\\[0.5em]
    \textbf{but }&\sin{60\degree}+\sin{30\degree}=\frac{\sqrt{3}}{2}+\frac{1}{2}=\frac{\sqrt{3}+1}{2}
\end{align*}

\vspace{-0.3cm}

Instead, the following expansions are provided in the formula sheet, known as the \textbf{addition formulae}:

\vspace{-0.3cm}

\begin{center}
\begin{tcolorbox}[colback=red!5,colframe=red!70!black]
\centering
\textbf{Addition Formulae} \textit{(as provided on the formula sheet)}:
\begin{align*}
    \sin{(A\pm B)}&=\sin{A}\cos{B}\pm\cos{A}\sin{B}\\[1em]
    \cos{(A\pm B)}&=\cos{A}\cos{B}\mp\sin{A}\sin{B}
\end{align*}

\end{tcolorbox}
\end{center}

\vspace{-0.3cm}

They can be expanded into four formulae, covering addition and subtraction for each of \(\sin\) and \(\cos\):

\vspace{-0.5cm}

\begin{multicols}{2}

\begin{align*}
    \sin{(A+B)}&=\sin{A}\cos{B}+\cos{A}\sin{B}\\[0.5em]
    \sin{(A-B)}&=\sin{A}\cos{B}-\cos{A}\sin{B}
\end{align*}

\begin{align*}
    \cos{(A+B)}&=\cos{A}\cos{B}-\sin{A}\sin{B}\\[0.5em]
    \cos{(A-B)}&=\cos{A}\cos{B}+\sin{A}\sin{B}
\end{align*}

\end{multicols}

\vspace{-0.3cm}

They can be used to expand the usage of known \textit{exact values} (0\degree,30\degree,45\degree,60\degree,90\degree,180\degree,270\degree,360\degree).

\begin{tcolorbox}[title=Example 12.1.1,colback=Mulberry!1, colframe=Mulberry]

Evaluate $\sin{105}$.

\tcblower

\vspace{-0.3cm}

\begin{align*}
   \sin{105\degree}=\;&\sin{(60\degree+45\degree)}&&\color{Mulberry}\longleftarrow \text{Relate to known exact values}\\[0.5em]
    =\;&\sin{60\degree}\cos{45\degree}+\cos{60\degree}\sin{45\degree}&&\color{Mulberry}\longleftarrow \text{Expand using formula}\\[0.5em]
    =\;&\frac{\sqrt{3}}{2}\times\frac{1}{\sqrt{2}}+\frac{1}{2}\times\frac{1}{\sqrt{2}}&&\color{Mulberry}\longleftarrow \text{Substitute exact values}\\[0.5em]
    =\;&\frac{\sqrt{3}}{2\sqrt{2}}+\frac{1}{2\sqrt{2}}&&\\[0.5em]
    =\;&\frac{\sqrt{3}+1}{2\sqrt{2}}&&\color{Mulberry}\longleftarrow \text{Simplify}
\end{align*}

\end{tcolorbox}

\begin{tcolorbox}[title=Example 12.1.2,colback=Mulberry!1, colframe=Mulberry]

Evaluate $\sin{40\degree}\cos{10}-\cos{40\degree}\sin{10\degree}$.

\tcblower

\vspace{-0.3cm}

\begin{align*}
    \sin{40\degree}\cos{10}-\cos{40\degree}\sin{10\degree}=\;&\sin{(40\degree-10\degree)}&&\color{Mulberry}\longleftarrow \text{Recognise expanded form of }\sin{(A-B)}\\[0.5em]
    =\;&\sin{30\degree}&&\\[0.5em]
    =\;&\frac{1}{2}&&\color{Mulberry}\longleftarrow \text{State exact value}
\end{align*}

\end{tcolorbox}

\pagebreak

\section{Applications to Right-Angled Triangles}\label{applications-to-right-angled-triangles}

\vspace{-0.5cm}

Given a right-angled triangle containing an angle \(x\), values of \(\sin{x}\), \(\cos{x}\) and \(\tan{x}\) can be stated using \textbf{SOHCATOA}. For example with Pythagoras' Theorem used to find a missing side length if needed.

\begin{center}
    \begin{tikzpicture}
        \draw (0,0) -- (5,0) -- (5,3) -- cycle;
        \draw (4.6,0) rectangle (5,0.4);
        \node[below] at (2.5,0) {$5$ cm};
        \node[right] at (5,1.5) {$3$ cm};
        \node at (1,0.3) {$x$};
        \draw (0.7,0) arc (0:31:0.7);
        \node[left] at (2.7,1.8) {\color{Mulberry}By Pythagoras: $\sqrt{5^2+3^2}=$\color{black}$\sqrt{34}$};
        \node at (10,1.5) {
        \begin{minipage}{5cm}
            \begin{align*}
                \sin{x}&=\frac{\text{opp}}{\text{hyp}}=\frac{3}{\sqrt{34}}\\[1em]
                \cos{x}&=\frac{\text{adj}}{\text{hyp}}=\frac{5}{\sqrt{34}}\\[1em]
                \tan{x}&=\frac{\text{opp}}{\text{adj}}=\frac{3}{5}
            \end{align*}
        \end{minipage}
        };
    \end{tikzpicture}
\end{center}

These skills are often required in combination with applications of the addition formulae.

\begin{tcolorbox}[title=Example 12.2.1,colback=Mulberry!1, colframe=Mulberry]

The diagram below shows angles $p$ and $q$ contained within right-angled. Find $\sin{(p+q)}$.

\begin{center}
    \begin{tikzpicture}
        \draw (0,0) -- (6,0) -- (0,2) -- cycle;
        \draw (0,0) -- (-4,2) -- (0,2) -- cycle;
        \draw (-3.3,2) arc (0:-26.6:0.7);
        \draw (5.3,0) arc (180:161.6:0.7);
        \node at (-3,1.75) {$p$};
        \node at (4.7,0.2) {$q$};
        \draw (0,0) rectangle (0.3,0.3);
        \draw (0,2) rectangle (-0.3,1.7);
        \node[below] at (3,0) {$6$ cm};
        \node[above] at (-2,2) {$4$ cm};
        \node[below left] at (-1.8,1.2) {$\sqrt{20}$ cm};
    \end{tikzpicture}
\end{center}

\tcblower

\vspace{-0.3cm}

\begin{align*}
    \text{Opposite side for }p&=2&&\color{Mulberry}\longleftarrow \text{By Pythagoras: }\sqrt{\sqrt{20}^2-4^2}=\sqrt{20-16}=\sqrt{4}=2\\[0.5em]
    \text{Hypotenuse for }q&=\sqrt{40}&&\color{Mulberry}\longleftarrow \text{By Pythagoras: }\sqrt{6^2+2^2}=\sqrt{36+4}=\sqrt{40}\\[0.5em]
\end{align*}

\centering
\textcolor{Mulberry}{Once all side length have been calculated, values of $\sin{p},\cos{p},\sin{q},\cos{q}$ can be obtained.}

\begin{align*}
    \sin{(p+q)}&=\sin{p}\cos{q}+\cos{p}\sin{q}&&\color{Mulberry}\longleftarrow \text{Expand addition formula}\\[0.5em]
    &=\frac{2}{\sqrt{20}}\times\frac{6}{\sqrt{40}}+\frac{4}{\sqrt{20}}\times\frac{2}{\sqrt{40}}&&\color{Mulberry}\longleftarrow \text{Substitute using SOHCAHTOA}\\[0.5em]
    &=\frac{12}{\sqrt{800}}+\frac{8}{\sqrt{800}}&&\\[0.5em]
    &=\frac{20}{20\sqrt{2}}&&\\[0.5em]
    &=\frac{1}{\sqrt{2}}&&\color{Mulberry}\longleftarrow \text{Simplify (and rationalise if required)}\\[0.5em]
\end{align*}

\end{tcolorbox}

Other expansions such as \(\cos{(p+q)}\) can be calculated similarly, and note that \(\tan{(p+q)}=\dfrac{\sin{(p+q)}}{\cos{(p+q)}}\).

\pagebreak

\section{The Double Angle Formulae}\label{the-double-angle-formulae}

Expansions for \(\sin{2x}\) and \(\cos{2x}\) in terms of \(\sin{x}\) and \(\cos{x}\) can be obtained from the addition formulae:

\begin{align*}
    \sin{2x}&\color{Mulberry}=\sin{(x+x)}&\cos{2x}&\color{Mulberry}=\cos{(x+x)}\\
    &\color{Mulberry}=\sin{x}\cos{x}+\cos{x}\sin{x}&&\color{Mulberry}=\cos{x}\cos{x}-\sin{x}\sin{x}\\
    &=2\sin{x}\cos{x}&&=\cos^2{x}-\sin^2{x}\\
    &&&=2\cos^2{x}-1\\
    &&&=1-2\sin^2{x}\\
\end{align*}

The other arrangements of the \(\cos{2x}\) expansion come from the trigonometric identify \(\sin^2{x}+\cos^2{x}=1\).
These expansions are provided in the formula sheet. Care is needed to determine \textit{whether} to expand a \textit{double angle} trig term and, in the case of \(\cos{2x}\), \textit{which} expansion to use.

\begin{tcolorbox}[title=Example 12.3.1,colback=Mulberry!1, colframe=Mulberry]

The diagram below shows angle $p$ contained within a right-angled triangle. Find $\sin{(2p)}$.

\begin{center}
    \begin{tikzpicture}
        \draw (0,0) -- (6,0) -- (0,2) -- cycle;
        \draw (0,0) -- (-4,2) -- (0,2) -- cycle;
        \draw (-3.3,2) arc (0:-26.6:0.7);
        \draw (5.3,0) arc (180:161.6:0.7);
        \node at (-3,1.75) {$p$};
        \node at (4.7,0.2) {$q$};
        \draw (0,0) rectangle (0.3,0.3);
        \draw (0,2) rectangle (-0.3,1.7);
        \node[below] at (3,0) {$6$ cm};
        \node[above] at (-2,2) {$4$ cm};
        \node[below left] at (-1.8,1.2) {$\sqrt{20}$ cm};
    \end{tikzpicture}
\end{center}

\end{tcolorbox}

\begin{tcolorbox}[title=Example 12.3.2,colback=Mulberry!1, colframe=Mulberry]

Given $\sin{q}=\dfrac{4}{11}$, where $0<q<\dfrac{\pi}{2}$, find $\cos{2q}$. 

\end{tcolorbox}

\section*{Review Exercise}\label{review-exercise-11}
\addcontentsline{toc}{section}{Review Exercise}

\chapterfont{\color{white}}

\chapter{The Circle}\label{the-circle}

\vspace{-12cm}
\begin{center}
\begin{tikzpicture}
\draw[white] (10,5) circle (0.01);
\draw[WildStrawberry,rounded corners,very thick] (1,0.2) rectangle (5,1.8);
\draw[WildStrawberry,very thick] (5,1) -- (6,1);
\draw[WildStrawberry,fill=WildStrawberry,rounded corners] (6,1.8) rectangle (18.2,0.2);
\node[white] at (12.1,1) {\Huge{\textsc{The Circle}}};
\node at (3,1) {\Large{\textsc{Chapter 13}}};
\end{tikzpicture}
\end{center}

\section{The Equation of a Circle}\label{the-equation-of-a-circle}

\newpage

\subsection*{Exercise}\label{exercise-44}
\addcontentsline{toc}{subsection}{Exercise}

\newpage

\section{The General Equation of a Circle}\label{the-general-equation-of-a-circle}

\newpage

\subsection*{Exercise}\label{exercise-45}
\addcontentsline{toc}{subsection}{Exercise}

\newpage

\section{The Intersection of a Line and a Circle}\label{the-intersection-of-a-line-and-a-circle}

\newpage

\subsection*{Exercise}\label{exercise-46}
\addcontentsline{toc}{subsection}{Exercise}

\newpage

\section{Finding the Equation of a Tangent to a Circle}\label{finding-the-equation-of-a-tangent-to-a-circle}

\newpage

\subsection*{Exercise}\label{exercise-47}
\addcontentsline{toc}{subsection}{Exercise}

\newpage

\section{Points Inside and Outside Circles}\label{points-inside-and-outside-circles}

\newpage

\subsection*{Exercise}\label{exercise-48}
\addcontentsline{toc}{subsection}{Exercise}

\newpage

\section{The Intersections of Circles}\label{the-intersections-of-circles}

\newpage

\subsection*{Exercise}\label{exercise-49}
\addcontentsline{toc}{subsection}{Exercise}

\newpage

\section*{Review Exercise}\label{review-exercise-12}
\addcontentsline{toc}{section}{Review Exercise}

\chapterfont{\color{white}}

\chapter{The Wave Function}\label{the-wave-function}

\vspace{-12cm}

\begin{center}
    \begin{tikzpicture}
        \draw[white] (10,5) circle (0.01);
        \draw[Mulberry,rounded corners,very thick] (1,0.2) rectangle (5,1.8);
        \draw[Mulberry,very thick] (5,1) -- (6,1);
        \draw[Mulberry,fill=Mulberry,rounded corners] (6,1.8) rectangle (18.2,0.2);
        \node[white] at (12.1,1) {\Huge{\textsc{The Wave Function}}};
        \node at (3,1) {\Large{\textsc{Chapter 14}}};
    \end{tikzpicture}
\end{center}

In this chapter, the sums and differences of \emph{equal-angled} trigonometric operations will be explored.

\[\text{e.g. }3\sin{x}\degree+4\cos{x}\degree\]

The graphs of basic trigonometric functions such as \(y=3\sin{x}\degree\) and \(y=4\cos{x}\degree\) should be familiar:

\begin{center}
    \begin{tikzpicture}
        \draw[-stealth] (0,0) -- (14,0) node[below] {$x$};
        \draw[-stealth] (0,-3) -- (0,3) node[left] {$y$};
        \draw[smooth,domain=0:2*3.141592,xscale=2] plot (\x,{0.5*3*sin(\x r)});
        \draw[smooth,domain=0:2*3.141592,xscale=2] plot (\x,{0.5*4*cos(\x r)});
        \node[above] at (4*3.141592,0) {$360\degree$};
        \foreach \y in {-5,-4,...,5}{
        \draw (0,0.5*\y) -- (-0.1,0.5*\y) node[left] {\y};
}
        \node[below] at (3.5*3.141592,-1.35) {$y=3\sin{x}\degree$};
        \node[above] at (4*3.141592,2) {$y=4\cos{x}\degree$};
    \end{tikzpicture}
\end{center}

The graph of \(\color{Mulberry}y=3\sin{x}\degree+4\cos{x}\degree\) is the same as that of \(\color{Mulberry}y=5\cos{(x-36.9)}\degree\).

\begin{center}
    \begin{tikzpicture}
        \draw[-stealth] (0,0) -- (14,0) node[below] {$x$};
        \draw[-stealth] (0,-3) -- (0,3) node[left] {$y$};
        \draw[black!50,smooth,domain=0:2*3.141592,xscale=2] plot (\x,{0.5*3*sin(\x r)});
        \draw[black!50,smooth,domain=0:2*3.141592,xscale=2] plot (\x,{0.5*4*cos(\x r)});
        \draw[smooth,very thick,Mulberry,domain=0:2*3.141592,xscale=2] plot (\x,{0.5*4*cos(\x r)+0.5*3*sin(\x r)});
        \node[above] at (4*3.141592,0) {$360\degree$};
        \foreach \y in {-5,-4,...,5}{
        \draw (0,0.5*\y) -- (-0.1,0.5*\y) node[left] {\y};
}
        \node[black!50,below] at (3.5*3.141592,-1.35) {$y=3\sin{x}\degree$};
        \node[black!50,above] at (4*3.141592,2) {$y=4\cos{x}\degree$};
        \node[above,Mulberry] at (1.3*3.141592,2.5) {$y=3\sin{x}\degree+4\cos{x}\degree=5\cos{(x-36.9)}\degree$};
        \draw[dashed,Mulberry] (2*0.644,2.5) -- (2*0.644,0) node[below] {$36.9\degree$};
    \end{tikzpicture}
\end{center}

Determining that \(3\sin{x}\degree+4\cos{x}\degree=5\cos{(x-36.9)}\degree\) allows the maximum values and minimum values of \(3\sin{x}\degree+4\cos{x}\degree\) to be determined and the values of \(x\) which produce them, and allow equations such as \(3\sin{x}\degree+4\cos{x}\degree=2\) to be solved.

\emph{Any} trigonometric expression \(k_1\sin{x}\pm k_2\cos{x}\) can be written as either \(k\sin{(x\pm a)}\) or \(k\cos{(x\pm a)}\).

This chapter will cover how to do this and explore the various applications of this property.

\pagebreak

\section{\texorpdfstring{The Wave Function using \(k\cos{(x-a)}\)}{The Wave Function using k\textbackslash cos\{(x-a)\}}}\label{the-wave-function-using-kcosx-a}

\vspace{-0.5cm}

Whilst any of \(k\sin{(x\pm a)}\) or \(k\cos{(x\pm a)}\) may be used in general, any Higher exam question is likely to specify a particular form to use. The simplest is often \(k\cos{(x-a)}\).

To express \(3\sin{x}\degree+4\cos{x}\degree\) in the form \(k\cos{(x-a)}\degree\), two key \emph{trigonometric identities} are needed:

\[\sin^2{x}+\cos^2{x}=1\hspace{2cm}\text{and}\hspace{2cm}\tan{x}=\frac{\sin{x}}{\cos{x}}\]

First, \(k\cos{(x-a)}\degree\) can be expanded using the the formula \(\cos{(A\pm B)}=\cos{A}\cos{B}\mp \sin{A}\sin{B}\):

\[k\cos{(x-a)}\degree=k\cos{x}\degree\cos{a}\degree+k\sin{x}\degree\sin{a}\degree\]

It is required that this expansion is equal to the original expression:

\[3\underline{\sin{x}\degree}+4\doubleunderline{\cos{x}\degree}=k\doubleunderline{\cos{x}\degree}\cos{a}\degree+k\underline{\sin{x}\degree}\sin{a}\degree\]

Equating the coefficients of \(\underline{\sin{x}\degree}\) and \(\doubleunderline{\cos{x}\degree}\) gives a set of simultaneous equations in \(k\) and \(a\):

\begin{align*}
    k\sin{a}\degree&=3&&\color{Mulberry}\longleftarrow \text{Coefficients of }\underline{\sin{x}\degree}\\
    k\cos{a}\degree&=4&&\color{Mulberry}\longleftarrow \text{Coefficients of }\doubleunderline{\cos{x}\degree}
\end{align*}

Squaring both sides of each equation and adding allows \(k\) to be determined using \(\sin^2{x}+\cos^2{x}=1\):

\begin{align*}
    k^2\sin^2{a}\degree+k^2\cos^2{a}\degree&=3^2+4^2\\
    k^2\left(\sin^2{x}\degree+\cos^2{x}\degree\right)&=3^2+4^2\\
    k^2(1)&=3^2+4^2\\
    \color{Mulberry}k&\color{Mulberry}=\sqrt{3^2+4^2}\\
    k&=\sqrt{25}\\
    k&=5
\end{align*}

Dividing \(k\sin{a}\degree\) by \(k\cos{a}\degree\) allows \(a\) to be calculated, noting \(k\sin{a}\degree\) and \(k\cos{a}\degree\) are both positive:

\vspace{2.8cm}
\hspace{14cm}
\begin{tikzpicture}[remember picture, overlay, scale=0.5]
        \draw (-1.5,0) -- (1.5,0);
        \draw (0,-1) -- (0,1);
        \node at (0.7,0.5) {A};
        \node at (-0.7,0.5) {S};
        \node at (-0.7,-0.5) {T};
        \node at (0.7,-0.5) {C};
        \node[Mulberry,above right] at (0.7,0.65) {\scriptsize{$a\degree$}};
        \node[Mulberry,right] at (0.7,0.5) {\ding{51}\ding{51}};
        \node[Mulberry,left] at (-0.7,0.5) {\ding{51}};
        \node[Mulberry,right] at (0.7,-0.5) {\ding{51}};
\end{tikzpicture}

\vspace{-3.8cm}

\begin{align*}
    \color{Mulberry}\frac{k\sin{a}\degree}{k\cos{a}\degree}&\color{Mulberry}=\frac{3}{4}\\
    \tan{a}\degree&=\frac{3}{4}\\
    a\degree_{\text{acute}}&=\tan^{-1}{\left(\frac{3}{4}\right)}=36.9\degree\\
    a\degree&=36.9\degree
\end{align*}

Hence \(3\sin{x}\degree+4\cos{x}\degree=5\cos{(x-36.9)}\degree\). Here it has been assumed that \(k>0\) and \(0<a<360\).

\pagebreak

Some abbreviations of the working shown on the previous page are routinely permitted in the Higher exam. The following example demonstrates an appropriate level of detail in its solution.

\begin{tcolorbox}[title=Example,colback=Mulberry!1!, colframe=Mulberry]

Express $3\cos{x}-\sin{x}$ in the form $k\cos{(x-a)}$ where $k>0$ and $0<a<2\pi$.

\tcblower

\vspace{10cm}
\hspace{2.2cm}
\begin{tikzpicture}[remember picture, overlay, scale=0.5]
        \draw[white] (-8,0) -- (8,0);
        \draw (-1.5,0) -- (1.5,0);
        \draw (0,-1) -- (0,1);
        \node at (0.7,0.5) {A};
        \node at (-0.7,0.5) {S};
        \node at (-0.7,-0.5) {T};
        \node at (0.7,-0.5) {C};
        \node[Mulberry,below right] at (0.7,-0.65) {\scriptsize{$360\degree-\text{acute}\degree$}};
        \node[Mulberry,right] at (0.7,-0.5) {\ding{51}\ding{51}};
        \node[Mulberry,right] at (0.7,0.5) {\ding{51}};
        \node[Mulberry,left] at (-0.7,-0.5) {\ding{51}};
\end{tikzpicture}

\vspace{-4cm}
\hspace{8cm}
\begin{tikzpicture}[remember picture, overlay]
        \draw[thick,Mulberry,-stealth] (160:8) arc (160:200:8);
\end{tikzpicture}

\vspace{-7.5cm}
\begin{align*}
    3\cos{x}-\sin{x}&=k\cos{(x-a)}&&\\[0.5em]
    3\doubleunderline{\cos{x}}-\underline{\sin{x}}&=k\doubleunderline{\cos{x}}\cos{a}+k\underline{\sin{x}}\sin{a}&&\color{Mulberry}\longleftarrow \text{Expand }k\cos{(x-a)}\\[1.5em]
    k\sin{a}&=-1&&\color{Mulberry}\longleftarrow \text{Equate }\sin{x}\text{ coefficients}\\[0.5em]
    k\cos{a}&=3&&\color{Mulberry}\longleftarrow \text{Equate }\cos{x}\text{ coefficients}\\[1.5em]
    k&=\sqrt{(-1)^2+3^2}&&\color{Mulberry}\longleftarrow \text{Calculate }k\\[0.5em]
    &=\sqrt{10}\\[1em]
    \tan{a}&=\frac{-1}{3}&&\color{Mulberry}\longleftarrow \text{Find }\tan{a}\\[0.5em]
    \text{acute}\degree&=\tan^{-1}{\left(\frac{1}{3}\right)}=18.4\degree &&\color{Mulberry}\longleftarrow \text{Find acute angle first}\\[1.5em]
    &&&\color{Mulberry}\longleftarrow \text{Negative }\sin{a}\text{: quadrants T and C}\\[0.5em]
    &&&\color{Mulberry}\longleftarrow \text{Positive }\cos{a}\text{: quadrants A and C}\\[2em]
    a\degree &=341.6\degree&&\color{Mulberry}\longleftarrow \text{Using }360\degree-18.4\degree\text{ (Double-ticked quadrant)}\\[0.5em]
    a &=5.96&&\color{Mulberry}\longleftarrow \text{Convert to radians: }341.6\times\frac{\pi}{180}\\[1.5em]
    3\cos{x}-\sin{x}&=\sqrt{10}\cos{(x-5.96)}&&\color{Mulberry}\longleftarrow \text{State solution}
\end{align*}

\end{tcolorbox}

\vspace{-0.5cm}

\subsubsection*{Exercise 14.1}\label{exercise-14.1}
\addcontentsline{toc}{subsubsection}{Exercise 14.1}

\vspace{-0.3cm}

\begin{enumerate}
    \item Express each of the following in the form $k\cos{(x-a)}\degree$, where $k>0$ and $0<a<360$. \faCalculator
    \vspace{-0.2cm}
    \begin{enumerate}
        \begin{multicols}{3}
            \item $5\cos{x}\degree+12\sin{x}\degree$
            \item $4\sin{x}\degree+5\cos{x}\degree$
            \item $6\cos{x}\degree+\sin{x}\degree$
        \end{multicols}
    \end{enumerate}
    \item Express each of the following in the form $k\cos{(x-a)}$, where $k>0$ and $0<a<2\pi$. \faCalculator
    \vspace{-0.2cm}
    \begin{enumerate}
        \begin{multicols}{3}
            \item $8\sin{x}-6\cos{x}$
            \item $\cos{x}-3\sin{x}$
            \item $-2\sin{x}-\cos{x}$
        \end{multicols}
    \end{enumerate}
    \item Express each of the following in the form $k\cos{(x-a)}\degree$, where $k>0$ and $0<a<360$.
    \vspace{-0.2cm}
    \begin{enumerate}
        \begin{multicols}{3}
            \item $\sin{x}\degree+\cos{x}\degree$
            \item $\cos{x}\degree-\sqrt{3}\sin{x}\degree$
            \item $-\sin{x}\degree-\sqrt{3}\cos{x}\degree$
        \end{multicols}
    \end{enumerate}
\end{enumerate}

\pagebreak

\section{Other Forms of the The Wave Function}\label{other-forms-of-the-the-wave-function}

\vspace{-0.5cm}

As well as \(k\cos{(x-a)}\), three other forms may be used, shown below including their expansions:

\vspace{-1.0cm}

\begin{align*}
    k\cos{(x+a)}&\color{Mulberry}=k\cos{x}\cos{a}-k\sin{x}\sin{a}\\
    k\sin{(x+a)}&\color{Mulberry}=k\sin{x}\cos{a}+k\sin{x}\cos{a}\\
    k\sin{(x-a)}&\color{Mulberry}=k\sin{x}\cos{a}-k\sin{x}\cos{a}
\end{align*}

\vspace{-0.3cm}

\begin{tcolorbox}[title=Example,colback=Mulberry!1!, colframe=Mulberry]

Express $12\cos{x}\degree-5\sin{x}\degree$ in the form $k\sin{(x-a)}\degree$ where $k>0$ and $0<a<360$.

\tcblower

\vspace{9.7cm}
\hspace{3.2cm}
\begin{tikzpicture}[remember picture, overlay, scale=0.5]
        \draw[white] (-8,0) -- (8,0);
        \draw (-1.5,0) -- (1.5,0);
        \draw (0,-1) -- (0,1);
        \node at (0.7,0.5) {A};
        \node at (-0.7,0.5) {S};
        \node at (-0.7,-0.5) {T};
        \node at (0.7,-0.5) {C};
        \node[Mulberry,below left] at (-0.7,-0.65) {\scriptsize{$180\degree+\text{acute}\degree$}};
        \node[Mulberry,left] at (-0.7,-0.5) {\ding{51}\ding{51}};
        \node[Mulberry,right] at (0.7,-0.5) {\ding{51}};
        \node[Mulberry,left] at (-0.7,0.5) {\ding{51}};
\end{tikzpicture}

\vspace{-3.7cm}
\hspace{8.5cm}
\begin{tikzpicture}[remember picture, overlay]
        \draw[thick,Mulberry,-stealth] (160:8) arc (160:200:8);
\end{tikzpicture}

\vspace{-7.5cm}
\begin{align*}
    12\cos{x}\degree-5\sin{x}\degree&=k\sin{(x-a)}\degree&&\\[0.5em]
    12\doubleunderline{\cos{x}\degree}-5\underline{\sin{x}\degree}&=k\underline{\sin{x}\degree}\cos{a}\degree-k\doubleunderline{\cos{x}\degree}\sin{a}\degree&&\color{Mulberry}\longleftarrow \text{Expand }k\sin{(x-a)}\degree\\[1.5em]
    -k\sin{a}\degree&=12\implies k\sin{a}\degree=-12&&\color{Mulberry}\longleftarrow \text{Equate }\sin{x}\degree\text{ coefficients}\\[0.5em]
    k\cos{a}\degree&=-5&&\color{Mulberry}\longleftarrow \text{Equate }\cos{x}\degree\text{ coefficients}\\[1.5em]
    k&=\sqrt{(-12)^2+(-5)^2}&&\color{Mulberry}\longleftarrow \text{Calculate }k\\[0.5em]
    &=13\\[1.5em]
    \tan{a}\degree&=\frac{-12}{-5}=\frac{12}{5}&&\color{Mulberry}\longleftarrow \text{Find }\tan{a}\degree\\[0.5em]
    \text{acute}\degree&=\tan^{-1}{\left(\frac{12}{5}\right)}=67.4\degree &&\color{Mulberry}\longleftarrow \text{Find acute angle first}\\[0.8em]
    &&&\color{Mulberry}\longleftarrow \text{Negative }\sin{a}\degree\text{: quadrants T and C}\\[0.5em]
    &&&\color{Mulberry}\longleftarrow \text{Negative }\cos{a}\degree\text{: quadrants S and T}\\[1.5em]
    a\degree &=247.4\degree&&\color{Mulberry}\longleftarrow \text{Using }180\degree+67.4\degree\text{ (Double-ticked)}\\[1em]
    12\cos{x}\degree-5\sin{x}\degree&=13\sin{(x-247.4)}\degree&&\color{Mulberry}\longleftarrow \text{State solution}
\end{align*}

\end{tcolorbox}

\vspace{-0.5cm}

\subsubsection*{Exercise 14.2}\label{exercise-14.2}
\addcontentsline{toc}{subsubsection}{Exercise 14.2}

\vspace{-0.3cm}

\begin{enumerate}
    \item Express each of the following in the form $k\sin{(x+a)}\degree$, where $k>0$ and $0<a<360$. \faCalculator
    \vspace{-0.2cm}
    \begin{enumerate}
        \begin{multicols}{3}
            \item $3\cos{x}\degree+2\sin{x}\degree$
            \item $4\sin{x}\degree+3\cos{x}\degree$
            \item $7\cos{x}\degree-\sin{x}\degree$
        \end{multicols}
    \end{enumerate}
    \item Express each of the following in the form $k\cos{(x+a)}$, where $k>0$ and $0<a<2\pi$. \faCalculator
    \vspace{-0.2cm}
    \begin{enumerate}
        \begin{multicols}{3}
            \item $3\sin{x}+8\cos{x}$
            \item $-3\cos{x}-4\sin{x}$
            \item $-6\sin{x}+2\cos{x}$
        \end{multicols}
    \end{enumerate}
    \item Express each of the following in the form $k\sin{(x-a)}$, where $k>0$ and $0<a<2\pi$.
    \vspace{-0.2cm}
    \begin{enumerate}
        \begin{multicols}{3}
            \item $\sqrt{3}\sin{x}+\cos{x}$
            \item $\cos{x}-\sin{x}$
            \item $2\sqrt{3}\sin{x}-2\cos{x}$
        \end{multicols}
    \end{enumerate}
\end{enumerate}

\pagebreak

\section{Maximum and Minimum Values using the Wave Function}\label{maximum-and-minimum-values-using-the-wave-function}

\vspace{-0.5cm}

The graphs of \(y=\sin{x}\degree\) and \(y=\cos{x}\degree\), including their turning points, should already be familiar:

\begin{multicols}{2}
    \begin{center}
        \begin{tikzpicture}
            \draw[-stealth] (0,0) -- (7,0) node[below] {$x$};
            \draw[-stealth] (0,-1.4) -- (0,1.4) node[left] {$y$};
            \draw[smooth,domain=0:2*3.141592] plot (\x,{sin(\x r)});
            \node[above] at (2*3.141592,0) {$360$};
            \node[left] at (0,1) {$1$};
            \node[left] at (0,0) {$0$};
            \node[left] at (0,-1) {$-1$};
            \node at (3.3,1) {$y=\sin{x}\degree$};
            \foreach \x in {0.5,1,1.5,2}{
            \draw (3.141592*\x,0) -- (3.141592*\x,-0.1);
}
        \end{tikzpicture}
    \end{center}
    \begin{center}
        \begin{tikzpicture}
            \draw[-stealth] (0,0) -- (7,0) node[below] {$x$};
            \draw[-stealth] (0,-1.4) -- (0,1.4) node[left] {$y$};
            \draw[smooth,domain=0:2*3.141592] plot (\x,{cos(\x r)});
            \node[below] at (2*3.141592,0) {$360$};
            \node[left] at (0,1) {$1$};
            \node[left] at (0,0) {$0$};
            \node[left] at (0,-1) {$-1$};
            \node at (3.3,1) {$y=\cos{x}\degree$};
            \draw[dashed] (2*3.141592,1) -- (2*3.141592,0);
            \foreach \x in {0.5,1,1.5,2}{
            \draw (3.141592*\x,0) -- (3.141592*\x,-0.1);
}
        \end{tikzpicture}
    \end{center}
\end{multicols}

Graphs of the form \(k\sin{(x\pm a)}\) and \(k\cos{(x\pm a)}\) have maximum/minimum values of \(\pm k\), and their \emph{horizontal translation} is described by \(\pm a\) following the rules covered in the Graph Transformations chapter.

\begin{tcolorbox}[title=Example,colback=Mulberry!1!, colframe=Mulberry]

Given that $3\sin{x}\degree+4\cos{x}\degree$ can be expressed as $5\cos{(x-36.9)}\degree$, state the minimum value of $f(x)=3\sin{x}\degree+4\cos{x}\degree$ and the value of $x$ at which it occurs for $0<x<360$.

\tcblower

\textcolor{Mulberry}{Sketch the graph of $y=5\cos{x}\degree$ translated $36.9\degree$ to the right:}

\begin{center}
        \begin{tikzpicture}
            \draw[-stealth] (0,0) -- (7,0) node[below] {$x$};
            \draw[-stealth] (0,-1.6) -- (0,2) node[left] {$y$};
            \draw[thin,black!50,smooth,domain=0:2*3.141592] plot (\x,{0.3*5*cos(\x r)});
            \draw[thick,Mulberry,smooth,domain=0:2*3.141592] plot (\x,{0.3*3*sin(\x r)+0.3*4*cos(\x r)});
            \draw[thick,dashed,Mulberry,smooth,domain=2*3.141592:6.9272] plot (\x,{0.3*3*sin(\x r)+0.3*4*cos(\x r)});
            \node[black!50] at (2*3.141592,1.8) {$y=5\cos{x}\degree$};
            \node[right,Mulberry] at (6.9272,1.5) {$y=5\cos{(x-36.9)}\degree$};
            \node[below] at (2*3.141592,0) {$360$};
            \node[below] at (3.141592,0) {$180$};
            \node[left] at (0,1.5) {$5$};
            \node[left] at (0,0) {$0$};
            \node[left] at (0,-1.5) {$-5$};
            \node at (3.3,1) {$y=\cos{x}\degree$};
            \draw[dashed,black!50] (2*3.141592,1) -- (2*3.141592,0);
            \foreach \x in {0.5,1,1.5,2}{
            \draw (3.141592*\x,0) -- (3.141592*\x,-0.1);
}
            \draw[stealth-,Mulberry] (3.8859,-1.6) -- (4.1859,-1.9) node[right] {$(216.9\degree,-5)$};
            \draw[stealth-,black!50] (3.041592,-1.6) -- (2.741592,-1.9) node[left] {$(180\degree,-5)$};
            \draw[Mulberry,fill=Mulberry] (3.7859,-1.5) circle (0.05);
            \draw[black!50,fill=black!50] (3.141592,-1.5) circle (0.05);
        \end{tikzpicture}
    \end{center}

\textcolor{Mulberry}{Use the graph to state the answer:}

\vspace{0.1cm}

Hence the minimum value of $f(x)$ is $-5$, which occurs when $x=216.9$.

\end{tcolorbox}

\vspace{-0.5cm}

\subsubsection*{Exercise 14.3}\label{exercise-14.3}
\addcontentsline{toc}{subsubsection}{Exercise 14.3}

\vspace{-0.3cm}

\begin{enumerate}
    \item Sketch each for $0<x<360$, showing the coordinates of any roots and turning points:
    \vspace{-0.2cm}
    \begin{enumerate}
        \begin{multicols}{3}
            \item $7\cos{(x-10)}\degree$
            \item $3\sin{(x-20)}\degree$
            \item $\sqrt{5}\cos{(x+40)}\degree$
        \end{multicols}
    \end{enumerate}
    \item Find the maximum value of each and the value(s) of $x$ for which they occur for $0<x<360$:
    \vspace{-0.2cm}
    \begin{enumerate}
        \begin{multicols}{3}
            \item $8\sin{(x-50)}\degree$
            \item $\sqrt{2}\cos{(x+27)}\degree$
            \item $7\sin{(x+42.3)}\degree$
        \end{multicols}
    \end{enumerate}
    \item Find the minimum value of each and the value(s) of $x$ for which they occur for $0<x<2\pi$:
    \vspace{-0.2cm}
    \begin{enumerate}
        \begin{multicols}{3}
            \item $3\cos{\left(x-\dfrac{\pi}{6}\right)}$
            \item $5\sin{\left(x+\dfrac{\pi}{3}\right)}+2$
            \item $-12\sin{\left(x+\dfrac{\pi}{4}\right)}$
        \end{multicols}
    \end{enumerate}
    \item
    \begin{enumerate}
        \item Express $6\sin{x}\degree-7\cos{x}\degree$ in the form $k\sin{(x-a)}\degree$ where $k>0$ and $0<a<360$. \faCalculator
        \item Hence state the coordinates of the turning points of $12\sin{x}\degree-14\cos{x}\degree$ for $0<x<360$.
    \end{enumerate}
\end{enumerate}

\pagebreak

\section{Solving Equations using the Wave Function}\label{solving-equations-using-the-wave-function}

\vspace{-0.5cm}

Solving an equation like \(3\sin{x}\degree+4\cos{x}\degree=2\) can be approached by using the wave function to rewrite it as \(5\cos{(x-36.9)}\degree=2\), before solving it in the manner covered in Chapter 6.

\begin{tcolorbox}[title=Example,colback=Mulberry!1!, colframe=Mulberry]

It can be shown that $5\sin{x}\degree-2\cos{x}\degree$ can be expressed as $\sqrt{29}\sin{(x-21.8)}\degree$.

Hence, solve the equation $5\sin{x}\degree-2\cos{x}\degree=4$ where $0<x<360$.

\tcblower

\vspace{5.6cm}
\hspace{7cm}
\begin{tikzpicture}[remember picture, overlay, scale=0.5]
        \draw[white] (-8,0) -- (8,0);
        \draw (-1.5,0) -- (1.5,0);
        \draw (0,-1) -- (0,1);
        \node at (0.7,0.5) {A};
        \node at (-0.7,0.5) {S};
        \node at (-0.7,-0.5) {T};
        \node at (0.7,-0.5) {C};
        \node[Mulberry,above right] at (0.7,0.65) {\scriptsize{$a\degree$}};
        \node[Mulberry,above left] at (-0.7,0.65) {\scriptsize{$180\degree-a\degree$}};
        \node[Mulberry,right] at (0.7,0.5) {\ding{51}};
        \node[Mulberry,left] at (-0.7,0.5) {\ding{51}};
\end{tikzpicture}

\vspace{-6.6cm}
\begin{align*}
    5\sin{x}\degree-2\cos{x}\degree&=4&&\\[0.5em]
    \sqrt{29}\sin{(x-21.8)}\degree&=4&&\color{Mulberry}\longleftarrow \text{Substitute the wave function form}\\[0.5em]
    \sin{(x-21.8)}\degree&=\frac{4}{\sqrt{29}}&&\color{Mulberry}\longleftarrow \text{Rearrange to }\sin{(\dots)}=\dots\\[0.5em]
    \color{Mulberry}a&\color{Mulberry}=\sin^{-1}{\left(\frac{4}{\sqrt{29}}\right)=48.0\degree}&&\color{Mulberry}\longleftarrow \text{Calculate acute angle}\\[2.5em]
    &&&\color{Mulberry}\longleftarrow \text{Positive }\sin{a}\degree\text{: quadrants A and S}\\[2.5em]
    x-21.8\degree&=48.0\degree,180\degree-48.0\degree&&\color{Mulberry}\longleftarrow \text{Apply ticked quadrants}\\[0.5em]
    x-21.8\degree&=48.0\degree,132.0\degree&&\\[0.5em]
    x\degree&=69.8\degree,153.8\degree&&\color{Mulberry}\longleftarrow \text{Add }21.8\degree\text{ to both sides}
\end{align*}

\end{tcolorbox}

Note that any solutions outwith the domain (often \(0<x<360\)) should have \(360\degree\) added or subtracted to bring it back within the domain, where possible.

\vspace{-0.5cm}

\subsubsection*{Exercise 14.4}\label{exercise-14.4}
\addcontentsline{toc}{subsubsection}{Exercise 14.4}

\vspace{-0.3cm}

\begin{enumerate}
    \item Solve each equation for $0<x<360$: \faCalculator
    \begin{enumerate}
        \begin{multicols}{3}
            \item $7\sin{(x-18)}\degree=4$
            \item $3\cos{(x+34.1)}\degree=-2$
            \item $\sqrt{5}\sin{(x-106)}\degree+1=0$
        \end{multicols}
    \end{enumerate}
    \item Given $6\sin{x}\degree-8\cos{x}\degree=10\sin{(x-53.1)}\degree$, solve $6\sin{x}\degree-8\cos{x}\degree=5$ where $0<x<360$.
    \item 
    \begin{enumerate}
        \item Express $\sqrt{3}\sin{x}\degree+\cos{x}\degree$ in the form $k\sin{(x-a)}\degree$ where $k>0$ and $0<a<360$. 
        \item Hence solve the equation $\sqrt{3}\sin{x}\degree+\cos{x}\degree=1$ where $0<x<360$.
    \end{enumerate}
    \item Solve each equation for $0<x<2\pi$: \faCalculator
    \begin{enumerate}
        \begin{multicols}{3}
            \item $4\sin{(x+0.31)}+2=1$
            \item $9\cos{(x+1.24)}=5$
            \item $2\sqrt{3}\sin{(x-0.82)}=\sqrt{5}$
        \end{multicols}
    \end{enumerate}
    \item 
    \begin{enumerate}
        \item Express $3\cos{x}+2\c{sinx}$ in the form $k\cos{(x+a)}$ where $k>0$ and $0<a<2\pi$. \faCalculator
        \item Hence solve the equation $2+6\sin{x}+4\cos{x}=5$ where $0<x<2\pi$. \faCalculator
    \end{enumerate}
\end{enumerate}

\pagebreak

\section{Multiple Angles and Different Variables}\label{multiple-angles-and-different-variables}

\vspace{-0.5cm}

The techniques covered in this chapter can be applied to trigonometric expressions beyond those containing only \(\sin{x}\) and \(\cos{x}\); they work for any sum or difference of \emph{equal-angled} trigonometric operations.

\vspace{-0.2cm}

\[\text{e.g. }\qquad 3\sin{t}\degree+4\cos{t}\degree \qquad \text{ or } \qquad 2\sin{2x}-5\cos{2x}\]

\begin{tcolorbox}[title=Example,colback=Mulberry!1!, colframe=Mulberry]

Express $5\cos{2t}\degree-3\sin{2t}\degree$ in the form $k\sin{(2t+a)}\degree$ where $k>0$ and $0<a<360$.

\tcblower

\vspace{10cm}
\hspace{4cm}
\begin{tikzpicture}[remember picture, overlay, scale=0.5]
        \draw[white] (-8,0) -- (8,0);
        \draw (-1.5,0) -- (1.5,0);
        \draw (0,-1) -- (0,1);
        \node at (0.7,0.5) {A};
        \node at (-0.7,0.5) {S};
        \node at (-0.7,-0.5) {T};
        \node at (0.7,-0.5) {C};
        \node[Mulberry,above left] at (-0.7,0.65) {\scriptsize{$180\degree-\text{acute}\degree$}};
        \node[Mulberry,left] at (-0.7,0.5) {\ding{51}\ding{51}};
        \node[Mulberry,right] at (0.7,0.5) {\ding{51}};
        \node[Mulberry,left] at (-0.7,-0.5) {\ding{51}};
\end{tikzpicture}

\vspace{-4cm}
\hspace{8.3cm}
\begin{tikzpicture}[remember picture, overlay]
        \draw[thick,Mulberry,-stealth] (160:8) arc (160:200:8);
\end{tikzpicture}

\vspace{-7.6cm}
\begin{align*}
    5\cos{2t}\degree-3\sin{2t}\degree&=k\sin{(2t+a)}\degree&&\\[0.5em]
    5\doubleunderline{\cos{2t}\degree}-3\underline{\sin{2t}\degree}&=k\underline{\sin{2t}\degree}\cos{a}\degree+k\doubleunderline{\cos{2t}\degree}\sin{a}\degree&&\color{Mulberry}\longleftarrow \text{Expand }k\sin{(2t+a)}\degree\\[1.5em]
    k\sin{a}\degree&=5&&\color{Mulberry}\longleftarrow \text{Equate }\sin{2t}\degree\text{ coefficients}\\[0.5em]
    k\cos{a}\degree&=-3&&\color{Mulberry}\longleftarrow \text{Equate }\cos{2t}\degree\text{ coefficients}\\[1.5em]
    k&=\sqrt{(5)^2+(-3)^2}&&\color{Mulberry}\longleftarrow \text{Calculate }k\\[0.5em]
    &=\sqrt{34}\\[1.5em]
    \tan{a}\degree&=\frac{5}{-3}=-\frac{5}{3}&&\color{Mulberry}\longleftarrow \text{Find }\tan{a}\degree\\[0.5em]
    \text{acute}\degree&=\tan^{-1}{\left(\frac{5}{3}\right)}=59.0\degree &&\color{Mulberry}\longleftarrow \text{Find acute angle first}\\[0.8em]
    &&&\color{Mulberry}\longleftarrow \text{Positive }\sin{a}\degree\text{: quadrants A and S}\\[0.5em]
    &&&\color{Mulberry}\longleftarrow \text{Negative }\cos{a}\degree\text{: quadrants S and T}\\[1.5em]
    a\degree &=121.0\degree&&\color{Mulberry}\longleftarrow \text{Using }180\degree-59.0\degree\text{ (Double-ticked)}\\[1em]
    5\cos{2t}\degree-3\sin{2t}\degree&=\sqrt{34}\sin{(2t+121.0)}\degree&&\color{Mulberry}\longleftarrow \text{State solution}
\end{align*}

\end{tcolorbox}

\vspace{-0.5cm}

\subsubsection*{Exercise 14.5}\label{exercise-14.5}
\addcontentsline{toc}{subsubsection}{Exercise 14.5}

\vspace{-0.3cm}

\begin{enumerate}
    \item Express $4\cos{t}\degree-3\sin{t}\degree$ in the form $k\sin{(t-a)}\degree$, where $k>0$ and $0<a<360$. \faCalculator
    \item Express $2\sin{2x}-\cos{2x}$ in the form $k\cos{(2x-a)}$ where $k>0$ and $0<a<2\pi$. \faCalculator
    \item 
    \begin{enumerate}
        \item Express $12\cos{t}\degree+5\sin{t}\degree$ in the form $k\sin{(t+a)}\degree$, where $k>0$ and $0<a<360$. \faCalculator
        \item Hence state:
        \begin{enumerate}
            \item The maximum value of the function $f(x)=12\cos{t}\degree+5\sin{t}\degree$, $0<t<360$.
            \item The value(s) of $t$ for which it occurs.
        \end{enumerate}
    \end{enumerate}
    \item 
    \begin{enumerate}
        \item Express $\sin{2x}-\sqrt{3}\cos{2x}$ in the form $k\cos{(2x-a)}$ where $k>0$, $0<a<2\pi$.
        \item Hence solve $\sin{2x}-\sqrt{3}\cos{2x}-1=0$, $0<x<2\pi$.
        \item Sketch $y=-\sin{2x}+\sqrt{3}\cos{2x}-1=0$ for $0\leqslant x \leqslant 2\pi$.
    \end{enumerate}
\end{enumerate}

\pagebreak

\section*{Wave Function Review Exercise}\label{wave-function-review-exercise}
\addcontentsline{toc}{section}{Wave Function Review Exercise}

\begin{enumerate}
    \item Express $8\sin{x}\degree+7\cos{x}\degree$ in the form $k\sin{(x-a)}\degree$ where $k>0$ and $0<a<360$. \faCalculator\\[0.5em]
    \item Express $\sqrt{5}\sin{x}+\cos{x}$ in the form $k\cos{(x-a)}$ where $k>0$ and $0<a<2\pi$. \faCalculator\\[0.5em]
    \item Express $\sqrt{3}\cos{t}\degree-\sin{t}\degree$ in the form $k\sin{(t+a)}\degree$ where $k>0$ and $0<a<360$.\\[0.5em]
    \item Part of the graphs of $y=3\cos{x}\degree-5\sin{x}\degree$ and $y=-4$ are shown in the diagram below: \faCalculator
    \begin{center}
        \begin{tikzpicture}[scale=1.5]
            \draw[-stealth] (-0.5,0) -- (7.5,0) node[below] {$x$};
            \draw[-stealth] (0,-1.5) -- (0,1.5) node[left] {$y$};
            \node[below left] {O};
            \draw[thick,smooth,domain=0:2*3.141592] plot (\x,{0.6*cos(\x r)-1.0*sin(\x r)}) node[right] {$y=3\cos{x}\degree-5\sin{x}\degree$};
            \draw[dashed] (2*3.141592,0.6) -- (2*3.141592,0) node[below] {$360\degree$};
            \draw[thick] (0,-0.8) -- (2*3.141592,-0.8) node[right] {$y=-4$};
            \draw[fill] (1.29639,-0.8) circle (0.04) node[below left] {P};
            \draw[fill] (2.92604,-0.8) circle (0.04) node[below right] {Q};
        \end{tikzpicture}
    \end{center}
    Points P and Q are points of intersection.\\
    \begin{enumerate}
        \item Express $y=3\cos{x}\degree-5\sin{x}\degree$ in the form $k\cos{(x+a)}\degree$ where $k>0$ and $0<a<360$.\\[0.1em]
        \item Hence determine the coordinates of P and Q.\\[0.5em]
    \end{enumerate}
    \item 
    \begin{enumerate}
        \item Express $2\sin{x}\degree-4\cos{x}\degree$ in the form $k\sin{(x-a)}\degree$ where $k>0$ and $0<x<360$. \faCalculator\\[0.1em]
        \item Hence sketch the graph of $y=2\sin{x}\degree-4\cos{x}\degree$ for $0<x<360$.\\[0.5em]
    \end{enumerate}
    \item 
    \begin{enumerate}
        \item Express $\cos{x}+\sqrt{3}\sin{x}$ in the form $k\cos{(x-a)}$ where $k>0$ and $0<a<2\pi$.\\[0.1em]
        \item Hence sketch the graph of $y=2\cos{x}+2\sqrt{3}\sin{x}$ for $0<x<2\pi$.\\[0.5em]
    \end{enumerate}
    \item 
    \begin{enumerate}
        \item Express $6\cos{t}-3\sin{t}$ in the form $k\cos{(t+a)}$ where $k>0$ and $0<a<2\pi$. \faCalculator\\[0.1em]
        \item Hence solve $2\cos{t}-\sin{t}+2=1$ where $0<t<2\pi$. \faCalculator\\[0.5em]
    \end{enumerate}
    \item 
    \begin{enumerate}
        \item Express $\sin{2x}\degree-\cos{2x}\degree$ in the form $k\sin{(2x-a)}$ where $k>0$ and $0<a<360$.\\[0.1em]
        \item Hence solve the equation $\sin{2x}\degree=\cos{2x}\degree$ where $0<x<360$.
    \end{enumerate}
\end{enumerate}

\pagebreak

\chapterfont{\color{white}}

\chapter{Logs and Exponentials}\label{logs-and-exponentials}

\vspace{-12cm}
\begin{center}
\begin{tikzpicture}
\draw[white] (10,5) circle (0.01);
\draw[PineGreen,rounded corners,very thick] (1,0.2) rectangle (5,1.8);
\draw[PineGreen,very thick] (5,1) -- (6,1);
\draw[PineGreen,fill=PineGreen,rounded corners] (6,1.8) rectangle (18.2,0.2);
\node[white] at (12.1,1) {\Huge{\textsc{Logs and Exponentials}}};
\node at (3,1) {\Large{\textsc{Chapter 15}}};
\end{tikzpicture}
\end{center}

Each part of an operation of the form \(a^n\) can be described using the following terminology:

\begin{center}
    \begin{tikzpicture}
        \node at (0,0) {\large{$a^n$}};
        \draw[stealth-,PineGreen,thick] (0.3,0.2) --++(1,0.5) node[right] {"power" or "exponent" or "index"};
        \draw[stealth-,PineGreen,thick] (-0.32,0) --++(-1,0.2) node[left] {"base"};
    \end{tikzpicture}
\end{center}

\emph{Power functions} take the form \(f(x)=x^n\), with \(x\) as the \textcolor{PineGreen}{\textit{base}} and a constant power, \(n\).

\[\text{e.g.}\quad f(x)=x^3\]

\emph{Exponential functions} take the form \(f(x)=a^x\), with \(x\) as the \textcolor{PineGreen}{\textit{exponent}} and a constant base, \(a>0,a\ne 1\).

\[\text{e.g.}\quad f(x)=3^x\]

An example of an exponential function in everyday life is that of something \emph{appreciating} by a percentage of its value, such as an antique vase of value £4000 increasing by 20\% each year. Its value after \(0\) years, \(1\) years, \(2\) years, and so on, can be calculated as follows:

\begin{align*}
    \color{black!25}4000\times 1.2^0\;=\;&4000 && \color{PineGreen}\longleftarrow\text{After 0 years}\\[0.5em]
    4000\times 1.2^{\color{PineGreen}1}\color{black}\;=\;&4800 && \color{PineGreen}\longleftarrow\text{After 1 year}\\[0.5em]
    4000\times 1.2^{\color{PineGreen}2}\color{black}\;=\;&5760 && \color{PineGreen}\longleftarrow\text{After 2 years}\\[0.5em]
    4000\times 1.2^{\color{PineGreen}3}\color{black}\;=\;&6912 && \color{PineGreen}\longleftarrow\text{After 3 years}\\[0.5em]
    4000\times 1.2^{\color{PineGreen}4}\color{black}\;=\;&8294.40 && \color{PineGreen}\longleftarrow\text{After 4 years}
\end{align*}

The function to describe its value \(V\) after \(\color{PineGreen}x\) years is given by: \(V(\color{PineGreen}x\color{black})=4000\times1.2^{\color{PineGreen}x}\)

One advantage of defining this function is the ability to calculate the value at times other that after whole years. For example, the value after \textcolor{PineGreen}{three and a half years} can be calculated as:

\[V(\color{PineGreen}3.5\color{black})=4000\times1.2^{\color{PineGreen}3.5}=7571.72\]

This chapter will introduce a range of skills required when working with exponential functions.

\pagebreak

\section{Graphs of Exponential Functions}\label{graphs-of-exponential-functions}

\vspace{-0.5cm}

\setlength{\columnsep}{30pt}

\begin{multicols}{2}

Where $a>1$:

\vspace{-0.2cm}

$y=a^x$ is \textit{strictly increasing} on $x\in\mathbb{R}$:

\vspace{-0.5cm}
\begin{center}
    \begin{tikzpicture}[scale=0.8]
        \draw[-stealth] (-5,0) -- (5,0) node[below] {$x$};
        \draw[-stealth] (0,-1) -- (0,4.5) node[left] {$y$};
        \node[below left] {O};
        \draw[smooth,thick,PineGreen,domain=-5:5] plot (\x,{1.3^(\x)}) node[above left] {$y=a^x$};
        \draw[PineGreen,fill=PineGreen] (0,1) circle (0.05) node[above left] {$1$};
        \draw[PineGreen,fill=PineGreen] (1,1.3) circle (0.05) node[above] {$(1,a)$};
    \end{tikzpicture}
\end{center}

\vspace{-0.5cm}
This describes \textbf{exponential growth}.

\columnbreak

Where $0<a<1$:

\vspace{-0.2cm}

$y=a^x$ is \textit{strictly decreasing} on $x\in\mathbb{R}$:

\vspace{-0.5cm}
\begin{center}
    \begin{tikzpicture}[scale=0.8]
        \draw[-stealth] (-5,0) -- (5,0) node[below] {$x$};
        \draw[-stealth] (0,-1) -- (0,4.5) node[left] {$y$};
        \node[below left] {O};
        \draw[smooth,thick,PineGreen,domain=-5:5] plot (\x,{0.76923^(\x)}) node[above] {$y=a^x$};
        \draw[PineGreen,fill=PineGreen] (0,1) circle (0.05) node[above left] {$1$};
        \draw[PineGreen,fill=PineGreen] (1,0.76923) circle (0.05) node[above] {$(1,a)$};
    \end{tikzpicture}
\end{center}

\vspace{-0.5cm}
This describes \textbf{exponential decay}.

\end{multicols}

\vspace{-0.5cm}

\setlength{\columnsep}{10pt}

Since \(a^0=1\), any graph of the form \(y=a^x\) will pass through the point \((0,1)\) for all \(a\ne 0\).

Since \(a^1=a\), any graph of the form \(y=a^x\) will pass through the point \((1,a)\) for all \(a\).

The graph of \(y=a^x\) has the \(x\)-axis as a \emph{horizontal asymptote} - a line that it approaches but never meets.

Determining the equation of the graph of an exponential function can typically be achieved using substitution or consideration of graph transformations, along with knowledge of points \((0,1)\) and \((1,a)\).

\begin{tcolorbox}[title=Example,colback=PineGreen!2!, colframe=PineGreen]

The graph of $y=a^x+b$ is below. Find the values of $a$ and $b$, and state the range of $f(x)=a^x+b$.

\vspace{-0.5cm}
\begin{center}
    \begin{tikzpicture}[scale=0.9]
        \draw[-stealth] (-5,0) -- (5,0) node[below] {$x$};
        \draw[-stealth] (0,-0.3) -- (0,4) node[left] {$y$};
        \node[below left,xshift=0.2,yshift=0.2] {\small{O}};
        \draw[smooth,thick,domain=-3:4,yscale=0.2] plot (\x,{2^(\x)+1}) node[above] {$y=a^x+b$};
        \draw[fill] (0,0.4) circle (0.05) node[above left] {$2$};
        \draw[fill] (2,1) circle (0.05) node[above left] {$(2,5)$};
        \draw[dashed] (-4,0.17) -- (4,0.17);
    \end{tikzpicture}
\end{center}

\vspace{-0.4cm}

\tcblower

\vspace{-0.6cm}
\begin{align*}
    2&=a^0+b && \color{PineGreen}\longleftarrow\text{Substitute }(0,2)\\[0.5em]
    2&=1+b &&\\[0.5em]
    1&=b && \color{PineGreen}\longleftarrow\text{Solve to obtain }b\\[1em]
    5&=a^2+1 && \color{PineGreen}\longleftarrow\text{Substitute }(2,5)\text{ and }b=1\\[0.5em]
    4&=a^2 &&\\[0.5em]
    2&=a && \color{PineGreen}\longleftarrow\text{Solve to obtain }a\\[1em]
    y&=2^x+1&&\color{PineGreen}\longleftarrow\text{State equation}\\[0.5em]
    f(x)&\geqslant 1&&\color{PineGreen}\longleftarrow\text{State range}
\end{align*}

\end{tcolorbox}

\pagebreak

\subsection*{Exercise 15.1}\label{exercise-15.1}
\addcontentsline{toc}{subsection}{Exercise 15.1}

\vspace{-0.5cm}

\begin{enumerate}
    \item Find the equation of each exponential graph using the form given, then determine the value of $k$.
    \begin{enumerate}
    \vspace{-0.2cm}
        \begin{multicols}{2}
            \item[(a)] $y=a^x$\\
            \begin{tikzpicture}[scale=0.8]
            \draw[-stealth] (-4,0) -- (4,0) node[below] {$x$};
            \draw[-stealth] (0,-1.2) -- (0,4) node[left] {$y$};
            \draw[smooth,thick,domain=-2:2] plot (\x,{0.4*3^(\x)});
            \draw[fill] (0,0.4) circle (0.07) node[above left] {$1$};
            \draw[fill] (1,1.2) circle (0.07) node[right,xshift=0.1cm] {$(1,3)$};
            \draw[fill,PineGreen] (2,3.6) circle (0.07) node[below right,xshift=-0.1cm,yshift=0.1cm] {$(2,k)$};
            \node[below left] {\scriptsize{O}};
            \end{tikzpicture}
            \item[(c)] $y=a^x$\\
            \begin{tikzpicture}[scale=0.8]
            \draw[-stealth] (-4,0) -- (4,0) node[below] {$x$};
            \draw[-stealth] (0,-1.2) -- (0,4) node[left] {$y$};
            \draw[smooth,thick,domain=-2.5:3.3,] plot (\x,{0.4*4^(0.5*\x)});
            \draw[fill] (0,0.4) circle (0.07) node[above left] {$1$};
            \draw[fill] (2,1.6) circle (0.07) node[below right] {$(2,4)$};
            \draw[fill,PineGreen] (3,3.2) circle (0.07) node[below right] {$(3,k)$};
            \node[below left] {\scriptsize{O}};
            \end{tikzpicture}
            \item[(e)] $y=a^x+2$\\
            \begin{tikzpicture}[scale=0.8]
            \draw[-stealth] (-4,0) -- (4,0) node[below] {$x$};
            \draw[-stealth] (0,-1.2) -- (0,4) node[left] {$y$};
            \draw[smooth,thick,domain=-3:2.5,yscale=0.5] plot (\x,{2^(\x)+2});
            \draw[fill] (0,1.5) circle (0.07) node[above left] {$3$};
            \draw[dashed,thin,PineGreen] (-4,1) -- (2.5,1) node[right] {$y=k$};
            \draw[fill] (2,3) circle (0.07) node[below right] {$(2,6)$};
            \node[below left] {\scriptsize{O}};
            \end{tikzpicture}
            \item[(g)] $y=a^x+b$\\
            \begin{tikzpicture}[scale=0.8]
            \draw[-stealth] (-4,0) -- (4,0) node[below] {$x$};
            \draw[-stealth] (0,-1.5) -- (0,3.7) node[left] {$y$};
            \draw[smooth,thick,domain=-3:1.3] plot (\x,{3^(\x)-1});
            \draw[fill] (0,0) circle (0.07);
            \draw[fill] (1,2) circle (0.07) node[below right] {$(1,2)$};
            \node[above left] {\scriptsize{O}};
            \draw[dashed,thin] (-4,-1) -- (2.5,-1) node[right] {$y=-1$};
            \draw[fill,PineGreen] (0.5,0.732) circle (0.07) node[below right] {$(0.5,k)$};
            \end{tikzpicture}
            \item[(b)] $y=a^x$\\
            \begin{tikzpicture}[scale=0.8]
            \draw[-stealth] (-4,0) -- (4,0) node[below] {$x$};
            \draw[-stealth] (0,-1.2) -- (0,4) node[left] {$y$};
            \draw[smooth,thick,domain=-3.5:3.5] plot (\x,{0.75^(\x)});
            \draw[fill,PineGreen] (0,1) circle (0.07) node[below left] {$k$};
            \draw[fill] (2.40942,0.5) circle (0.07) node[above right] {$(1,0.5)$};
            \node[below left] {\scriptsize{O}};
            \end{tikzpicture}
            \item[(d)] $y=4^x+b$\\
            \begin{tikzpicture}[scale=0.8]
            \draw[-stealth] (-4,0) -- (4,0) node[below] {$x$};
            \draw[-stealth] (0,-1.5) -- (0,3.7) node[left] {$y$};
            \draw[smooth,thick,domain=-2.8:2.1,yscale=0.2] plot (\x,{4^(\x)-5});
            \draw[fill] (2,2.2) circle (0.07) node[right] {$(2,11)$};
            \draw[dashed,thin,PineGreen] (-4,-1.1) -- (2.5,-1.1) node[right] {$y=k$};
            \node[below left] {\scriptsize{O}};
            \end{tikzpicture}
            \item[(f)] $y=a^x+b$\\
            \begin{tikzpicture}[scale=0.8]
            \draw[-stealth] (-4,0) -- (4,0) node[below] {$x$};
            \draw[-stealth] (0,-1.2) -- (0,4) node[left] {$y$};
            \draw[smooth,thick,domain=-3.5:2.5] plot (\x,{1.5^(\x)+1});
            \draw[fill] (0,2) circle (0.07) node[above left] {$2$};
            \draw[fill] (1,2.5) circle (0.07) node[below right] {$(1,2.5)$};
            \draw[dashed,thin] (-4,1) -- (4,1);
            \draw[fill,PineGreen] (2,3.25) circle (0.07) node[below right] {$(2,k)$};
            \node[below left] {\scriptsize{O}};
            \end{tikzpicture}
            \item[(h)] $y=a^x+b$\\
            \begin{tikzpicture}[scale=0.8]
            \draw[-stealth] (-4,0) -- (4,0) node[below] {$x$};
            \draw[-stealth] (0,-2) -- (0,3.2) node[left] {$y$};
            \draw[smooth,thick,domain=-3:1.4] plot (\x,{3^(\x)-2});
            \draw[fill] (0,-1) circle (0.07) node[left] {$-1$};
            \draw[fill] (1,1) circle (0.07) node[right] {$(1,1)$};
            \draw[dashed,thin] (-4,-2) -- (4,-2);
            \draw[fill,PineGreen] (-1,-1.666666666) circle (0.07) node[above left] {$(-1,k)$};
            \node[below left] {\scriptsize{O}};
            \end{tikzpicture}
        \end{multicols}
    \end{enumerate}
    \vspace{-0.5cm}
    \item Sketch the graph of each of the following, and state the range of $y=f(x)$:
    \vspace{-0.3cm}
    \begin{enumerate}
        \begin{multicols}{4}
            \item $y=3^x$
            \item $y=3^x+1$
            \item $y=3^x-2$
            \item $y=3^{x+2}$
        \end{multicols}
        \begin{multicols}{4}
            \item $y=0.5^x$
            \item $y=0.5^x-1$
            \item $y=0.5^{x-3}$
            \item $y=0.5^{1-x}$
        \end{multicols}
    \end{enumerate}
\end{enumerate}

\pagebreak

\section{Evaluating Logarithms}\label{evaluating-logarithms}

\vspace{-0.5cm}

The function \(f(x)=\color{PineGreen}\log_a{\color{black}x}\), where \(a>0\) and \(x>0\), is defined as the inverse function to \(f(x)=\color{PineGreen}a^{\color{black}x}\).

\begin{center}
\begin{tcolorbox}[colback=red!5,width=10cm,colframe=red!70!black,height=2.2cm]
\vspace{-0.5cm}

\begin{align*}
    \text{If }\hspace{1.9cm} x&=a^y\\[0.5em]
    \text{ then }\hspace{0.5cm} \log_a{x}&=y \qquad (\text{where }a>0,x>0)
\end{align*}

\end{tcolorbox}
\end{center}

\vspace{-0.2cm}

A numerical example will help build an understanding of how to interpret a logarithm:

\vspace{-0.8cm}

\begin{align*}
    \text{Since }\hspace{1.2cm} 8&=\color{PineGreen}2^{\color{black}3}\\[0.5em]
    \text{ then }\hspace{0.5cm}\color{PineGreen}\log_2\color{black}{8}&=3
\end{align*}

\vspace{-0.3cm}

Hence \(\log_2{8}\) (``log to the base 2 of 8'') can be read as: ``2 raised to \emph{which power} gives a value of 8?''

\begin{multicols}{2}

\begin{tcolorbox}[title=Example,colback=PineGreen!2!, colframe=PineGreen,width=8.5cm,height=2.8cm]

Evaluate $\log_3{81}$.

\vspace{-0.1cm}
\tcblower

$\log_3{81}=4\color{PineGreen}\longleftarrow\text{Since }3^4=81$

\end{tcolorbox}



\begin{tcolorbox}[title=Example,colback=PineGreen!2!, colframe=PineGreen,width=8.5cm,height=2.8cm]

Evaluate $\log_7{49}$.

\vspace{-0.1cm}
\tcblower

$\log_7{49}=2\color{PineGreen}\longleftarrow\text{Since }7^2=49$

\end{tcolorbox}

\begin{tcolorbox}[title=Example,colback=PineGreen!2!, colframe=PineGreen,width=8.5cm,height=2.8cm]

Evaluate $\log_5{1}$.

\vspace{-0.1cm}
\tcblower

$\log_5{1}=0\color{PineGreen}\longleftarrow\text{Since }5^0=1$

\end{tcolorbox}

\begin{tcolorbox}[title=Example,colback=PineGreen!2!, colframe=PineGreen,width=8.5cm,height=2.8cm]

Evaluate $\log_3{\frac{1}{9}}$.

\vspace{-0.1cm}
\tcblower

$\log_3{\frac{1}{9}}=-2\color{PineGreen}\longleftarrow\text{Since }3^{-2}=\frac{1}{3^2}=\frac{1}{9}$

\end{tcolorbox}

\vspace{-0.4cm}

\end{multicols}

\vspace{-1.2cm}

\subsection*{Exercise 15.2}\label{exercise-15.2}
\addcontentsline{toc}{subsection}{Exercise 15.2}

\vspace{-0.6cm}

\begin{enumerate}
    \item State the value each of the following logarithms:
    \begin{enumerate}
        \begin{multicols}{5}
            \item $\log_6{36}$
            \item $\log_5{125}$
            \item $\log_4{16}$
            \item $\log_2{16}$
            \item $\log_9{81}$
        \end{multicols}
        \begin{multicols}{5}
            \item $\log_3{81}$
            \item $\log_2{32}$
            \item $\log_{10}{100}$
            \item $\log_8{64}$
            \item $\log_4{64}$
        \end{multicols}
        \begin{multicols}{5}
            \item $\log_7{7}$
            \item $\log_3{9}$
            \item $\log_8{1}$
            \item $\log_5{25}$
            \item $\log_5{5}$
        \end{multicols}
        \begin{multicols}{5}
            \item $\log_2{32}$
            \item $\log_{4}{4}$
            \item $\log_2{1}$
            \item $\log_6{216}$
            \item $\log_{2}{128}$
        \end{multicols}
    \end{enumerate}
    \item By considering negative and fractional indices, evaluate each:
    \begin{enumerate}
        \begin{multicols}{5}
            \item $\log_9{3}$
            \item $\log_{64}{8}$
            \item $\log_{25}{5}$
            \item $\log_100{10}$
            \item $\log_{8}{2}$
        \end{multicols}
        \begin{multicols}{5}
            \item $\log_{27}{3}$
            \item $\log_4{2}$
            \item $\log_{16}{2}$
            \item $\log_{81}{3}$
            \item $\log_{11}{121}$
        \end{multicols}
        \begin{multicols}{5}
            \item $\log_5{\frac{1}{5}}$
            \item $\log_3{\frac{1}{3}}$
            \item $\log_{6}{\frac{1}{36}}$
            \item $\log_2{\frac{1}{8}}$
            \item $\log_9{\frac{1}{3}}$
        \end{multicols}
    \end{enumerate}
    \item Calculate the value of each to three significant figures: \faCalculator
    \begin{enumerate}
        \begin{multicols}{5}
            \item $\log_2{31}$
            \item $\log_{7}{50}$
            \item $\log_{10}{99}$
            \item $\log_5{120}$
            \item $\log_{2}{0.9}$
        \end{multicols}
    \end{enumerate}
\end{enumerate}

\pagebreak

\section{Laws of Logs}\label{laws-of-logs}

\vspace{-0.5cm}

There are two key identities that can be obtained from the definition of a logarithm. For all \(a>0\):

\begin{multicols}{5}

\begin{center}
\begin{tcolorbox}[colback=white,width=3.5cm,colframe=white,height=1cm]

\end{tcolorbox}
\end{center}

\begin{center}
\begin{tcolorbox}[colback=red!5,width=3.5cm,colframe=red!70!black,height=1cm]

\begin{center}
$\log_a{1}=0$ 
\end{center}

\end{tcolorbox}
\end{center}

\begin{center}
\begin{tcolorbox}[colback=white,width=3.5cm,colframe=white,height=1cm]

\begin{center}
and
\end{center}

\end{tcolorbox}
\end{center}

\begin{center}
\begin{tcolorbox}[colback=red!5,width=3.5cm,colframe=red!70!black,height=1cm]

\begin{center}
$\log_a{a}=1$ 
\end{center}

\end{tcolorbox}
\end{center}

\begin{center}
\begin{tcolorbox}[colback=white,width=3.5cm,colframe=white,height=1cm]


\end{tcolorbox}
\end{center}

\end{multicols}

\vspace{-0.9cm}

There also three \emph{``log laws''} that can be applied for logarithms of the \emph{same base}. For all \(a,x,y>0\):

\vspace{-0.5cm}

\begin{multicols}{3}

\begin{center}
\begin{tcolorbox}[colback=red!5,width=5.8cm,colframe=red!70!black,height=2cm]

The \textbf{product law}:\\[-0.2cm]

$\log_a{(xy)}=\log_a{x}+\log_a{y}$ 

\end{tcolorbox}
\end{center}

\begin{center}
\begin{tcolorbox}[colback=red!5,width=5.8cm,colframe=red!70!black,height=2cm]

The \textbf{quotient law}:\\[-0.2cm]

$\log_a{\left(\frac{x}{y}\right)}=\log_a{x}-\log_a{y}$ 

\end{tcolorbox}
\end{center}

\begin{center}
\begin{tcolorbox}[colback=red!5,width=5.8cm,colframe=red!70!black,height=2cm]

The \textbf{power law}:\\[-0.2cm]

$\log_a{\left(x^y\right)}=y\log_a{x}$ 

\end{tcolorbox}
\end{center}

\end{multicols}

\vspace{-0.3cm}

\begin{tcolorbox}[title=Example,colback=PineGreen!2!, colframe=PineGreen,height=9.3cm]

Evaluate $\log_{6}{8}+2\log_{6}{3}-\log_{6}{2}$.

\tcblower

\vspace{-0.5cm}
\begin{align*}
    &\log_{6}{8}+2\log_{6}{3}-\log_{6}{2}&&\color{PineGreen}\longleftarrow\text{Consider order of operations}\\[0.5em]
    =&\log_{6}{8}+\log_{6}{\left(3^2\right)}-\log_{6}{2} && \color{PineGreen}\longleftarrow\text{Apply power law}\\[0.5em]
    =&\log_{6}{8}+\log_{6}{9}-\log_{6}{2} &&\\[0.5em]
    =&\log_{6}{\left(8\times 9\right)}-\log_{6}{2} && \color{PineGreen}\longleftarrow\text{Apply product law}\\[0.5em]
    =&\log_{6}{72}-\log_{6}{2} &&\\[0.3em]
    =&\log_{6}{\left(\frac{72}{2}\right)} && \color{PineGreen}\longleftarrow\text{Apply product law}\\[0.3em]
    =&\log_{6}{36} &&\\[0.5em]
    =&2 && \color{PineGreen}\longleftarrow\text{Evaluate}\\[0.5em]
\end{align*}

\vspace{-0.5cm}

\end{tcolorbox}

\vspace{-0.6cm}

\subsection*{Exercise 15.3}\label{exercise-15.3}
\addcontentsline{toc}{subsection}{Exercise 15.3}

\vspace{-0.4cm}

\begin{enumerate}
    \item Evaluate each of the following:
    \begin{enumerate}
        \begin{multicols}{3}
            \item $\log_6{9}+\log_6{4}$
            \item $\log_{12}{2}+\log_{12}{6}$
            \item $\log_{2}{3}+\log_2{\frac{1}{3}}$
        \end{multicols}
        \begin{multicols}{3}
            \item $\log_{3}{54}-\log_{3}{2}$
            \item $\log_{5}{10}-\log_{5}{2}$
            \item $\log_{5}{100}-\log_{5}{4}$
        \end{multicols}
        \begin{multicols}{3}
            \item $2\log_4{6}-\log_{4}{9}$
            \item $\log_{12}{4}+2\log_{12}{6}$
            \item $3\log_6{2}+\log_{6}{9}-\log_{6}{2}$
        \end{multicols}
        \begin{multicols}{3}
            \item $2\log_6{3}+\log_{6}{\frac{2}{3}}$
            \item $\log_5{4}+2\log_{5}{10}$
            \item $2\log_{2}{6}-\log_{2}{12}+\log_2{\frac{1}{6}}$
        \end{multicols}
    \end{enumerate}
    \item Simplify each, giving an answer in the form $\log_p{q}$ where $p$ and $q$ are positive integers:
    \begin{enumerate}
        \begin{multicols}{3}
            \item $\log_7{3}+\log_7{6}-\log_{7}{9}$
            \item $\frac{1}{2}\log_{6}{9}-\log_{6}{2}$
            \item $2\log_{8}{9}-3\log_2{3}$
        \end{multicols}
    \end{enumerate}
    \item Simplify each, giving an answer in the form $\log_a{p}$ where $p$ is a positive integer:
    \begin{enumerate}
        \begin{multicols}{3}
            \item $\log_a{3}+2\log_a{4}$
            \item $\frac{1}{3}\log_{a}{8}-\log_{a}{4}$
            \item $\log_{a}{20}-\log_a{4}+\log_a{\frac{1}{15}}$
        \end{multicols}
    \end{enumerate}
\end{enumerate}

\pagebreak

\section{Solving Log Equations}\label{solving-log-equations}

\vspace{-0.5cm}

To solve a \emph{log equation}, it is typically desirable to have each side within a logarithm of the \emph{same base}:

\vspace{-0.4cm}

\begin{center}
\begin{tcolorbox}[colback=red!5,width=11cm,colframe=red!70!black,height=2.2cm]
\vspace{-0.5cm}

\begin{align*}
    \text{If }\hspace{1.0cm} \log_a{x}&=\log_a{y}\\[0.5em]
    \text{ then }\hspace{1.4cm} x&=y \qquad \qquad (\text{where }a>0,x>0,y>0)
\end{align*}

\end{tcolorbox}
\end{center}

\vspace{-0.5cm}

``Cancelling the logs'', mathematically, is raising each side to an exponential base \(a\), using \(a^{\log_a{b}}=b\).

\begin{tcolorbox}[title=Example,colback=PineGreen!2!, colframe=PineGreen,height=5.5cm]

Solve the equation $\log_3{x}+\log_3{5}=\log_3{20}$.

\tcblower

\vspace{-0.5cm}

\begin{align*}
\log_3{x}+\log_3{5}&=\log_3{20} &&\\[0.5em]
\log_3{5x}&=\log_3{20} && \color{PineGreen}\longleftarrow\text{Apply product law}\\[0.5em]
5x&=20 && \color{PineGreen}\longleftarrow\text{``Cancel'' }\log_3\text{ from both sides}\\[0.5em]
x&=4 && \color{PineGreen}\longleftarrow\text{Solve}
\end{align*}

\end{tcolorbox}

To express any term as a logarithm, multiply it by \(\log_a{a}\) (which equals 1) and then apply the power law:

\begin{tcolorbox}[title=Example,colback=PineGreen!2!, colframe=PineGreen,height=6cm]

Solve the equation $\log_3{x}-\log_3{2}=2$.

\tcblower

\vspace{-0.5cm}

\begin{align*}
\log_3{x}-\log_3{2}&=2\color{PineGreen}\log_3{3} &&\color{PineGreen}\longleftarrow\text{Multiply the constant by}\log_33\\[0.5em]
\log_3{\left(\frac{x}{2}\right)}&=\log_3{3^2} && \color{PineGreen}\longleftarrow\text{Apply quotient and power laws}\\[0.5em]
\frac{x}{2}&=9 && \color{PineGreen}\longleftarrow\text{``Cancel'' }\log_3\text{ from both sides}\\[0.5em]
x&=18 && \color{PineGreen}\longleftarrow\text{Multiply both sides by }2\text{ to solve}
\end{align*}

\end{tcolorbox}

\vspace{-0.5cm}

\subsection*{Exercise 15.4}\label{exercise-15.4}
\addcontentsline{toc}{subsection}{Exercise 15.4}

\vspace{-0.4cm}

\begin{enumerate}
    \item Solve:
    \vspace{-0.2cm}
    \begin{enumerate} 
        \begin{multicols}{2}
            \item $\log_{5}{3}+\log_{5}{x}=\log_{5}{18}$
            \item $\log_{7}{2x}+\log_{7}{4}=\log_{7}{56}$
        \end{multicols}
        \begin{multicols}{2}    
            \item $\log_{6}{5}+\log_{5}{(x-1)}=\log_{6}{30}$
            \item $\log_{a}{x}-2\log_{a}{3}=\log_{a}{18}$
        \end{multicols}
        \begin{multicols}{2}
            \item $\log_{8}{2x}-\log_{8}{3}=\log_{8}{1}$
            \item $\log_{3}{8}-\log_{3}{x}=\log_{3}{2}$
        \end{multicols}
    \end{enumerate}
    \item Solve each:
    \vspace{-0.2cm}
    \begin{enumerate}
        \begin{multicols}{3}
            \item $\log_{6}{x}+\log_{6}{12}=2$
            \item $\log_{4}{x}+\log_{4}{8}=2$
            \item $\log_{2}{10}+\log_{2}{x}=4$
        \end{multicols}
        \begin{multicols}{3}
            \item $\log_{5}{x}-\log_{5}{2}=1$
            \item $\log_{3}{x}-2\log_{3}{2}=2$
            \item $\log_{4}{2}+3\log_{4}{x}=2$
        \end{multicols}
        \begin{multicols}{3}
            \item $\log_{2}{5}+3=\log_{2}{(x+2)}$
            \item $\log_{3}{(2x)}-1=\log_{3}{8}$
            \item $2-\log_{6}{(x-3)}=2\log_{6}{3}$
        \end{multicols}
    \end{enumerate}
\end{enumerate}

\pagebreak

Some algebraic manipulations used to solve an equation may produce solutions do not, in fact, satisfy the original equation. These \textbf{extraneous solutions} can arise when solving log equations.

Since \(\log_a{x}\) is only defined for \(a>0\) and \(x>0\), any solutions which contradict this must be discarded.

\begin{tcolorbox}[title=Example,colback=PineGreen!2!, colframe=PineGreen,height=8.6cm]

Solve the equation $\log_3{x}+\log_3{(x-2)}=\log_3{8}$.

\tcblower

\begin{align*}
\log_3{x}+\log_3{(x-2)}&=\log_3{8}&&\color{PineGreen}\longleftarrow\text{Note }x>0\text{ and }(x-2)>0\text{ therefore }x>2\\[0.5em]
\log_3{\left(x(x-2)\right)}&=\log_3{8} && \color{PineGreen}\longleftarrow\text{Apply product law}\\[0.5em]
x^2-2x&=8 && \color{PineGreen}\longleftarrow\text{``Cancel'' }\log_3\text{ and expand}\\[0.5em]
x^2-2x-8&=0 && \color{PineGreen}\longleftarrow\text{Equate quadratic equation to zero}\\[0.5em]
(x+2)(x-4)&=0 && \color{PineGreen}\longleftarrow\text{Factorise}\\[0.5em]
x+2=0,x-4&=0 &&\\[0.5em]
\hcancel[PineGreen]{x=-2},x&=4 && \color{PineGreen}\longleftarrow\text{Discard }x=-2\text{ since }x>2
\end{align*}

\end{tcolorbox}

All solutions for \(x\) must be checked carefully to verify that they satisfy the \emph{original} log equation.

\begin{enumerate}[resume]
    \item Solve for $x$:
    \begin{enumerate} 
        \begin{multicols}{2}
            \item $\log_{a}{x}+\log_{a}{(x+1)}=\log_{a}{6}$
            \item $\log_{4}{x}+\log_{4}{(x-3)}=\log_{4}{10}$
        \end{multicols}
        \begin{multicols}{2}    
            \item $\log_{6}{(x-1)}+\log_{6}{x}=1$
            \item $\log_{3}{(x+3)}-\log_{3}{x}=2$
        \end{multicols}
        \begin{multicols}{2}
            \item $\log_{5}{(x+2)}+\log_{5}{(x-3)}=\log_{5}{6}$
            \item $\log_{a}{3}=\log_{a}{(x+5)}+\log_{a}{2x}$
        \end{multicols}
    \end{enumerate}
    \item Solve each:
    \begin{enumerate}
        \begin{multicols}{2}
            \item $\log_{5}{x^2}+\log_{5}{4}=\log_{5}{36}$
            \item $\log_{6}{2}+\log_{6}{x^2}=\log_{6}{50}$
        \end{multicols}
        \begin{multicols}{2}
            \item $\log_{a}{7}+\log_{a}{x^2}=\log_{a}{14}$
            \item $2\log_{a}{x}+\log_{a}{3}=\log_{a}{48}$
        \end{multicols}
        \begin{multicols}{2}
            \item $\log_{9}{4}+2\log_{9}{x}=2\log_{9}{8}$
            \item $2\log_{7}{3x}+\log_{7}{2}=\log_{7}{18}$
        \end{multicols}
    \end{enumerate}
    \item Solve each for $a$:
    \begin{enumerate}
        \begin{multicols}{2}
            \item $\log_{a}{3}+\log_{a}{2}=1$
            \item $\log_{a}{18}-\log_{a}{2}=1$
        \end{multicols}
        \begin{multicols}{2}
            \item $\log_{a}{4}+\log_{a}{2}=3$
            \item $\log_{a}{12}-\log_{a}{4}=\frac{1}{2}$
        \end{multicols}
        \begin{multicols}{2}
            \item $\log_{a}{16}+\log_{a}{4}=2$
            \item $\log_{a}{50}-\log_{a}{2}=2$
        \end{multicols}
    \end{enumerate}
\end{enumerate}

\pagebreak

\section{Graphs of Logarithmic Functions}\label{graphs-of-logarithmic-functions}

\vspace{-0.5cm}

Given a function \(f\), defined on a suitable domain such that an inverse function exists, the graph of its inverse function \(y=f^{-1}(x)\) can be obtained by \textcolor{PineGreen}{\textbf{reflecting}} the graph of \(y=f(x)\) in the line \(y=x\).

Since the inverse to the exponential function \(f(x)=a^x\) is the \textcolor{PineGreen}{\textbf{logarithmic function}} \(f^{-1}(x)=\color{PineGreen}\log_a{x}\), knowledge of exponential graphs allows log graphs to be sketched:

\vspace{-0.6cm}

\begin{center}
    \begin{tikzpicture}[scale=0.8]
        \draw[-stealth] (-4,0) -- (5,0) node[below] {$x$};
        \draw[-stealth] (0,-4) -- (0,4) node[left] {$y$};
        \draw[dashed,thin] (-2.5,-2.5) -- (3.5,3.5) node[above right] {$y=x$};
        \draw[smooth,thick,domain=-4:3] plot (\x,{1.5^(\x)}) node[above left] {$y=a^x$};
        \draw[PineGreen,smooth,thick,domain=0.2:3.36] plot (\x,{(ln(\x))/ln(1.5)}) node[below right] {$y=\log_ax$};
        \draw[fill] (0,1) circle (0.07) node[above left] {$1$};
        \draw[PineGreen,fill] (1,0) circle (0.07) node[below right] {$1$};
        \draw[fill] (1,1.5) circle (0.07) node[above,xshift=-0.5em] {$(1,a)$};
        \draw[PineGreen,fill] (1.5,1) circle (0.07) node[right] {$(a,1)$};
        \node[below] at (12,4.5) {
        \begin{minipage}{7.5cm}
        The graph of $y=a^x$ passes through...
        \vspace{0.3cm}
        \begin{center}
        $(0,1)$ and $(1,a)$
        \end{center}
        \vspace{0.2cm}
        ...and has a \textbf{horizontal} asymptote of $x=0$.\\[0.1em]
        \end{minipage}
        };
        \node[below] at (12,1) {
        \begin{minipage}{7.5cm}
        The graph of \textcolor{PineGreen}{$y=\log_ax$} passes through...
        \begin{center}
        \textcolor{PineGreen}{$(1,0)$} and \textcolor{PineGreen}{$(a,1)$}.
        \end{center}
        \vspace{-0.1cm}
        ..and has a \textcolor{PineGreen}{\textbf{vertical}} asymptote of \textcolor{PineGreen}{$y=0$}.
        \end{minipage}
        };
        \node[below] at (12,-2.5) {
        \begin{minipage}{7.5cm}
        In general, if $y=f(x)$ passes through $(x,y)$ then $y=f^{-1}(x)$ passes through $(y,x)$.
        \end{minipage}
        };
    \end{tikzpicture}
\end{center}

\vspace{-0.5cm}

Knowledge of the graph of \(y=\log_a{x}\) may be combined with and understanding of graph transformations.

\begin{tcolorbox}[title=Example,colback=PineGreen!2!, colframe=PineGreen,height=6.7cm]

Sketch the graph of $y=\log_3{(x-1)}+2$.

\vspace{-0.2cm}

\tcblower

\textcolor{PineGreen}{Consider $y=\log_3{(x-1)}+2$ as a $f(x-1)+3$ transformation of $y=\log_3{x}$:}

\vspace{-0.4cm}

\begin{center}
    \begin{tikzpicture}[scale=0.6]
        \draw[-stealth,xshift=-12cm] (-0.5,0) -- (5,0) node[below] {$x$};
        \draw[-stealth,xshift=-12cm] (0,-2.2) -- (0,4) node[left] {$y$};
        \draw[smooth,thin,domain=0.1:3.5,PineGreen,xshift=-12cm] plot (\x,{(ln(\x))/(ln(3))}) node[right] {$y=\log_3{x}$};
        \draw[PineGreen,fill,xshift=-12cm] (1,0) circle (0.05) node[below] {$1$};
        \draw[PineGreen,fill,xshift=-12cm] (3,1) circle (0.05) node[above] {$(3,1)$};
        \draw[-stealth] (-0.5,0) -- (8,0) node[below] {$x$};
        \draw[-stealth] (0,-2.2) -- (0,4) node[left] {$y$};
        \draw[smooth,thick,domain=0.04:3.5,xshift=1cm,yshift=2cm] plot (\x,{(ln(\x))/(ln(3))}) node[right] {$y=\log_3{(x-1)}+2$};
        \draw[black,fill] (2,2) circle (0.05) node[below right] {$(2,2)$};
        \draw[black,fill] (4,3) circle (0.05) node[above] {$(4,3)$};
        \draw[dashed] (1,-2) -- (1,3.5) node[above] {\scriptsize{$x=1$}};
        \draw[PineGreen,-stealth] (-4.5,1) -- (-1.5,1);
        \node[above,PineGreen] at (-3,1) {\textit{``Right 1''}};
        \node[below,PineGreen] at (-3,1) {\textit{``Up 2''}};
    \end{tikzpicture}
\end{center}

\end{tcolorbox}

\vspace{-0.5cm}

\subsection*{Exercise 15.5}\label{exercise-15.5}
\addcontentsline{toc}{subsection}{Exercise 15.5}

\vspace{-0.5cm}

\begin{enumerate}
    \item Sketch each of the following:
    \begin{enumerate}
    \begin{multicols}{3}
        \item $y=\log_{5}{x}$
        \item $y=\log_{2}{x}$
        \item $y=\log_{9}{x}$
    \end{multicols}
    \end{enumerate}
    \item Sketch each of the following, showing clearly where any asymptote is not along an axis:
    \begin{enumerate}
    \begin{multicols}{3}
        \item $y=\log_{4}{x}+1$
        \item $y=\log_{7}{x}-2$
        \item $y=\log_{5}{x}+3$
    \end{multicols}
    \begin{multicols}{3}
        \item $y=\log_{8}{(x+5)}$
        \item $y=\log_{2}{(x-3)}$
        \item $y=\log_{6}{(x+2)}$
    \end{multicols}
    \begin{multicols}{3}
        \item $y=\log_{7}{(x+3)}+2$
        \item $y=\log_{5}{(x-3)}-1$
        \item $y=\log_{6}{(x+2)}+1$
    \end{multicols}
    \end{enumerate}
\end{enumerate}

\pagebreak

As in section 15.1 (Graphs of Exponential Functions), substitution of known coordinates on the graph of a log function into its equation can allow unknown constants it contains to be determined.

\begin{tcolorbox}[title=Example,colback=PineGreen!2!, colframe=PineGreen]

The graph of $y=\log_a{(x+b)}$ is below. Find the values of $a$ and $b$.

\begin{center}
    \begin{tikzpicture}[scale=0.9]
        \draw[-stealth] (-4,0) -- (5,0) node[below] {$x$};
        \draw[-stealth] (0,-1.5) -- (0,1.5) node[left] {$y$};
        \draw[smooth,thick,domain=0.1:7,xshift=-3cm] plot (\x,{ln(\x)/ln(5)}) node[right] {$y=\log_a{(x+b)}$};
        \draw[fill] (-2,0) circle (0.05) node[below] {$-2$};
        \draw[fill] (2,1) circle (0.05) node[below] {$(2,1)$};
        \draw[dashed] (-3,-1.5) -- (-3,1.1) node[above] {$x=-3$};
    \end{tikzpicture}
\end{center}

\vspace{-0.3cm}

\tcblower

\vspace{-0.6cm}
\begin{align*}
    0&=\log_a{(-2+b)} && \color{PineGreen}\longleftarrow\text{Substitute }(-2,0)\\[0.5em]
    -2+b&=1 &&\color{PineGreen}\longleftarrow\text{Since }\log_a{1}=0\text{ for all }a\\[0.5em]
    b&=3 && \color{PineGreen}\longleftarrow\text{Solve to obtain }b\\[1em]
    1&=\log_a{(2+3)} && \color{PineGreen}\longleftarrow\text{Substitute }(2,1)\text{ and }b=3\\[0.5em]
    a&=5 &&\color{PineGreen}\longleftarrow\text{Since }\log_a{a}=1\text{ for all }a\\[1em]
    y&=\log_5{(x+3)}&&\color{PineGreen}\longleftarrow\text{State equation}
\end{align*}

\end{tcolorbox}

\begin{enumerate}[resume]
    \item Determine the equation of each, using the form given:
    \begin{enumerate}
        \begin{multicols}{2}
            \item $y=\log_4{(x+a)}$\\
                \begin{tikzpicture}
                    \draw[white] (-2,-2) rectangle (5.5,3);
                    \draw[-stealth] (-2,0) -- (5,0) node [above] {$x$};
                    \draw[-stealth] (0,-2) -- (0,3) node[left] {$y$};
                    \draw[smooth,thick,domain=0.1:6,xshift=-2cm,yshift=0cm] plot (\x,{ln(\x)/ln(4)});
                    \draw[fill] (-1,0) circle (0.05) node[below] {$-1$};
                    \draw[fill] (2,1) circle (0.05) node[above] {$(2,1)$};
                    \draw[dashed] (-2,-2) -- (-2,3);
                \end{tikzpicture}
            \item $y=\log_a{(x-2)}$\\
                \begin{tikzpicture}
                    \draw[white] (-2,-2) rectangle (5.5,3);
                    \draw[-stealth] (-2,0) -- (5,0) node [above] {$x$};
                    \draw[-stealth] (-2,-2) -- (-2,3) node[left] {$y$};
                    \draw[smooth,thick,domain=0.1:4,xshift=0cm,yshift=0cm] plot (\x,{ln(\x)/ln(3)});
                    \draw[fill] (1,0) circle (0.05) node[below] {$3$};
                    \draw[fill] (3,1) circle (0.05) node[above] {$(5,1)$};
                    \draw[dashed] (0,-2) -- (0,3);
                \end{tikzpicture}
        \end{multicols}
        \begin{multicols}{2}
            \item $y=\log_a{(x+b)}$\\
                \begin{tikzpicture}
                    \draw[white] (-0.5,-2) rectangle (5.5,3);
                    \draw[-stealth] (-0.5,0) -- (6,0) node [above] {$x$};
                    \draw[-stealth] (0,-2) -- (0,3) node[left] {$y$};
                    \draw[smooth,thick,domain=0.1:5.5,xshift=1cm,yshift=0cm] plot (\x,{ln(\x)/ln(5)});
                    \draw[fill] (2,0) circle (0.05) node[below] {$2$};
                    \draw[fill] (6,1) circle (0.05) node[above] {$(6,1)$};
                    \draw[dashed] (1,-2) -- (1,3);
                \end{tikzpicture}
            \item $y=\log_a{(bx)}$\\
                \begin{tikzpicture}
                    \draw[white] (-2,-2) rectangle (5.5,3);
                    \draw[-stealth] (-2,0) -- (5,0) node [above] {$x$};
                    \draw[-stealth] (0,-2) -- (0,3) node[left] {$y$};
                    \draw[smooth,thick,domain=0.1:9,xshift=0cm,yshift=0cm,xscale=0.5] plot (\x,{ln(\x)/ln(6)});
                    \draw[fill] (0.5,0) circle (0.05) node[below right] {$0.5$};
                    \draw[fill] (3,1) circle (0.05) node[above] {$(3,1)$};
                \end{tikzpicture}
        \end{multicols}
    \end{enumerate}
\end{enumerate}

\pagebreak

\section{Solving Exponential Equations}\label{solving-exponential-equations}

\vspace{-0.5cm}

To solve an \textcolor{PineGreen}{\textit{exponential equation}} such as \(10=2^{\color{PineGreen}x}\), which has a non-integer solution, the relationship between a log and an exponential of the same base, stated in section 15.2, can be used as follows:

\vspace{-0.3cm}

\begin{align*}
    \text{If }\hspace{1.9cm} 10&=2^{\color{PineGreen}x}\\[0.5em]
    \text{ then }\hspace{0.5cm} \color{PineGreen}\log_2{\color{black}10}&=x
\end{align*}

\vspace{-0.3cm}

The solution, using a calculator and correct to three significant figures, is \(x=3.32\). This makes use of the property \(\color{PineGreen}\log_a{a^{\color{black}x}}\color{black}=x\), which follows from the inverse relationship between a log and exponential.

\begin{tcolorbox}[title=Example,colback=PineGreen!2!, colframe=PineGreen]

Solve the equation $14=3\times5^x$, giving the answer correct to three significant figures. \faCalculator

\tcblower

\vspace{-0.5cm}

\begin{align*}
14&=3\times 5^x &&\\[0.5em]
\frac{14}{3}&=5^x && \color{PineGreen}\longleftarrow\text{Divide both sides by 3}\\[0.5em]
\log_5{\left(\frac{14}{3}\right)}&=x && \color{PineGreen}\longleftarrow\text{Take }\log_5\text{ of both sides}\\[0.5em]
x&=0.957 \text{ (to 3sf)} && \color{PineGreen}\longleftarrow\text{Calculate the solution}
\end{align*}

\end{tcolorbox}

\vspace{-0.3cm}

\subsection*{Exercise 15.6}\label{exercise-15.6}
\addcontentsline{toc}{subsection}{Exercise 15.6}

\vspace{-0.3cm}

\begin{enumerate}
    \item Solve, giving the answers to three significant figures: \faCalculator
    \begin{enumerate}
        \begin{multicols}{4}
            \item $30=5^x$
            \item $6=3^x$
            \item $13=8^x$
            \item $3=7^x$
        \end{multicols}
        \begin{multicols}{4}
            \item $2^x=11$
            \item $10^x=500$
            \item $6^x=3$
            \item $4^x=65$
        \end{multicols}
        \begin{multicols}{4}
            \item $19=2.1^t$
            \item $1.3^{2y}=0.7$
            \item $7^{4p}=\frac{1}{2}$
            \item $\frac{2}{3}=4^{0.7k}$
        \end{multicols}
    \end{enumerate}
    \item Solve, giving the answers to three significant figures: \faCalculator
    \begin{enumerate}
        \begin{multicols}{4}
            \item $12=4\times8^x$
            \item $24=6\times7^x$
            \item $2=7\times5^x$
            \item $21=4\times3^x$
        \end{multicols}
        \begin{multicols}{4}
            \item $6\times8^x=5$
            \item $3\times4^x=45$
            \item $0.03\times8^x=0.7$
            \item $18\times5^x=0.7$
        \end{multicols}
        \begin{multicols}{4}
            \item $5=3\times4^q$
            \item $7\times6^{0.03t}=5$
            \item $10\times10^{-3y}=9$
            \item $10=9\times10^{-0.037k}$
        \end{multicols}
    \end{enumerate}
    \item Solve, giving the answer to three significant figures: \faCalculator
    \begin{enumerate}
        \begin{multicols}{3}
            \item $20=4\times7^x+3$
            \item $7=\frac{1}{2}\times3^x-6$
            \item $5+3\times4^x=23$
        \end{multicols}
        \begin{multicols}{3}
            \item $3\times\left(\frac{1}{2}\right)^x=7$
            \item $6\times1.4^x-2.3=31.4$
            \item $0.4=3\times0.3^x-0.5$
        \end{multicols}
        \begin{multicols}{3}
            \item $3\times4^{0.0723k}=21$
            \item $11\times3^{-0.104x}=6$
            \item $3=5\times2^{-0.00562k}$
        \end{multicols}
    \end{enumerate}
\end{enumerate}

\pagebreak

\section{Natural Logarithms}\label{natural-logarithms}

\vspace{-0.5cm}

Some logarithms with particular \emph{bases} are routinely used in various fields, such as the \emph{binary logarithm} \(\left(\log_2\right)\) in computer science, or the the \emph{decimal} or \emph{common logarithm} \(\left(\log_{10}\text{ or just }\log\right)\) in engineering.

The most useful logarithm in mathematics is the \textcolor{PineGreen}{\textbf{natural logarithm}} (\(\log_{\color{PineGreen}e}\) or more often just \textcolor{PineGreen}{$\ln$}), which takes Euler's Number \textcolor{PineGreen}{$e$} as its base, where \textcolor{PineGreen}{$e\approx 2.719$}. (Its lies in the ease which which exponentials in base \(e\) can be used together with calculus, which is beyond the scope of this course.)

The \emph{laws of logs} apply to natural logarithms since they apply to \emph{all} \(\log_a\) where \(a>0\):

\begin{tcolorbox}[title=Example,colback=PineGreen!2!, colframe=PineGreen,height=6.7cm]

Simplify $\ln{48}-3\ln{2}$, giving the answer in the form $\ln{p}$.

\tcblower

\vspace{-0.5cm}

\begin{align*}
&\ln{48}-2\ln{2} &&\\[0.5em]
=&\ln{48}-\ln{\left(2^3\right)} && \color{PineGreen}\longleftarrow\text{Apply the power law}\\[0.5em]
=&\ln{48}-\ln{8} &&\\[0.5em]
=&\ln{\left(\frac{48}{8}\right)}&& \color{PineGreen}\longleftarrow\text{Apply the quotient law}\\[0.5em]
=&\ln{6} && \color{PineGreen}\longleftarrow\text{Simplify}
\end{align*}

\vspace{-0.2cm}

\end{tcolorbox}

Exponential equations in base \(\color{PineGreen}e\) are solved using the natural logarithm \(\color{PineGreen}\ln\), since \(\color{PineGreen}\ln{\left(e^{\color{black}x}\right)}=x\):

\begin{tcolorbox}[title=Example,colback=PineGreen!2!, colframe=PineGreen,height=7.6cm]

Given $37=21e^{-0.48x}$, find the value of $k$ to three significant figures. \faCalculator

\tcblower

\vspace{-0.5cm}

\begin{align*}
37&=21e^{-0.48k} &&\\[0.5em]
\frac{37}{21}&=e^{-0.48k} && \color{PineGreen}\longleftarrow\text{Divide both sides by 21}\\[0.5em]
\ln{\left(\frac{37}{21}\right)}&=-0.48k && \color{PineGreen}\longleftarrow\text{Take }\ln\text{ of both sides}\\[0.5em]
\frac{\ln{\left(\frac{37}{21}\right)}}{-0.48}&=k && \color{PineGreen}\longleftarrow\text{Divide both sides by }-0.48\\[0.5em]
k&=1.18 \text{ (to 3sf)} && \color{PineGreen}\longleftarrow\text{Calculate the solution}
\end{align*}

\vspace{-0.2cm}

\end{tcolorbox}

\vspace{-0.5cm}

\subsection*{Exercise 15.7}\label{exercise-15.7}
\addcontentsline{toc}{subsection}{Exercise 15.7}

\vspace{-0.5cm}

\begin{enumerate}
    \item Fo each, find the value of $k$ to three significant figures: \faCalculator
    \vspace{-0.2cm}
    \begin{enumerate}
        \begin{multicols}{4}
            \item $0.7=e^{k}$
            \item $1.8=e^{k}$
            \item $3e^{k}=24$
            \item $4=7e^{x}$
        \end{multicols}
        \begin{multicols}{4}
            \item $125=5e^{2k}$
            \item $50=7e^{4k}$
            \item $10=12e^{-3k}$
            \item $17=100^{-5k}$
        \end{multicols}
    \end{enumerate}
    \item Solve each for $x$, giving the answers to three significant figures: \faCalculator
    \vspace{-0.2cm}
    \begin{enumerate}
        \begin{multicols}{4}
            \item $120=85e^{0.475x}$
            \item $50=32e^{0.0273x}$
            \item $14=23e^{-0.381x}$
            \item $3P=Pe^{0.045x}$
        \end{multicols}
    \end{enumerate}
\end{enumerate}

\vspace{-0.2cm}

\pagebreak

\section{Exponential Growth and Decay}\label{exponential-growth-and-decay}

\vspace{-0.5cm}

The behaviour of many real-world things are described by \textcolor{PineGreen}{\textbf{exponential growth}} or \textcolor{PineGreen}{\textbf{exponential decay}}.

Time, \(t\), is a common variable for such a function in place of \(x\), and it typically describes the behaviour of some real-world variable which is named suitably in place of \(y\) of \(f(x)\), such as \(V\) for \emph{``value''}, \(P\) for \emph{``population size''} or \(M\) for \emph{``mass''}. Where \(\color{PineGreen}k>0\):

\setlength{\columnsep}{30pt}

\begin{multicols}{2}

\vspace{-0.2cm}

$V=V_0e^{\color{PineGreen}k\color{black}t}$ describes the \textcolor{PineGreen}{\textbf{exponential growth}} of $V$ from some initial value $V_0$ as $t$ increases.

\vspace{-0.5cm}
\begin{center}
    \begin{tikzpicture}[scale=0.7]
        \draw[-stealth] (0,0) -- (8,0) node[below] {$t$};
        \draw[-stealth] (0,0) -- (0,6) node[left] {$V$};
        \node[below left] {O};
        \draw[smooth,thick,PineGreen,domain=0:5] plot (\x,{0.5*1.6^(\x)}) node[right] {$\color{black}V=V_0e^{\color{PineGreen}k\color{black}t}$};
        \draw[fill] (0,0.5) circle (0.05) node[left] {$V_0$};
    \end{tikzpicture}
\end{center}

\columnbreak

\vspace{-0.2cm}

$V=V_0e^{\color{PineGreen}-k\color{black}t}$ describes the \textcolor{PineGreen}{\textbf{exponential decay}} of $V$ from some initial value $V_0$ as $t$ increases.

\vspace{-0.5cm}
\begin{center}
    \begin{tikzpicture}[scale=0.7]
        \draw[-stealth] (0,0) -- (8,0) node[below] {$t$};
        \draw[-stealth] (0,0) -- (0,6) node[left] {$V$};
        \node[below left] {O};
        \draw[smooth,thick,PineGreen,domain=0:5] plot (\x,{4*0.7^(\x)}) node[right] {$\color{black}V=V_0e^{\color{PineGreen}-k\color{black}t}$};
        \draw[fill] (0,4) circle (0.05) node[left] {$V_0$};
    \end{tikzpicture}
\end{center}

\end{multicols}

\vspace{-0.5cm}

\setlength{\columnsep}{10pt}

\begin{tcolorbox}[title=Example,colback=PineGreen!2!, colframe=PineGreen]

80 micrograms of a radioactive substance, which decays over time, is obtained. The mass $M$ micrograms of the substance remaining after $t$ hours can be modelled by the equation:

\vspace{-0.3cm}

\[M=M_0e^{-kt}\]

\vspace{-0.1cm}

\begin{enumerate}
    \item[(a)] State the value of $M_0$.
\end{enumerate}

\tcblower

\vspace{-0.5cm}

\begin{align*}
M_0&=80 && \color{PineGreen}\longleftarrow M_0\text{ is the initial value}
\end{align*}

\end{tcolorbox}

Constant \(k\) may be found through substitution if initial value \(V_0\) is known and the value of \(V\) at time \(t\).

\begin{tcolorbox}[colback=PineGreen!2!, colframe=PineGreen]

\begin{enumerate}
    \item[(b)] After 5 hours, only 50.7 micrograms of the substance remains.\newline Determine the value of $k$, rounded to three significant figures. \faCalculator
\end{enumerate}

\tcblower

\vspace{-0.5cm}

\begin{align*}
50.7&=80e^{-k\times 5} && \color{PineGreen}\longleftarrow \text{Substitute }M_0=80\text{, }M=50.7\text{ and }t=5\\[0.5em]
\frac{50.7}{80}&=e^{-5k} && \color{PineGreen}\longleftarrow \text{Divide by }80\\[0.5em]
\ln{\left(\frac{50.7}{80}\right)}&=-5k && \color{PineGreen}\longleftarrow \text{Take natural log of both sides }\\[0.5em]
\frac{\ln{\left(\frac{50.7}{80}\right)}}{-5}&=k && \color{PineGreen}\longleftarrow \text{Divide both sides by }(-5)\\[0.5em]
k&=0.0912 && \color{PineGreen}\longleftarrow \text{Calculate and round as required}
\end{align*}

\end{tcolorbox}

A feature of exponential functions is that the time taken for the value to grow or decay by any given proportion is always constant. For example the time taken for an exponential growth function to double a value from 20 to 40 will be the same time it will take to double from 50 to 100.

\pagebreak

\begin{tcolorbox}[colback=PineGreen!2!, colframe=PineGreen]

\begin{enumerate}
    \item[(c)] Find the time it will take for only half of the original substance to remain (its \textit{half-life}). \faCalculator
\end{enumerate}

\tcblower

\vspace{-0.5cm}

\begin{align*}
40&=80e^{-0.0912t} && \color{PineGreen}\longleftarrow \text{Substitute }M=40\text{ and }k=0.0912\text{ is now known}\\[0.5em]
\frac1{2}&=e^{-0.0912t} && \color{PineGreen}\longleftarrow \text{Divide by }80\text{ and simplify}\\[0.5em]
\ln{\left(\frac{1}{2}\right)}&=-0.0912t && \color{PineGreen}\longleftarrow \text{Take natural log of both sides }\\[0.5em]
\frac{\ln{\left(\frac{1}{2}\right)}}{-0.0912}&=t && \color{PineGreen}\longleftarrow \text{Divide both sides by }(-5)\\[0.5em]
t&=7.60\text{ hours} && \color{PineGreen}\longleftarrow \text{Calculate}
\end{align*}

\end{tcolorbox}

\vspace{-0.5cm}

\subsection*{Exercise 15.8}\label{exercise-15.8}
\addcontentsline{toc}{subsection}{Exercise 15.8}

\vspace{-0.3cm}

All questions: \faCalculator

\vspace{-0.2cm}

\begin{enumerate}
    \item For each \textit{exponential growth} function, find the value of $t$ it will take for the value to double:
    \begin{enumerate}
        \begin{multicols}{3}
            \item $V=40e^{0.3t}$
            \item $P(t)=25e^{0.08t}$
            \item $M=M_0e^{0.00572t}$
        \end{multicols}
    \end{enumerate}
    \item The number of bacteria in a population, $B$, can be modelled using the equation $B=400e^{0.0376t}$ where $t$ is time measured in minutes.
    \begin{enumerate}
        \item State the number of bacteria in the population initially.
        \item Find, to the nearest thousand, the number of bacteria after 45 minutes.
        \item Determine the time it takes for the population of bacteria to double in size.
    \end{enumerate}
    \item A population of deer is growing exponentially in an area, and the size of the population, $P$, can be described by the equation $P=860e^{0.173t}$, where $t$ is the number of years from 2026.
    \begin{enumerate}
        \item State the number of deer in the population in 2026.
        \item Find, to the nearest ten, the number of deer expected to be in the population in 2028.
        \item Determine the time it will take for the population of deer to double in size.
    \end{enumerate}
    \item For each \textit{exponential decay} function, find the value of $t$ it will take for the value to halve:
    \begin{enumerate}
        \begin{multicols}{3}
            \item $T=20e^{-0.8t}$
            \item $V(t)=360e^{-0.072t}$
            \item $P=P_0e^{-0.00197t}$
        \end{multicols}
    \end{enumerate}
    \item The temperature $T$ (in $\degree\text{C}$) of a substance $t$ minutes after the end of an experiment can be modelled by the equation $T=T_0e^{-kt}$. After 7 minutes, the substance has cooled from $140\degree\text{C}$ to $55\degree\text{C}$.
    \begin{enumerate}
        \item Determine the value of $k$ to three significant figures.
        \item Find the time taken for the substance to cool to $30\degree\text{C}$.
    \end{enumerate}
    \item The concentration $C$ of a drug in a patient's blood $t$ hours after being given a dose can be modelled by the equation $C=28e^{-kt}$.
    \begin{enumerate}
        \item If the concentration is 23 after 90 minutes, calculate the value of $k$ to three significant figures.
        \item Find the time taken for the concentrate to reduce by half.
        \item Find the time taken for the concentration to be only 15\% of its original value.
    \end{enumerate}
\end{enumerate}

\pagebreak

\section{\texorpdfstring{Experimental Data of the form \(y=ax^b\)}{Experimental Data of the form y=ax\^{}b}}\label{experimental-data-of-the-form-yaxb}

\vspace{-0.5cm}

Where a relationship between variables \(x\) and \(y\) is a \textcolor{PineGreen}{power} function in the form
\(\color{PineGreen}y=a\color{PineGreen}x^{\color{PineGreen}b}\), taking any \(\log\) of both sides will reveal the \textcolor{PineGreen}{linear relationship} that therefore exists between \(\color{PineGreen}\log{x}\) and \(\color{PineGreen}\log{y}\).

\vspace{-0.4cm}

\begin{minipage}{0.33\textwidth}
\begin{center}
\begin{tikzpicture}[scale=0.8]
          \draw[-stealth] (-0.2,0) -- (4,0) node[below] {$\color{PineGreen}x$};
          \draw[-stealth] (0,-0.2) -- (0,4) node[left] {$\color{PineGreen}y$};
          \node[below left] {\scriptsize{O}};
          \draw[PineGreen,thick,smooth,domain=0:2.6] plot (\x,{0.5*(\x)^2}) node[above] {$y\color{black}=a\color{PineGreen}x^{\color{black}b}$};
\end{tikzpicture}
\end{center}
\end{minipage}\begin{minipage}{0.33\textwidth}
\begin{align*}
    \color{PineGreen}y&=a\color{PineGreen}x^{\color{black}b}\\[0.3em]
    \log{y}&=\log{\left(ax^b\right)}\\[0.5em]
    \log{y}&=\log{x^b}+\log{a}\\[0.5em]
    \color{PineGreen}\log{y}&=b\color{PineGreen}\log{x}\color{black}+\log{a}\\
    &\hspace{0.45cm} \downarrow\hspace{1.6cm}\downarrow\\
    \color{PineGreen}\log{y}&=m\color{PineGreen}\log{x}\color{black}+c
\end{align*}
\end{minipage}\begin{minipage}{0.33\textwidth}
\begin{center}
\begin{tikzpicture}[scale=0.8]
          \draw[-stealth] (-0.2,0) -- (4,0) node[below] {$\color{PineGreen}\log{x}$};
          \draw[-stealth] (0,-0.2) -- (0,4) node[left] {$\color{PineGreen}\log{y}$};
          \node[below left] {\scriptsize{O}};
          \draw[PineGreen,thick,smooth,domain=0:3] plot (\x,{0.8*(\x)+1});
          \draw[fill] (0,1) circle (0.05) node[left] {$\log{a}$};
          \node[rotate=38.66] at (1.8,2) {gradient $=b$};
\end{tikzpicture}
\end{center}
\end{minipage}

Drawing such a plot of \(\color{PineGreen}\log{y}\) against \(\color{PineGreen}\log{x}\) and obtaining a straight line can be used to check for a power relationship. Where a relationship is of the form \(y=ax^b\), determining the values of \(a\) and \(b\) is possible by first finding the \emph{gradient} and \(y\)\emph{-intercept} on the \(\color{PineGreen}\log{y}\sim\log{x}\) graph.

\begin{tcolorbox}[title=Example,colback=PineGreen!2!, colframe=PineGreen]

Two variables, $x$ and $y$, are connected by the equation $y=kx^n$. The graph of $\log_3{y}$ against $\log_3{x}$ is a straight line as shown. Determine the values of $k$ and $n$.

\vspace{-0.3cm}

\begin{center}
\begin{tikzpicture}[scale=0.8]
          \draw[-stealth] (-0.2,0) -- (4,0) node[below] {$\log_3{x}$};
          \draw[-stealth] (0,-0.2) -- (0,4) node[left] {$\log_3{y}$};
          \node[below left] {\scriptsize{O}};
          \draw[thick] plot (0,1) -- (3,2.5);
          \draw[fill] (0,1) circle (0.05) node[left] {$(0,3)$};
          \draw[fill] (2,2) circle (0.05) node[below right] {$(2,6)$};
\end{tikzpicture}
\end{center}

\vspace{-0.6cm}

\tcblower

\vspace{-0.5cm}

\begin{align*}
\log_{3}y&=\log_3{\left(kx^n\right)} && \color{PineGreen}\longleftarrow \text{Take }\log_3\text{ of both sides}\\[0.5em]
\log_{3}y&=\log_3{x^n}+\log_3{k} && \color{PineGreen}\longleftarrow \text{Apply product law}\\[0.5em]
\log_{3}y&=n\log_3{x}+\log_3{k} && \color{PineGreen}\longleftarrow \text{Apply power law}\\[1em]
\text{gradient}&=\frac{6-3}{2-0}=\frac{3}{2} && \color{PineGreen}\longleftarrow \text{Find gradient}\\[0.5em]
n&=\frac{3}{2} && \color{PineGreen}\longleftarrow \text{Equate to }n\\[1em]
y\text{-intercept}&=3 && \color{PineGreen}\longleftarrow \text{Find }y\text{-intercept}\\[0.5em]
\log_{3}k&=3 && \color{PineGreen}\longleftarrow \text{Equate to }\log_3{k}\\[0.5em]
\log_{3}k&=3\color{PineGreen}\log_3{3} && \color{PineGreen}\longleftarrow \text{Solve for }k\\[0.5em]
\log_{3}k&=\log_3{3^3} &&\\[0.5em]
k&=27 &&
\end{align*}

\end{tcolorbox}

Therefore the relationship between \(x\) and \(y\) is \(y=27x^{\frac{3}{2}}\).

\pagebreak

\newpage

\subsection*{Exercise 15.9}\label{exercise-15.9}
\addcontentsline{toc}{subsection}{Exercise 15.9}

\begin{enumerate}
    \item For each graph where the relationship between $x$ and $y$ is $y=kx^n$, determine the values of $k$ and $n$:
    \begin{multicols}{2}
        \begin{center}
            \begin{tikzpicture}
                \draw[white] (-3,-1) rectangle (6,4);
                \node[below right] at (-3,4) {(a)};
                \draw[-stealth] (-2,0) -- (5,0) node[below] {$\log_7{x}$};
                \draw[-stealth] (0,-1) -- (0,3.5) node[left] {$\log_7{y}$};
                \draw[thick] (0,0.5) node[left] {$(0,2)$} -- (3,3) -- (3.3,3.25);
                \node[below right,xshift=-0.1cm,yshift=0.1cm] at (3,3) {$(2,12)$};
                \draw[fill] (0,0.5) circle (0.05);
                \draw[fill] (3,3) circle (0.05);
                \node[below left] {\scriptsize{O}};
            \end{tikzpicture}
        \end{center}
        \begin{center}
            \begin{tikzpicture}
                \draw[white] (-3,-1) rectangle (6,4);
                \node[below right] at (-3,4) {(b)};
                \draw[-stealth] (-2,0) -- (5,0) node[below] {$\log_2{x}$};
                \draw[-stealth] (0,-1) -- (0,3.5) node[left] {$\log_2{y}$};
                \draw[thick] (0,0.8) node[left] {$(0,4)$} -- (4,2.4) -- (4.2,2.48);
                \node[below right,xshift=-0.1cm,yshift=0.1cm] at (4,2.4) {$(16,12)$};
                \draw[fill] (0,0.8) circle (0.05);
                \draw[fill] (4,2.4) circle (0.05);
                \node[below left] {\scriptsize{O}};
            \end{tikzpicture}
        \end{center}
    \end{multicols}
    \begin{multicols}{2}
        \begin{center}
            \begin{tikzpicture}
                \draw[white] (-3,-1) rectangle (6,4);
                \node[below right] at (-3,4) {(c)};
                \draw[-stealth] (-2,0) -- (5,0) node[below] {$\log_5{x}$};
                \draw[-stealth] (0,-1) -- (0,3.5) node[left] {$\log_5{y}$};
                \draw[thick] (0,2.4) node[left] {$(0,3)$} -- (3,0.8) -- (3.75,0.4);
                \node[above right,xshift=-0.1cm,yshift=-0.1cm] at (3,0.8) {$(3,1)$};
                \draw[fill] (0,2.4) circle (0.05);
                \draw[fill] (3,0.8) circle (0.05);
                \node[below left] {\scriptsize{O}};
            \end{tikzpicture}
        \end{center}
        \begin{center}
            \begin{tikzpicture}
                \draw[white] (-3,-1) rectangle (6,4);
                \node[below right] at (-3,4) {(d)};
                \draw[-stealth] (-2,0) -- (5,0) node[below] {$\log_8{x}$};
                \draw[-stealth] (0,-1) -- (0,3.5) node[left] {$\log_8{y}$};
                \draw[thick] (0,2) node[left] {$2$} -- (4,0) node[below] {$4$};
                \draw[fill] (0,2) circle (0.05);
                \draw[fill] (4,0) circle (0.05);
                \node[below left] {\scriptsize{O}};
            \end{tikzpicture}
        \end{center}
    \end{multicols}
    \item For each graph where the relationship between $x$ and $y$ is $y=ax^b$, determine the values of $a$ and $b$:
    \begin{multicols}{2}
        \begin{center}
            \begin{tikzpicture}
                \draw[white] (-3,-2.5) rectangle (6,2.5);
                \node[below right] at (-3,2.5) {(a)};
                \draw[-stealth] (-2,0) -- (5,0) node[below] {$\log_3{x}$};
                \draw[-stealth] (0,-2.5) -- (0,2) node[left] {$\log_3{y}$};
                \draw[thick] (0,-1) -- (4.4,1.2);
                \draw[fill] (0,-1) circle (0.05);
                \draw[fill] (4,1) circle (0.05);
                \node[left] at (0,-1) {$-1$};
                \node[below right,xshift=-0.1cm,yshift=0.1cm] at (4,1) {$(4,1)$};
                \node[below left] {\scriptsize{O}};
            \end{tikzpicture}
        \end{center}
        \begin{center}
            \begin{tikzpicture}
                \draw[white] (-3,-2.5) rectangle (6,2.5);
                \node[below right] at (-3,2.5) {(b)};
                \draw[-stealth] (-2,0) -- (5,0) node[below] {$\log_5{x}$};
                \draw[-stealth] (0,-2.5) -- (0,2) node[left] {$\log_5{y}$};
                \node[below left] {\scriptsize{O}};
                \draw[thick] (0,-2) -- (2.25,1);
                \draw[fill] (0,-2) circle (0.05) node[left] {$\left(0,-2\right)$};
                \draw[fill] (2.25,1) circle (0.05);
                \node[right] at (2.25,1) {$(3,1)$};
            \end{tikzpicture}
        \end{center}
    \end{multicols}
    \begin{multicols}{2}
        \begin{center}
            \begin{tikzpicture}
                \draw[white] (-3,-2.5) rectangle (6,2.5);
                \node[below right] at (-3,2.5) {(c)};
                \draw[-stealth] (-2,0) -- (5,0) node[below] {$\log_9{x}$};
                \draw[-stealth] (0,-2.5) -- (0,2) node[left] {$\log_9{y}$};
                \node[below left] {\scriptsize{O}};
                \draw[fill] (0,1) circle (0.05);
                \draw[thick] (0,1) node[left] {$\frac{1}{2}$} -- (3,-1) node[below] {gradient $=-2$};
            \end{tikzpicture}
        \end{center}
        \begin{center}
            \begin{tikzpicture}
                \draw[white] (-3,-2.5) rectangle (6,2.5);
                \node[below right] at (-3,2.5) {(d)};
                \draw[-stealth] (-2,0) -- (5,0) node[below] {$\log_8{x}$};
                \draw[-stealth] (0,-2.5) -- (0,2) node[left] {$\log_8{y}$};
                \node[below left] {\scriptsize{O}};
                \draw[thick] (0,-1) node[left] {$-\frac{1}{3}$} -- (2.5,1.5);
                \draw[fill] (0,-1) circle (0.05);
                \draw[fill] (1,0) circle (0.05) node[below] {$6$};
            \end{tikzpicture}
        \end{center}
    \end{multicols}
\end{enumerate}

\newpage

\section{\texorpdfstring{Experimental Data of the form \(y=ab^x\)}{Experimental Data of the form y=ab\^{}x}}\label{experimental-data-of-the-form-yabx}

\vspace{-0.5cm}

Where a relationship between variables \(x\) and \(y\) is an \textcolor{PineGreen}{exponential} function in the form
\(\color{PineGreen}y=ab^{\color{PineGreen}x}\), taking any \(\log\) of both sides will reveal the \textcolor{PineGreen}{linear relationship} that therefore exists between \(\color{PineGreen}x\) and \(\color{PineGreen}\log y\).

\vspace{-0.3cm}

\begin{minipage}{0.33\textwidth}
\begin{center}
\begin{tikzpicture}[scale=0.8]
          \draw[-stealth] (-0.2,0) -- (4,0) node[below] {$\color{PineGreen}x$};
          \draw[-stealth] (0,-0.2) -- (0,4) node[left] {$\color{PineGreen}y$};
          \node[below left] {\scriptsize{O}};
          \draw[PineGreen,thick,smooth,domain=0:3.1] plot (\x,{0.5*1.8^(\x)}) node[above] {$y\color{black}=ab^{\color{PineGreen}x}$};
\end{tikzpicture}
\end{center}
\end{minipage}\begin{minipage}{0.33\textwidth}
\begin{align*}
    \color{PineGreen}y&=ab^{\color{PineGreen}x}\\[0.5em]
    \log{y}&=\log{\left(ab^x\right)}\\[0.5em]
    \log{y}&=\log{b^x}+\log{a}\\[0.5em]
    \color{PineGreen}\log{y}&=\color{PineGreen}x\color{black}\log{b}\color{black}+\log{a}\\
    &\hspace{0.75cm} \downarrow\hspace{1.1cm}\downarrow\\
    \color{PineGreen}\log{y}&=\color{PineGreen}x\color{black}m\hspace{0.3cm}+\hspace{0.2cm}c
\end{align*}
\end{minipage}\begin{minipage}{0.33\textwidth}
\begin{center}
\begin{tikzpicture}[scale=0.8]
          \draw[-stealth] (-0.2,0) -- (4,0) node[below] {$x$};
          \draw[-stealth] (0,-0.2) -- (0,4) node[left] {$\color{PineGreen}\log{y}$};
          \node[below left] {\scriptsize{O}};
          \draw[PineGreen,thick,smooth,domain=0:3] plot (\x,{0.8*(\x)+1});
          \draw[fill] (0,1) circle (0.05) node[left] {$\log{a}$};
          \node[rotate=38.66] at (1.8,2) {gradient $=\log{b}$};
\end{tikzpicture}
\end{center}
\end{minipage}

Drawing such a plot of \(\color{PineGreen}\log{y}\) against \(\color{PineGreen}x\) and obtaining a straight line can be used to check for an exponential relationship. Where a relationship is of the form \(y=ab^x\), determining the values of \(a\) and \(b\) is possible by first finding the \emph{gradient} and \(y\)\emph{-intercept} on the \(\color{PineGreen}\log{y}\sim x\) graph.

\begin{tcolorbox}[title=Example,colback=PineGreen!2!, colframe=PineGreen]

Two variables, $x$ and $y$, are connected by the equation $y=ab^x$. The graph of $\log_9{y}$ against $x$ is a straight line as shown. Determine the values of $a$ and $b$.

\vspace{-0.3cm}

\begin{center}
\begin{tikzpicture}[scale=0.8]
          \draw[-stealth] (-0.2,0) -- (4,0) node[below] {$x$};
          \draw[-stealth] (0,-0.2) -- (0,4) node[left] {$\log_9{y}$};
          \node[below left] {\scriptsize{O}};
          \draw[thick] plot (0,1) -- (3,2.5);
          \draw[fill] (0,1) circle (0.05) node[left] {$(0,2)$};
          \draw[fill] (2,2) circle (0.05) node[below right] {$(4,4)$};
\end{tikzpicture}
\end{center}

\vspace{-0.6cm}

\tcblower

\vspace{-0.5cm}

\begin{align*}
\log_{9}y&=\log_9{\left(ab^x\right)} && \color{PineGreen}\longleftarrow \text{Take }\log_9\text{ of both sides}\\[0.5em]
\log_{9}y&=\log_9{b^x}+\log_9{a} && \color{PineGreen}\longleftarrow \text{Apply product law}\\[0.5em]
\log_{9}y&=x\log_9{b}+\log_9{a} && \color{PineGreen}\longleftarrow \text{Apply power law}\\[1em]
\text{gradient}&=\frac{4-2}{4-0}=\frac{2}{4}=\frac{1}{2} && \color{PineGreen}\longleftarrow \text{Find gradient}\\[0.5em]
\log_9{b}&=\frac{1}{2} && \color{PineGreen}\longleftarrow \text{Equate to }\log_9{b}\\[0.5em]
b&=3 && \color{PineGreen}\longleftarrow \text{Solve for }b\\[1em]
y\text{-intercept}&=2 && \color{PineGreen}\longleftarrow \text{Find }y\text{-intercept}\\[0.5em]
\log_{9}a&=2 && \color{PineGreen}\longleftarrow \text{Equate to }\log_9{a}\\[0.5em]
a&=81 && \color{PineGreen}\longleftarrow \text{Solve for }a
\end{align*}

\end{tcolorbox}

Therefore the relationship between \(x\) and \(y\) is \(y=81\times3^x\).

Note the the process of solving for \(a\) and solving for \(b\) may require solving each log equation step-by-step.

\pagebreak

\newpage

\subsection*{Exercise 15.10}\label{exercise-15.10}
\addcontentsline{toc}{subsection}{Exercise 15.10}

\begin{enumerate}
    \item For each graph where the relationship between $x$ and $y$ is $y=ab^x$, determine the values of $a$ and $b$:
    \begin{multicols}{2}
        \begin{center}
            \begin{tikzpicture}
                \draw[white] (-3,-1) rectangle (6,4);
                \node[below right] at (-3,4) {(a)};
                \draw[-stealth] (-2,0) -- (5,0) node[below] {$x$};
                \draw[-stealth] (0,-1) -- (0,3.5) node[left] {$\log_5{y}$};
                \node[below left] {\scriptsize{O}};
                \draw[thick,fill] (0,1) node[left] {$2$} circle (0.05) -- (3,2) node[above] {gradient $=3$}; 
            \end{tikzpicture}
        \end{center}
        \begin{center}
            \begin{tikzpicture}
                \draw[white] (-3,-1) rectangle (6,4);
                \node[below right] at (-3,4) {(b)};
                \draw[-stealth] (-2,0) -- (5,0) node[below] {$x$};
                \draw[-stealth] (0,-1) -- (0,3.5) node[left] {$\log_4{y}$};
                \node[below left] {\scriptsize{O}};
                \draw[thick,fill] (0,0.5) node[left] {$\frac{1}{2}$} circle (0.05) -- (2.5,3) node[above] {gradient $=2$};
            \end{tikzpicture}
        \end{center}
    \end{multicols}
    \begin{multicols}{2}
        \begin{center}
            \begin{tikzpicture}
                \draw[white] (-3,-1) rectangle (6,4);
                \node[below right] at (-3,4) {(c)};
                \draw[-stealth] (-2,0) -- (5,0) node[below] {$x$};
                \draw[-stealth] (0,-1) -- (0,3.5) node[left] {$\log_2{y}$};
                \node[below left] {\scriptsize{O}};
                \draw[thick,fill] (0,2.4) node[left] {$(0,3)$} circle (0.05) -- (3,0.8) node[right] {$(2,1)$} circle (0.05);
            \end{tikzpicture}
        \end{center}
        \begin{center}
            \begin{tikzpicture}
                \draw[white] (-3,-1) rectangle (6,4);
                \node[below right] at (-3,4) {(d)};
                \draw[-stealth] (-2,0) -- (5,0) node[below] {$x$};
                \draw[-stealth] (0,-1) -- (0,3.5) node[left] {$\log_3{y}$};
                \draw[thick,fill] (0,-0.5) node[left] {$-1$} circle (0.05) -- (4,1.5) node[right] {$(2,3)$} circle (0.05);
                \node[above left] {\scriptsize{O}};
            \end{tikzpicture}
        \end{center}
    \end{multicols}
    \item For each graph where the relationship between $x$ and $y$ is $y=ka^x$, determine the values of $a$ and $k$:
    \begin{multicols}{2}
        \begin{center}
            \begin{tikzpicture}
                \draw[white] (-3,-1) rectangle (6,4);
                \node[below right] at (-3,4) {(a)};
                \draw[-stealth] (-2,0) -- (5,0) node[below] {$x$};
                \draw[-stealth] (0,-1) -- (0,3.5) node[left] {$\log_{10}{y}$};
                \node[below left] {\scriptsize{O}};
                \draw[thick,fill] (0,2) node[left] {$(0,3)$} circle (0.05) -- (2,-0.7) node[right] {gradient $=-1$};
            \end{tikzpicture}
        \end{center}
        \begin{center}
            \begin{tikzpicture}
                \draw[white] (-3,-1) rectangle (6,4);
                \node[below right] at (-3,4) {(b)};
                \draw[-stealth] (-2,0) -- (5,0) node[below] {$x$};
                \draw[-stealth] (0,-1) -- (0,3.5) node[left] {$\log_{8}{y}$};
                \node[below left] {\scriptsize{O}};
                \draw[thick,fill] (0,0.5) node[left] {$\frac{1}{3}$} circle (0.05) -- (2.5,3) node[above] {gradient $=1$};
            \end{tikzpicture}
        \end{center}
    \end{multicols}
    \begin{multicols}{2}
        \begin{center}
            \begin{tikzpicture}
                \draw[white] (-3,-1) rectangle (6,4);
                \node[below right] at (-3,4) {(c)};
                \draw[-stealth] (-2,0) -- (5,0) node[below] {$x$};
                \draw[-stealth] (0,-1) -- (0,3.5) node[left] {$\log_3{y}$};
                \node[below left] {\scriptsize{O}};
                \draw[thick,fill] (0,2.4) node[left] {$4$} circle (0.05) -- (1.2,0) node[below] {$2$} circle (0.05);
            \end{tikzpicture}
        \end{center}
        \begin{center}
            \begin{tikzpicture}
                \draw[white] (-3,-1) rectangle (6,4);
                \node[below right] at (-3,4) {(d)};
                \draw[-stealth] (-2,0) -- (5,0) node[below] {$x$};
                \draw[-stealth] (0,-1) -- (0,3.5) node[left] {$\log_{64}{y}$};
                \node[below left] {\scriptsize{O}};
                \draw[thick,fill] (0,2) node[left] {$\frac{1}{2}$} circle (0.05) -- (3.5,0) node[below] {$\frac{3}{2}$} circle (0.05);
            \end{tikzpicture}
        \end{center}
    \end{multicols}
\end{enumerate}

\newpage

\section*{Logs and Exponentials Review Exercise}\label{logs-and-exponentials-review-exercise}
\addcontentsline{toc}{section}{Logs and Exponentials Review Exercise}

\vspace{-0.5cm}

\begin{enumerate}
    \item Sketch each of the following:
    \begin{enumerate}
        \begin{multicols}{3}
            \item $y=4^x+2$
            \item $y=5^{x+2}$
            \item $\log_3{(x-2)}+4$
        \end{multicols}
    \end{enumerate}
    \item Express $2\log_p{10}+\log_p{2}-\log_p{50}$ in the form $\log_p{q}$, where $q$ is a positive integer.
    \item Evaluate the following:
    \begin{enumerate}
        \begin{multicols}{2}
            \item $\log_{3}{\frac{1}{4}}+2\log_{3}{6}$
            \item $\frac{1}{2}\log_{5}{64}-\log_{5}{40}$
        \end{multicols}
    \end{enumerate}
    \item Solve each equation: 
    \begin{enumerate}
        \begin{multicols}{2}
            \item }$\log_3{x}-\log_3{5}=\log_3{2}$.
            \item $\log_2{x}+\log_2{(x+7)}=3$
        \end{multicols}
    \end{enumerate}
    \item Given that $\log_{p}{4}+3=\log_p{32}$, find the value of $p$.
    \item A population of mice is discovered living in some woodland, and number of mice can be modelled by:
    \[M=80e^{kt}\]
    where $M$ is the estimated number of mice in the population, $t$ is the number of weeks since the population was discovered and $k$ is a constant.
    \begin{enumerate}
        \item State the estimated number of mice in the population when it was first discovered.
        \item It is estimated there will be 120 mice in the population after 3 weeks. Find the value of $k$.
        \item Calculate the time taken for the population to double in size.
    \end{enumerate}
    \item Variables $x$ and $y$ are linked by the equation $y=ax^b$. The graph of $\log_7{y}$ against $\log_7{x}$ is below:
    \begin{center}
            \begin{tikzpicture}[scale=0.8]
                \draw[-stealth] (-2,0) -- (5,0) node[below] {$\log_7{x}$};
                \draw[-stealth] (0,-1) -- (0,3.5) node[left] {$\log_7{y}$};
                \draw[thick] (0,2.4) node[left] {$(0,2)$} -- (3,0.8) -- (3.75,0.4);
                \node[above right,xshift=-0.1cm,yshift=-0.1cm] at (3,1.2) {$(3,1)$};
                \draw[fill] (0,2.4) circle (0.05);
                \draw[fill] (3,0.8) circle (0.05);
                \node[below left] {\scriptsize{O}};
            \end{tikzpicture}
        \end{center}
        Find the values of $a$ and $b$.
        \item The graph of $y=f(x)$ is shown below where $f(x)=\log_3{(x+a)}+b$.
        \begin{multicols}{2}
        \begin{center}
        \begin{tikzpicture}[scale=0.8]
                    \draw[-stealth] (-0.5,0) -- (9,0) node [above] {$x$};
                    \draw[-stealth] (0,-2) -- (0,3) node[left] {$y$};
                    \draw[smooth,thick,domain=0.1:5.5,xshift=2cm,yshift=1cm] plot (\x,{ln(\x)/ln(3)}) node[right] {$y=f(x)$};
                    \draw[fill] (3,1) circle (0.05) node[above left] {$(3,1)$};
                    \draw[fill] (5,2) circle (0.05) node[above] {$(5,2)$};
                    \draw[dashed] (2,-2) -- (2,3);
                \end{tikzpicture}
                \end{center}
                \begin{enumerate}
                    \item State the domain of $f(x)$.
                    \item Find the values of $a$ and $b$.
                    \item Skech the graph of $y=f^{-1}(x)$.
                \end{enumerate}
                \end{multicols}
\end{enumerate}

\chapterfont{\color{white}}

\chapter{Further Calculus}\label{further-calculus}

\vspace{-12cm}
\begin{center}
\begin{tikzpicture}
\draw[white] (10,5) circle (0.01);
\draw[NavyBlue,rounded corners,very thick] (1,0.2) rectangle (5,1.8);
\draw[NavyBlue,very thick] (5,1) -- (6,1);
\draw[NavyBlue,fill=NavyBlue,rounded corners] (6,1.8) rectangle (18.2,0.2);
\node[white] at (12.1,1) {\Huge{\textsc{Further Calculus}}};
\node at (3,1) {\Large{\textsc{Chapter 16}}};
\end{tikzpicture}
\end{center}

\section{Differentiating Trig Functions}\label{differentiating-trig-functions}

\newpage

\subsection*{Exercise}\label{exercise-50}
\addcontentsline{toc}{subsection}{Exercise}

\newpage

\section{Integrating Trig Functions}\label{integrating-trig-functions}

\newpage

\subsection*{Exercise}\label{exercise-51}
\addcontentsline{toc}{subsection}{Exercise}

\newpage

\section{The Chain Rule for Differentiation}\label{the-chain-rule-for-differentiation}

\newpage

\subsection*{Exercise}\label{exercise-52}
\addcontentsline{toc}{subsection}{Exercise}

\newpage

\section{\texorpdfstring{Integrating \((ax+b)^n\)}{Integrating (ax+b)\^{}n}}\label{integrating-axbn}

\newpage

\subsection*{Exercise}\label{exercise-53}
\addcontentsline{toc}{subsection}{Exercise}

\newpage

\section*{Review Exercise}\label{review-exercise-13}
\addcontentsline{toc}{section}{Review Exercise}

\chapterfont{\color{white}}

\chapter*{Challenge Problems}\label{challenge-problems}
\addcontentsline{toc}{chapter}{Challenge Problems}

\vspace{-10.5cm}

\begin{center}
\begin{tikzpicture}
\draw[white] (10,5) circle (0.01);
\draw[white,rounded corners,very thick] (1,0.2) rectangle (5,1.8);
\draw[black!60,fill=black!60,rounded corners] (6,1.8) rectangle (18.2,0.2);
\node[white] at (12.1,1) {\Huge{\textsc{Challenge Problems}}};
\end{tikzpicture}
\end{center}

The following problems \textbf{do not} represent the kind of question expected to feature in a Higher Mathematics exam, either in the way they are presented or the level of difficulty. Instead, they aim to encourage a flexible approach towards problem-solving and an understanding that the skills covered in the course have applications beyond those featured in any typical exam. \emph{Some questions may be solvable without using the skills covered in this chapter, and some questions may be unrelated to this chapter.}

\subsection*{Warm-Up Problems}\label{warm-up-problems}
\addcontentsline{toc}{subsection}{Warm-Up Problems}

\newpage

\newpage

\subsection*{Chapter 1 Problems}\label{chapter-1-problems}
\addcontentsline{toc}{subsection}{Chapter 1 Problems}

\newpage

\newpage

\subsection*{Chapter 2 Problems}\label{chapter-2-problems}
\addcontentsline{toc}{subsection}{Chapter 2 Problems}

\newpage

\newpage

\subsection*{Chapter 3 Problems}\label{chapter-3-problems}
\addcontentsline{toc}{subsection}{Chapter 3 Problems}

\newpage

\newpage

\subsection*{Chapter 4 Problems}\label{chapter-4-problems}
\addcontentsline{toc}{subsection}{Chapter 4 Problems}

\newpage

\newpage

\subsection*{Chapter 5 Problems}\label{chapter-5-problems}
\addcontentsline{toc}{subsection}{Chapter 5 Problems}

\newpage

\newpage

\subsection*{Chapter 6 Problems}\label{chapter-6-problems}
\addcontentsline{toc}{subsection}{Chapter 6 Problems}

\newpage

\newpage

\subsection*{Chapter 7 Problems}\label{chapter-7-problems}
\addcontentsline{toc}{subsection}{Chapter 7 Problems}

\newpage

\newpage

\subsection*{Chapter 8 Problems}\label{chapter-8-problems}
\addcontentsline{toc}{subsection}{Chapter 8 Problems}

\newpage

\newpage

\subsection*{Chapter 9 Problems}\label{chapter-9-problems}
\addcontentsline{toc}{subsection}{Chapter 9 Problems}

\newpage

\newpage

\subsection*{Chapter 10 Problems}\label{chapter-10-problems}
\addcontentsline{toc}{subsection}{Chapter 10 Problems}

\newpage

\newpage

\subsection*{Chapter 11 Problems}\label{chapter-11-problems}
\addcontentsline{toc}{subsection}{Chapter 11 Problems}

\newpage

\newpage

\subsection*{Chapter 12 Problems}\label{chapter-12-problems}
\addcontentsline{toc}{subsection}{Chapter 12 Problems}

\newpage

\newpage

\subsection*{Chapter 13 Problems}\label{chapter-13-problems}
\addcontentsline{toc}{subsection}{Chapter 13 Problems}

\newpage

\newpage

\subsection*{Chapter 14 Problems}\label{chapter-14-problems}
\addcontentsline{toc}{subsection}{Chapter 14 Problems}

\newpage

\newpage

\subsection*{Chapter 15 Problems}\label{chapter-15-problems}
\addcontentsline{toc}{subsection}{Chapter 15 Problems}

\newpage

\newpage

\subsection*{Chapter 16 Problems}\label{chapter-16-problems}
\addcontentsline{toc}{subsection}{Chapter 16 Problems}

\newpage

\newpage

\subsection*{End of Course Problems}\label{end-of-course-problems}
\addcontentsline{toc}{subsection}{End of Course Problems}

\newpage

\newpage

\chapterfont{\color{white}}

\chapter*{Answers}\label{answers}
\addcontentsline{toc}{chapter}{Answers}

\vspace{-10.5cm}

\begin{center}
\begin{tikzpicture}
\draw[white] (10,5) circle (0.01);
\draw[white,rounded corners,very thick] (1,0.2) rectangle (5,1.8);
\draw[black!60,fill=black!60,rounded corners] (6,1.8) rectangle (18.2,0.2);
\node[white] at (12.1,1) {\Huge{\textsc{Answers}}};
\end{tikzpicture}
\end{center}

\subsection*{Chapter 1 Answers}\label{chapter-1-answers}
\addcontentsline{toc}{subsection}{Chapter 1 Answers}

\newpage

\newpage

\subsection*{Chapter 2 Answers}\label{chapter-2-answers}
\addcontentsline{toc}{subsection}{Chapter 2 Answers}

\newpage

\newpage

\subsection*{Chapter 3 Answers}\label{chapter-3-answers}
\addcontentsline{toc}{subsection}{Chapter 3 Answers}

\newpage

\newpage

\subsection*{Chapter 4 Answers}\label{chapter-4-answers}
\addcontentsline{toc}{subsection}{Chapter 4 Answers}

\newpage

\newpage

\subsection*{Chapter 5 Answers}\label{chapter-5-answers}
\addcontentsline{toc}{subsection}{Chapter 5 Answers}

\newpage

\newpage

\subsection*{Chapter 6 Answers}\label{chapter-6-answers}
\addcontentsline{toc}{subsection}{Chapter 6 Answers}

\newpage

\newpage

\subsection*{Chapter 7 Answers}\label{chapter-7-answers}
\addcontentsline{toc}{subsection}{Chapter 7 Answers}

\newpage

\newpage

\subsection*{Chapter 8 Answers}\label{chapter-8-answers}
\addcontentsline{toc}{subsection}{Chapter 8 Answers}

\newpage

\newpage

\subsection*{Chapter 9 Answers}\label{chapter-9-answers}
\addcontentsline{toc}{subsection}{Chapter 9 Answers}

\newpage

\newpage

\subsection*{Chapter 10 Answers}\label{chapter-10-answers}
\addcontentsline{toc}{subsection}{Chapter 10 Answers}

\newpage

\newpage

\subsection*{Chapter 11 Answers}\label{chapter-11-answers}
\addcontentsline{toc}{subsection}{Chapter 11 Answers}

\newpage

\newpage

\subsection*{Chapter 12 Answers}\label{chapter-12-answers}
\addcontentsline{toc}{subsection}{Chapter 12 Answers}

\newpage

\newpage

\subsection*{Chapter 13 Answers}\label{chapter-13-answers}
\addcontentsline{toc}{subsection}{Chapter 13 Answers}

\newpage

\newpage

\subsection*{Chapter 14 Answers}\label{chapter-14-answers}
\addcontentsline{toc}{subsection}{Chapter 14 Answers}

\newpage

\newpage

\subsection*{Chapter 15 Answers}\label{chapter-15-answers}
\addcontentsline{toc}{subsection}{Chapter 15 Answers}

\newpage

\newpage

\subsection*{Chapter 16 Answers}\label{chapter-16-answers}
\addcontentsline{toc}{subsection}{Chapter 16 Answers}

\newpage

\newpage

\bibliography{book.bib,packages.bib}

\end{document}
